%%%%%%%%%%%%%%%%%%%%%%%%%%%%%%%%%%%%%%%%%
% Masters/Doctoral Thesis 
% LaTeX Template
% Version 2.5 (27/8/17)
% Modified by udo mueller (2.3.2025)
%
% This template was downloaded from:
% http://www.LaTeXTemplates.com
%
% Version 2.x major modifications by:
% Vel (vel@latextemplates.com)
%
% This template is based on a template by:
% Steve Gunn (http://users.ecs.soton.ac.uk/srg/softwaretools/document/templates/)
% Sunil Patel (http://www.sunilpatel.co.uk/thesis-template/)
%
% Template license:
% CC BY-NC-SA 3.0 (http://creativecommons.org/licenses/by-nc-sa/3.0/)
%
%%%%%%%%%%%%%%%%%%%%%%%%%%%%%%%%%%%%%%%%%

%----------------------------------------------------------------------------------------
%	PACKAGES AND OTHER DOCUMENT CONFIGURATIONS
%----------------------------------------------------------------------------------------

\documentclass[
11pt, % The default document font size, options: 10pt, 11pt, 12pt
%oneside, % Two side (alternating margins) for binding by default, uncomment to switch to one side
english, % ngerman for German
singlespacing, % Single line spacing, alternatives: onehalfspacing or doublespacing
%draft, % Uncomment to enable draft mode (no pictures, no links, overfull hboxes indicated)
%nolistspacing, % If the document is onehalfspacing or doublespacing, uncomment this to set spacing in lists to single
%liststotoc, % Uncomment to add the list of figures/tables/etc to the table of contents
%toctotoc, % Uncomment to add the main table of contents to the table of contents
%parskip, % Uncomment to add space between paragraphs
%nohyperref, % Uncomment to not load the hyperref package
headsepline, % Uncomment to get a line under the header
%chapterinoneline, % Uncomment to place the chapter title next to the number on one line
%consistentlayout, % Uncomment to change the layout of the declaration, abstract and acknowledgements pages to match the default layout
]{MastersDoctoralThesis} % The class file specifying the document structure

\usepackage[utf8]{inputenc} % Required for inputting international characters
\usepackage[T1]{fontenc} % Output font encoding for international characters

\usepackage{mathpazo} % Use the Palatino font by default

\usepackage{xurl} % Allow line breaks in urls and texttt
\urlstyle{tt} % Set url style to typewriter font
\let\oldtexttt\texttt
\renewcommand{\texttt}[1]{\nolinkurl{#1}}

% use listings package for programming languages
\usepackage{listings}
\usepackage{caption}
\DeclareCaptionType{listing}

\usepackage{pdflscape}
% \usepackage{minted}
% \usepackage{listings}
\usepackage{inconsolata}
\usepackage{xcolor}

% Package for landscape orientation
\usepackage{pdflscape}

% Setup glossaries
\usepackage[automake,acronym]{glossaries}
\makeglossaries
\loadglsentries{glossary}

% Use the bober backend with the authoryear citation style 
% (which resembles APA)
% Use the biber backend instead of bibtex whenever possible.
% This adds unicode support
\usepackage[backend=biber, style=ieee, natbib=true, citestyle=numeric-comp]{biblatex}
% TODO use IEEE ok?
% \usepackage[backend=biber,style=authoryear,natbib=true]{biblatex} 

\addbibresource{thesis.bib} % The filename of the bibliography

\usepackage[autostyle=true]{csquotes} % Required to generate language-dependent quotes in the bibliography

%----------------------------------------------------------------------------------------
%	MARGIN SETTINGS
%----------------------------------------------------------------------------------------

\geometry{
	paper=a4paper, % Change to letterpaper for US letter
	inner=2.5cm, % Inner margin
	outer=3.8cm, % Outer margin
	bindingoffset=.5cm, % Binding offset
	top=1.5cm, % Top margin
	bottom=1.5cm, % Bottom margin
	%showframe, % Uncomment to show how the type block is set on the page
}

%----------------------------------------------------------------------------------------
%	THESIS INFORMATION
%----------------------------------------------------------------------------------------

\thesistitle{Generative AI for Security Automation in Hyperscale Cloud Platforms} % Your thesis title, this is used in the title and abstract, print it elsewhere with \ttitle

\supervisor{Prof. Udo \textsc{Müller}} % Your supervisor's name, this is used in the title page, print it elsewhere with \supname

\examiner{} % Your examiner's name, this is not currently used anywhere in the template, print it elsewhere with \examname

\degree{Master of Business Information Systems} % Your degree name, this is used in the title page and abstract, print it elsewhere with \degreename

\author{Daniel \textsc{Vera Gilliard}} % Your name, this is used in the title page and abstract, print it elsewhere with \authorname

\addresses{} % Your address, this is not currently used anywhere in the template, print it elsewhere with \addressname

\subject{Business Information Systems} % Your subject area, this is not currently used anywhere in the template, print it elsewhere with \subjectname

\keywords{} % Keywords for your thesis, this is not currently used anywhere in the template, print it elsewhere with \keywordnames

\university{\href{http://www.h-ka.de}{Hochschule Karlsruhe}} % Your university's name and URL, this is used in the title page and abstract, print it elsewhere with \univname
\department{\href{https://www.h-ka.de/en/study/study-in-german/master/business-information-systems/profile}{Business Information Systems}} % Your department's name and URL, this is used in the title page and abstract, print it elsewhere with \deptname

\group{\href{http://researchgroup.university.com}{Research Group Name}} % Your research group's name and URL, this is used in the title page, print it elsewhere with \groupname

\faculty{\href{https://www.h-ka.de/en/about-hka/faculties/computer-science-and-business-information-systems/overview}{Computer Science and Business Information Systems}} % Your faculty's name and URL, this is used in the title page and abstract, print it elsewhere with \facname

\AtBeginDocument{
\hypersetup{pdftitle=\ttitle} % Set the PDF's title to your title
\hypersetup{pdfauthor=\authorname} % Set the PDF's author to your name
\hypersetup{pdfkeywords=\keywordnames} % Set the PDF's keywords to your keywords
}



\usepackage{tikz}

\usepackage{tgheros}
\renewcommand*\familydefault{\sfdefault}



\begin{document}

\include{Frontpage/Frontpage}

%----------------------------------------------------------------------------------------
%	DECLARATION PAGE
%----------------------------------------------------------------------------------------

\begin{declaration}
\addchaptertocentry{\authorshipname} % Add the declaration to the table of contents

\noindent I, \authorname,
in lieu of an oath that I have written the Master's thesis presented
here independently and exclusively using the literature and other aids
provided. The thesis has not been submitted in the same or a similar form
to any other examination authority for the award of an academic degree.

 \vskip20pt

\noindent Signed:\\
\rule[0.5em]{25em}{0.5pt} % This prints a line for the signature
 
 \vskip10pt
\noindent Date:\\
\rule[0.5em]{25em}{0.5pt} % This prints a line to write the date
\end{declaration}

\cleardoublepage


% This goes in the body of your .tex document where you want the declaration to appear.
\begin{customdeclaration}{Declaration on the Use of Generative AI}
\addcontentsline{toc}{chapter}{Declaration on the Use of Generative AI}

\noindent I, \authorname, hereby declare that generative artificial intelligence (AI) was employed as a writing assistant in the development of this manuscript. The use of these tools was exclusively for linguistic enhancement, such as refining sentence structure, correcting grammar, and improving overall style. The conceptual framework, original ideas, research methodology, data analysis, and final conclusions presented in this work are the product of my own intellectual effort. I have critically reviewed, edited, and validated all content to ensure its accuracy and originality, and I bear complete responsibility for the entirety of this thesis.

\vspace{2cm}

\noindent\makebox[2.5cm]{Signed:} \rule[0.5em]{12cm}{0.5pt}

\vspace{1.5cm}

\noindent\makebox[2.5cm]{Date:} \rule[0.5em]{12cm}{0.5pt}

\end{customdeclaration}

\cleardoublepage
\include{Frontpage/Quotation}

%----------------------------------------------------------------------------------------
%	ABSTRACT PAGE
%----------------------------------------------------------------------------------------

\begin{abstract}
\addchaptertocentry{\abstractname} % Add the abstract to the table of contents
This thesis addresses the challenge of automating cloud security policy generation using Generative AI (GenAI), focusing on the integration of Large Language Models (LLMs) with traditional static analysis in a Retrieval-Augmented Generation (RAG) framework. Motivated by the need for scalable, context-aware, and reliable security automation in complex cloud environments, the research follows a Design Science Research paradigm, progressing through literature review, conceptual framework development, prototype implementation, and empirical evaluation.

The primary contributions are: (1) a novel conceptual framework that combines the speed of static analysis with the contextual reasoning of LLMs, grounded in curated data and validated through a Human-in-the-Loop (HITL) process; and (2) a functional prototype empirically evaluated on Infrastructure-as-Code for AWS using Terraform and tfsec. Results demonstrate measurable improvements in policy efficacy, generation speed, and contextual detection quality compared to static-only baselines, while highlighting the necessity of human oversight for nuanced decision-making.

The findings advance the shift-left security paradigm by providing a blueprint for embedding automated, preventative controls early in the development lifecycle. Limitations include the narrow technology focus and prototype scope, suggesting future research directions in broader stack generalization and GenAI optimization. This work offers a practical guide for integrating GenAI into DevSecOps pipelines, bridging the gap between rapid development and robust security assurance.
\end{abstract}


\include{Frontpage/Acknowledgements}



%----------------------------------------------------------------------------------------
%	LIST OF CONTENTS/FIGURES/TABLES PAGES
%----------------------------------------------------------------------------------------

\tableofcontents % Prints the main table of contents

\listoffigures % Prints the list of figures

\listoftables % Prints the list of tables

%----------------------------------------------------------------------------------------
%	Listings of programs - neccessary if you have any program texts
%----------------------------------------------------------------------------------------
\lstlistoflistings

%----------------------------------------------------------------------------------------
%	ABBREVIATIONS - not neccessary if you do not have any not well known abbreviations
%----------------------------------------------------------------------------------------

\begin{abbreviations}{ll} % Include a list of abbreviations (a table of two columns)

\textbf{ABAC} & \textbf{A}ttribute-\textbf{B}ased \textbf{A}ccess \textbf{C}ontrol\\
\textbf{ACSC} & \textbf{A}ustralian \textbf{C}yber \textbf{S}ecurity \textbf{C}entre\\
\textbf{AI} & \textbf{A}rtificial \textbf{I}ntelligence\\
\textbf{AI RMF} & \textbf{A}rtificial \textbf{I}ntelligence \textbf{R}isk \textbf{M}anagement \textbf{F}ramework\\
\textbf{APP} & \textbf{A}ustralian \textbf{P}rivacy \textbf{P}rinciples\\
\textbf{CIA} & \textbf{C}onfidentiality, \textbf{I}ntegrity, and \textbf{A}vailability\\
\textbf{CSP} & \textbf{C}loud \textbf{S}ervice \textbf{P}rovider\\
\textbf{DTA} & \textbf{D}igital \textbf{T}ransformation \textbf{A}gency\\
\textbf{GenAI} & \textbf{Gen}erative \textbf{A}rtificial \textbf{I}ntelligence\\
\textbf{LLM} & \textbf{L}arge \textbf{L}anguage \textbf{M}odel\\
\textbf{MLOps} & \textbf{M}achine \textbf{L}earning \textbf{Op}erations\\
\textbf{MTTD} & \textbf{M}ean \textbf{T}ime \textbf{t}o \textbf{D}etect\\
\textbf{MTTR} & \textbf{M}ean \textbf{T}ime \textbf{t}o \textbf{R}esolve\\
\textbf{RAG} & \textbf{R}etrieval-\textbf{A}ugmented \textbf{G}eneration\\
\textbf{RPO} & \textbf{R}ecovery \textbf{P}oint \textbf{O}bjectives\\
\textbf{RTO} & \textbf{R}ecovery \textbf{T}ime \textbf{O}bjectives\\
\textbf{SOC} & \textbf{S}ecurity \textbf{O}perations \textbf{C}enter\\
\textbf{SRM} & \textbf{S}hared \textbf{R}esponsibility \textbf{M}odel\\
\textbf{WAF} & \textbf{W}eb \textbf{A}pplication \textbf{F}irewalls\\
\textbf{ZTA} & \textbf{Z}ero \textbf{T}rust \textbf{A}rchitecture\\

\end{abbreviations}

%----------------------------------------------------------------------------------------
%	PHYSICAL CONSTANTS/OTHER DEFINITIONS - not neccessary if you do not have any
%----------------------------------------------------------------------------------------

% \begin{constants}{lr@{${}={}$}l} % The list of physical constants is a three column table

% The \SI{}{} command is provided by the siunitx package, see its documentation for instructions on how to use it

% Speed of Light & $c_{0}$ & \SI{2.99792458e8}{\meter\per\second} (exact)\\
%Constant Name & $Symbol$ & $Constant Value$ with units\\

% \end{constants}

%----------------------------------------------------------------------------------------
%	SYMBOLS - not neccessary if you do not have any
%----------------------------------------------------------------------------------------

% \begin{symbols}{lll} % Include a list of Symbols (a three column table)

% $a$ & distance & \si{\meter} \\
% $P$ & power & \si{\watt} (\si{\joule\per\second}) \\
%Symbol & Name & Unit \\

% \addlinespace % Gap to separate the Roman symbols from the Greek

% $\omega$ & angular frequency & \si{\radian} \\

% \end{symbols}



% Print the glossaries
\printglossaries
\include{Frontpage/Dedication}


%----------------------------------------------------------------------------------------
%	THESIS CONTENT - CHAPTERS
%----------------------------------------------------------------------------------------

\mainmatter % Begin numeric (1,2,3...) page numbering

\pagestyle{thesis} % Return the page headers back to the "thesis" style

% Include the chapters of the thesis as separate files from the Chapters folder
% Uncomment the lines as you write the chapters

% Chapter 1

\chapter{Introduction}
\label{chap:introduction}

%----------------------------------------------------------------------------------------

% Define some commands to keep the formatting separated from the content 
\newcommand{\keyword}[1]{\textbf{#1}}
\newcommand{\tabhead}[1]{\textbf{#1}}
\newcommand{\code}[1]{\texttt{#1}}
\newcommand{\file}[1]{\texttt{\bfseries#1}}
\newcommand{\option}[1]{\texttt{\itshape#1}}

%----------------------------------------------------------------------------------------

\section{Motivation and Problem Statement}
\label{sec:motivation_problem}

The modern software development landscape is dominated by the widespread adoption of hyperscale cloud platforms and the practice of managing infrastructure as code (IaC). This paradigm enables organizations to build and deploy applications with unprecedented speed and agility. However, this velocity comes at a cost: the scale, complexity, and dynamic nature of these environments have outpaced the capabilities of traditional, manual security practices\cite{khanna_enhancing_2024}.

Security teams are faced with a deluge of complex IaC configurations that change daily, making manual analysis slow, error-prone, and insufficient for modern DevSecOps cycles\cite{gunathilaka_context-aware_2025}. Misconfigurations have become a leading cause of cloud security breaches, and the manual creation of preventative security policies acts as a significant bottleneck, hindering development velocity\cite{tunc_cloud_2017, fu_ai_2025}. This creates a critical security gap between the speed of development and the pace of security assurance.

This thesis confronts this challenge by investigating the convergence of Generative AI (GenAI) and cloud security. The core problem this research addresses is the need for advanced, intelligent automation that can proactively analyze IaC configurations and automatically generate precise, preventative security controls. By leveraging GenAI, this work aims to create a system that keeps pace with rapid development cycles, reduces the manual burden on security teams, and embeds security directly into the development workflow, truly shifting security left.

% Section 1.2: Clearly state what this thesis aims to achieve and the questions it seeks to answer.
% This should be derived from your work in Chapters 4 and 5.
\section{Research Objectives and Questions}
\label{sec:objectives_questions}

This thesis explores how Generative AI (GenAI) can help solve the security challenges in modern cloud environments. The research focuses on using GenAI to automate security tasks and asks the following central question:

\textit{How can Generative AI technologies be effectively leveraged to automate security operations across hyperscale cloud platforms?}

To comprehensively answer this overarching question, the research is decomposed into four distinct but interrelated sub-questions.
The first sub-question explores the effectiveness and automation of GenAI for security policy generation: \textit{How can Generative AI technologies be effectively leveraged to automate security policy generation and management across hyperscale cloud platforms?}
The second investigates the architecture and orchestration needed for trust and accuracy: \textit{What specific architectural patterns and validation mechanisms are required to ensure trust, accuracy, and effective multi-cloud orchestration in GenAI-driven security automation?}
The third seeks to determine how to measure and validate the effectiveness of this automation: \textit{How can the effectiveness of GenAI-driven security automation be quantitatively measured and validated, particularly in terms of accuracy, reliability, and efficiency gains?}
Finally, the fourth sub-question explores the role of human oversight: \textit{What is the optimal balance between automation and human oversight to maximize security outcomes and mitigate the risks of GenAI-driven policy generation?}

These questions guide the structure of the thesis. The current state of the art is examined in Chapter~\ref{chap:Background and Related Work}. The conceptual framework to address the question of architecture and orchestration is detailed in Chapter~\ref{chap:conceptual_framework}. The practical implementation and validation of the framework, which addresses the remaining questions, are described in Chapter~\ref{chap:implementation}, and the results are evaluated in Chapter~\ref{chap:results}.

By methodically addressing these questions, this thesis aims to provide a comprehensive, empirically grounded contribution to the field of automated cloud security.

\section{Contribution and Significance}
\label{sec:contribution}

This thesis delivers two primary contributions to the field of automated cloud security: a novel conceptual framework and a functional prototype that provides an empirical validation of that framework.

First, it introduces a novel conceptual framework for GenAI-driven security automation. The framework's innovation lies in its hybrid architectural model, which systematically integrates the speed of traditional static analysis with the deep contextual reasoning of a Large Language Model (LLM). It leverages a Retrieval-Augmented Generation (RAG) architecture to ground the LLM's output in curated data—mitigating inaccuracies and incorporates a multi-stage validation process with a mandatory Human-in-the-Loop (HITL) review to ensure a critical balance between automation and expert oversight.

Second, this work presents the design and implementation of a functional prototype that realizes the conceptual model. More than just a theoretical construct, the prototype is a working system built with industry-standard technologies, including Amazon Web Services (AWS), Terraform for Infrastructure-as-Code (IaC), and Open Policy Agent (OPA) for policy enforcement. This demonstrates the practical feasibility and applicability of the proposed approach in a real-world technological stack.

The significance of this work is threefold. It advances the shift-left security paradigm by providing an automated mechanism for creating preventative controls early in the development lifecycle \cite{akto_shift_2025}. It also serves as a practical blueprint for organizations seeking to engineer and integrate GenAI into their DevSecOps pipelines. Finally, by automating a traditionally manual and time-consuming task, this research helps reduce the cognitive load and operational burden on security teams, allowing them to focus on higher-value strategic initiatives and ultimately bridging the critical gap between high-velocity development and robust security assurance.

\section{Outline of the Thesis}
\label{sec:outline}

This thesis is structured into eight chapters to systematically address the research questions and validate the proposed solution. Chapter~\ref{chap:introduction} introduces the research by defining the motivation and problem statement, outlining the core research objectives and questions, and summarizing the primary contributions and significance of the work. Chapter~\ref{chap:Background and Related Work} provides the necessary theoretical foundation, covering key concepts in cloud security and Generative AI, and includes a comprehensive literature review that analyzes the current state of the art and identifies the critical research gaps that this thesis aims to address. Chapter~\ref{chap:methodology} details the research approach, justifying the use of a Design Science Research paradigm and outlining the four-phase process followed in this study: a foundational literature review, the development of the conceptual framework, the implementation of a prototype, and its empirical evaluation. Chapter~\ref{chap:conceptual_framework} presents the architectural blueprint for the proposed solution, detailing its multi-layered design, the integration of a RAG-based LLM for analysis and policy generation, and the essential role of the Human-in-the-Loop validation process. Chapter~\ref{chap:implementation} describes the practical realization of the conceptual framework, covering the technology stack, the cloud infrastructure defined as code, the Python application that orchestrates the workflow, and its integration into a CI/CD pipeline. Chapter~\ref{chap:results} presents the empirical findings from the evaluation of the prototype, assessing its performance against a defined set of metrics including policy generation accuracy and effectiveness, generation speed, and the quality of its contextual reasoning. Chapter~\ref{chap:discussion} interprets these results, analyzing the key findings and their implications, addressing the limitations of the current study, and proposing concrete directions for future research. Finally, Chapter~\ref{chap:conclusion} summarizes the entire body of work, revisits the research questions to provide final answers, and reflects on the overall contribution of the thesis to the field of automated cloud security.

% Chapter 2

\chapter{Background and Related Work} % (fold)
\label{chap:Background and Related Work}

\section{Foundational Concepts in Cloud Computing} % (fold)
\label{sec:Foundational Concepts in Cloud Computing}

TBD

% section Foundational Concepts in Cloud Computing (end)

\section{State of Cloud Provider Ecosystems} % (fold)
\label{sec:State of Cloud Provider Ecosystems}

TBD

% section State of Cloud Provider Ecosystems (end)

\section{Literature Review} % (fold)
\label{sec:Literature Review}

TBD: After main content

% TODO rewrite introduction with content done

% In this literature review seminal and recent publications addressing the automation of security operations in hyperscale cloud environments through Generative Artificial Intelligence (GenAI) are analyzed. As cloud infrastructures grow increasingly complex, the integration of advanced AI capabilities offers promising solutions for security enhancement and operational efficiency across multi-cloud deployments.

% As mentioned in TODO The application of Generative AI to cloud security represents one of the most significant technological shifts in recent years. GenAI's capabilities extend far beyond traditional rule-based security approaches, offering adaptive, intelligent responses to emerging threats. Nevertheless, it introduces significant challenges regarding security enhancements, workload distribution, and cost optimization.

% Seth, Ratra, and Sundareswaran, “AI and Generative AI-Driven Automation for Multi-Cloud and Hybrid Cloud Architectures.”

\subsubsection{Methodology} % (fold)
\label{sec:Methodology}

This literature review followed a structured approach to identify relevant publications, focusing on peer-reviewed articles addressing GenAI applications in hyperscale cloud security published primarily within the last five years. The search utilized academic databases with key search terms related to generative AI, cloud security automation, hyperscale platforms, and multi-cloud orchestration.
Papers were selected based on their relevance to:

\begin{itemize}
\item GenAI applications specifically in cloud security contexts
\item Hyperscale or multi-cloud environments
\item Technical solutions for security automation
\item Empirical evidence or theoretical frameworks with substantial methodological rigor
\end{itemize}

The selection process involved initial screening of titles and abstracts followed by full-text review of promising papers. The analysis employed a thematic approach, identifying recurring concepts, methodological approaches, and gaps in existing research. Particular attention was paid to identifying the theoretical foundations underpinning GenAI applications in security contexts, empirical evidence of effectiveness, and limitations of current approaches.

% subsubsection Methodology (end)

\subsubsection{AI-Driven Security Approaches} % (fold)
\label{sec:AI-Driven Security Approaches}

The landscape of cloud security is undergoing a significant transformation, shifting from traditional, often reactive methods towards more proactive and adaptive strategies powered by Artificial Intelligence (AI), particularly Generative AI (GenAI). This evolution marks a move beyond basic anomaly detection towards sophisticated security postures capable of learning from and responding dynamically to novel threat vectors in complex cloud environments.

Foundational work by Khanna \cite{khanna_enhancing_2024} explores the integration of GenAI into cloud security, outlining its core applications. According to Khanna, modern GenAI implementations focus on key capabilities such as processing vast amounts of data (e.g., log entries, network packets) for advanced anomaly detection and threat intelligence, enabling automated response mechanisms that dynamically adjust security protocols, and facilitating predictive security measures to forecast potential vulnerabilities. While highlighting these advancements, Khanna also acknowledges inherent challenges, including the need for large datasets and mitigating potential adversarial manipulation \cite{khanna_enhancing_2024}.

Building upon these foundational capabilities, the integration of GenAI represents a significant leap beyond conventional rule-based security systems. Research indicates that GenAI enhances security automation, particularly within multi-cloud and hybrid architectures. It allows systems to adapt infrastructure dynamically in response to varying traffic patterns and implements AI-powered defenses against continuously evolving cyber threats. This adaptive capability directly addresses persistent challenges in cloud security related to optimizing workload distribution, ensuring performance, and managing costs effectively \cite{seth_ai_2025}.

Furthermore, recent studies underscore the practical impact of integrating GenAI with established cloud-native security tools. Patel et al. \cite{patel_generative_2025} demonstrate how layering GenAI onto platforms like AWS GuardDuty and Google Cloud Security Command Center significantly boosts automated threat detection, enables real-time incident response, and improves comprehensive vulnerability management across distributed cloud infrastructures. Their work provides empirical evidence, citing examples like Netflix and JPMorgan Chase, which reported measurable improvements in detection accuracy and notable reductions in security incidents following the adoption of GenAI-driven security automation strategies \cite{patel_generative_2025}. This convergence of GenAI with existing security frameworks highlights its potential to transform security operations centers (SOCs) by enhancing both efficiency and effectiveness.

% subsubsection AI-Driven Security Approaches (end)

\subsubsection{GenAI Security Scoping Matrix} % (fold)
\label{sec: GenAI Security Scoping Matrix}

With the introduction of the "Generative AI Security Scoping Matrix," by AWS, a structured and comprehensive framework for assessing security requirements based on the type of GenAI deployment is provided\cite{noauthor_securing_nodate}. This framework aids organizations in evaluating their security posture, identifying potential vulnerabilities, and implementing appropriate controls throughout the AI lifecycle\cite{noauthor_securing_nodate}. While core security disciplines remain essential, the matrix specifically helps address the unique risks and additional considerations introduced by generative AI workloads\cite{noauthor_securing_nodate}. It classifies implementations into five scopes, representing increasing levels of ownership and control\cite{noauthor_securing_2023}:

\begin{enumerate}
    \item \textbf{Consumer apps (Scope 1):} Utilising public third-party GenAI services (e.g., public chatbots), often with no cost or paid access, where the business does not own or see the training data or model and cannot modify it. Interaction occurs via APIs or direct application use according to provider terms\cite{noauthor_securing_2023}\cite{noauthor_securing_nodate}. This falls under the "Buying" category of GenAI usage\cite{noauthor_securing_nodate}.
    \item \textbf{Enterprise apps (Scope 2):} Employing third-party enterprise applications (e.g., Amazon Q) with embedded GenAI features, involving an established business relationship with the vendor\cite{noauthor_securing_2023}\cite{noauthor_securing_nodate}. This also falls under the "Buying" category\cite{noauthor_securing_nodate}.
    \item \textbf{Pre-trained models (Scope 3):} Building custom applications that integrate with existing third-party foundation models (e.g., Amazon Bedrock base models) via APIs\cite{noauthor_securing_2023}\cite{noauthor_securing_nodate}. This represents the start of the "Building" category\cite{noauthor_securing_nodate}.
    \item \textbf{Fine-tuned models (Scope 4):} Refining an existing third-party foundation model by fine-tuning it with business-specific data, resulting in a new, specialized model (e.g., Amazon Bedrock customized models)\cite{noauthor_securing_2023}\cite{noauthor_securing_nodate}. This is part of the "Building" category\cite{noauthor_securing_nodate}.
    \item \textbf{Self-trained models (Scope 5):} Constructing and training a GenAI model from the ground up using proprietary or acquired data, implying full ownership of the model (e.g., using Amazon SageMaker)\cite{noauthor_securing_2023}\cite{noauthor_securing_nodate}. This represents the highest level of ownership in the "Building" category\cite{noauthor_securing_nodate}.
\end{enumerate}

This matrix serves as a mental model\cite{noauthor_securing_nodate} that helps security teams prioritize focus areas by identifying five key security disciplines whose requirements vary across the deployment scopes: governance and compliance, legal and privacy, risk management, controls, and resilience\cite{noauthor_securing_2023}\cite{noauthor_securing_nodate}. As organizations move across the scopes from consuming consumer apps (Scope 1) towards building self-trained models (Scope 5), the demands within these disciplines evolve significantly. For instance, governance requirements escalate from basic compliance with terms of service to comprehensive frameworks for model development and monitoring. Similarly, risk management priorities shift, potentially focusing on prompt injection for pre-trained models (Scope 3) versus data poisoning or model extraction for fine-tuned or self-trained models (Scopes 4 and 5). The necessary security controls also transition from emphasizing access policies in lower scopes to implementing technical safeguards like adversarial testing and output filtering in higher scopes. Resilience planning also adapts based on the application's criticality. By mapping their GenAI activities to this matrix, organizations can systematically assess risks and apply appropriate security measures tailored to their specific implementation context.

\subsubsection{GenAI Security Frameworks} % (fold)
\label{sec:GenAI Security Frameworks}

A notable contribution to the field is the SecGenAI framework, which provides a comprehensive approach to securing cloud-based Generative AI (GenAI) applications, with particular attention to Retrieval-Augmented Generation (RAG) systems within the context of Australian Critical Technologies of National Interest\cite{haryanto_secgenai_2024}. This framework addresses the unique security challenges introduced by the rapid advancement of GenAI technologies\cite{haryanto_secgenai_2024}.

SecGenAI is structured around three core pillars: functional, infrastructure, and governance requirements\cite{haryanto_secgenai_2024}. It integrates an end-to-end security analysis to generate detailed specifications. These specifications emphasize critical areas such as data privacy, secure deployment methodologies, and the implementation of shared responsibility models between cloud service providers and users\cite{haryanto_secgenai_2024}. The framework's development addresses key questions surrounding GenAI security, the requirements for Confidentiality, Integrity, and Availability (CIA triad), specific RAG implementation options, constraints within the Australian regulatory landscape, and alignment with ethical principles\cite{haryanto_secgenai_2024}.

A key aspect of SecGenAI is its alignment with established Australian guidelines, including the Australian Privacy Principles (APP) \cite{resources_australias_2024}, particularly APP 11 concerning the security of personal information, Australia's AI Ethics Principles \cite{resources_australias_2024}, and guidance from the Australian Cyber Security Centre (ACSC) and the Digital Transformation Agency (DTA)\cite{haryanto_secgenai_2024}. This alignment ensures that security measures incorporate regulatory compliance without hindering operational efficiency\cite{haryanto_secgenai_2024}. The framework specifically targets GenAI-specific threats that often evade traditional security countermeasures. These threats include data leakage (potentially revealing sensitive training or knowledge base data), various adversarial attacks (such as prompt injection, data poisoning, and model inversion techniques designed to extract underlying data or manipulate outputs), jailbreaking attacks (bypassing content restrictions), and the potential for GenAI to generate insecure code or be misused by threat actors\cite{haryanto_secgenai_2024}.

SecGenAI proposes a multi-layered security strategy combining advanced machine learning techniques with robust security measures.

Functional Security Requirements:
\begin{itemize}
    \item \textbf{Identity and Access Management:} Utilizing continuous and adaptive authentication mechanisms, alongside Attribute-Based Access Control (ABAC), to manage user access dynamically based on behavior and context\cite{haryanto_secgenai_2024}.
    \item \textbf{Data Confidentiality and Integrity:} Employing techniques like homomorphic encryption to process encrypted data, data masking and tokenization to protect sensitive information, and data integrity verification using methods like hashing and artificial fingerprinting\cite{haryanto_secgenai_2024}.
    \item \textbf{Model Security:} Implementing adversarial attack mitigation, encrypting model parameters, and ensuring secure model training protocols, potentially using differential privacy or federated learning\cite{haryanto_secgenai_2024}.
\end{itemize}

Infrastructure Security Requirements:
\begin{itemize}
    \item Implementing sandboxed environments (e.g., using containerization or virtualization) often within dedicated cloud availability zones and virtual private networks\cite{haryanto_secgenai_2024}.
    \item Securing database connections using read replicas, strict IAM policies, and robust encryption methods (e.g., AWS KMS or CloudHSM)\cite{haryanto_secgenai_2024}.
    \item Establishing stringent network security settings, segregating internal and external connections, and utilizing security groups\cite{haryanto_secgenai_2024}.
    \item Deploying external attack prevention mechanisms like Web Application Firewalls (WAF) and DDoS mitigation services, potentially integrated with data processing and monitoring tools (e.g., AWS Kinesis, Glue, Athena, Grafana)\cite{haryanto_secgenai_2024}.
    \item Ensuring robust data backup and disaster recovery strategies, considering Recovery Time Objectives (RTO) and Recovery Point Objectives (RPO)\cite{haryanto_secgenai_2024}.
\end{itemize}

Governance Requirements:
\begin{itemize}
    \item Adhering to AI governance principles (based on the ISO 38500 Evaluate-Direct-Monitor cycle \cite{noauthor_isoiec_nodate}) covering fairness, accountability, content traceability (e.g., watermarking), data protection, regular audits, reliability, user consent, third-party risk management, transparency, and legal compliance\cite{haryanto_secgenai_2024}.
    \item Clearly defining roles and responsibilities through an AI-specific Shared Responsibility Model (SRM) for cloud environments, outlining obligations for Cloud Service Providers (CSPs), customers, and areas of shared duty\cite{haryanto_secgenai_2024}.
\end{itemize}

By addressing these functional, infrastructure, and governance aspects in detail, the SecGenAI framework provides actionable strategies for the secure implementation and operation of cloud-based GenAI systems, fostering innovation while safeguarding national interests and enhancing the overall reliability and trustworthiness of these transformative technologies\cite{haryanto_secgenai_2024}.

As organizations increasingly adopt multi-cloud strategies, effective policy orchestration across diverse environments becomes critical for maintaining consistent security postures and operational efficiency\cite{sushil_prabhu_prabhakaran_integration_2024}. A comprehensive analysis of unified AI and cloud platforms published in 2024 examines architectural frameworks and integration patterns that enable the convergence of AI tools, machine learning operations (MLOps), data processing systems, and workflow orchestration within cloud-native environments, primarily focusing on transforming process automation and decision systems\cite{sushil_prabhu_prabhakaran_integration_2024}. This research investigates how these unified platforms address challenges like managing distributed AI workloads, ensuring real-time processing, and maintaining regulatory compliance\cite{sushil_prabhu_prabhakaran_integration_2024}.

This research identifies three key innovations within these unified platforms with significant implications for advanced automation, including security automation:
\begin{itemize}
    \item \textbf{Federated AI implementations:} These allow organizations to train AI models across distributed nodes or cloud boundaries while preserving data sovereignty and privacy, as sensitive data does not need to be centralized. This approach also reduces network bandwidth requirements and utilizes techniques like secure aggregation protocols and differential privacy\cite{sushil_prabhu_prabhakaran_integration_2024}.
    \item \textbf{Real-time data processing architectures:} Leveraging advanced streaming technologies (like Apache Kafka or Flink), in-memory processing, and real-time analytics engines, these architectures enable sub-second decision-making and immediate response to events (such as potential threats) by efficiently handling continuous data streams with high reliability and fault tolerance\cite{sushil_prabhu_prabhakaran_integration_2024}.
    \item \textbf{Multi-cloud integration patterns:} These patterns establish standardized interfaces, communication protocols, service discovery mechanisms, load balancing, and security controls to ensure seamless and consistent operation, including policy enforcement, across different cloud providers. Hybrid cloud deployment strategies are also highlighted, intelligently distributing workloads between on-premise and cloud resources based on performance, cost, and compliance needs, managed by sophisticated orchestration\cite{sushil_prabhu_prabhakaran_integration_2024}.
\end{itemize}

These architectural approaches, integrating MLOps frameworks for lifecycle management, robust data processing, workflow orchestration, and advanced capabilities like federated learning and real-time analytics within a multi-cloud context, provide the necessary foundation upon which advanced solutions like GenAI-driven security automation can be built across heterogeneous cloud environments\cite{sushil_prabhu_prabhakaran_integration_2024}. While the analyzed paper focuses broadly on process automation and decision systems, the detailed exploration of scalable, resilient, governable, and interoperable architectures directly supports the implementation of sophisticated, automated security measures in complex multi-cloud settings\cite{sushil_prabhu_prabhakaran_integration_2024}.

For organizations operating containerized workloads across multiple clusters, particularly in multi-domain architectures involving different administrative entities, research from 2023 proposes an automated approach for generating network security policies in Kubernetes deployments\cite{bringhenti_security_2023}. Manually configuring security in such environments is complex, often leading to inconsistencies between policies defined in different clusters and requiring domain administrators to possess knowledge about other domains' configurations (like service locations or IP addresses), which is not always feasible\cite{bringhenti_security_2023}. This approach addresses two critical challenges in multi-cluster security: reducing the configuration errors commonly made by human administrators and creating transparent cross-cluster communications without requiring extensive information sharing between domains\cite{bringhenti_security_2023}.

The proposed solution involves a top-level entity named the "Multi-Cluster Orchestrator"\cite{bringhenti_security_2023}. This orchestrator acts as a central management point, receiving inputs from managers of different domains\cite{bringhenti_security_2023}. These inputs include:
\begin{itemize}
    \item A description of each domain's structure (listing clusters and exposed services with their details)\cite{bringhenti_security_2023}.
    \item High-level security requirements specifying allowed communications (e.g., between services within the same domain, services in different domains, or services and external IP addresses)\cite{bringhenti_security_2023}. These requirements can be defined using an extended YAML syntax with special labels (`service`, `cluster`, `domain`) that abstract away low-level details\cite{bringhenti_security_2023}.
\end{itemize}

Based on these inputs, the Multi-Cluster Orchestrator refines the high-level requirements into concrete configurations through a two-step process\cite{bringhenti_security_2023}:
\begin{enumerate}
    \item It generates a "Global Configuration" that tracks communication pairs between services and required links between clusters, optimizing the overall cluster mesh setup\cite{bringhenti_security_2023}.
    \item It derives "Single Configurations" for each individual cluster, containing the specific parameters needed to connect the cluster to the mesh (e.g., using technologies like Cilium Cluster Mesh), the Kubernetes Network Policies to enforce the desired security rules, and commands to create local service entries that enable transparent name resolution for services located in external clusters\cite{bringhenti_security_2023}.
\end{enumerate}

The implementation, known as Multi-Cluster Orchestrator (developed in Java with REST APIs), demonstrates how automated policy generation can improve security consistency across distributed environments while reducing the cognitive load on security administrators by handling the complexity of multi-domain interactions transparently\cite{bringhenti_security_2023}. This research is particularly relevant for hyperscale cloud platforms and organizations that utilize container orchestration technologies like Kubernetes to manage numerous workloads across multiple clusters, potentially spanning different regions, availability zones, or administrative boundaries\cite{bringhenti_security_2023}.

Another approach to security automation in the context of policies involves the use of digital twins for validating security policies before deployment in production environments\cite{hammar_digital_2023}. This approach utilizes an emulation system specifically designed to create high-fidelity digital replicas of target IT infrastructures\cite{hammar_digital_2023}. These digital twins replicate key functionalities of the corresponding physical or virtual systems, allowing security teams to play out complex security scenarios, such as intrusion attempts and defense responses, within a safe and controlled environment\cite{hammar_digital_2023}. This capability avoids impacting operational workflows on the real-world infrastructure\cite{hammar_digital_2023}.

The digital twin approach, as detailed in the research by Hammar and Stadler, facilitates a closed-loop learning process for developing and refining security policies\cite{hammar_digital_2023}. The process involves several key steps:
\begin{enumerate}
    \item \textbf{Create Digital Twin:} A digital twin of the target infrastructure is generated using an emulation system built on virtualization technologies like Docker containers, virtual links, and virtual switches\cite{hammar_digital_2023}.
    \item \textbf{Run Scenarios \& Collect Data:} Security scenarios, involving emulated attackers, defenders, and client populations, are executed within the digital twin\cite{hammar_digital_2023}. During these runs, detailed system measurements and logs are collected via monitoring agents that push metrics to data pipelines (e.g., using Kafka and Spark)\cite{hammar_digital_2023}.
    \item \textbf{Model \& Learn:} The collected data and statistics are used to instantiate simulations, often modeled as Markov decision processes \cite{hammar_digital_2023}. Reinforcement learning techniques are then applied to these simulations to learn potentially optimal security policies\cite{hammar_digital_2023}.
    \item \textbf{Validate \& Iterate:} The performance of the learned policies is then rigorously evaluated back within the high-fidelity digital twin environment\cite{hammar_digital_2023}.
\end{enumerate}

This methodology provides continuous, iterative feedback and improvement cycles, as the results from validation can inform further scenario runs and learning phases, enhancing policy effectiveness over time\cite{hammar_digital_2023}. The authors demonstrate this by applying the approach to an intrusion response scenario, showing that the digital twin provided the necessary evaluative feedback to learn near-optimal policies that outperformed baseline systems like the SNORT IDPS\cite{zhou_study_2010}. This represents a significant advancement in validation mechanisms, particularly relevant for potentially complex GenAI-driven security automation strategies, by bridging the gap between simulation-based learning and real-world applicability\cite{hammar_digital_2023}.

Regarding policies, ensuring the trustworthiness and accuracy of GenAI-generated security policies and responses remains a significant challenge. The already mentioned SecGenAI framework demonstrates how advanced machine learning techniques can be combined with robust security measures to enhance the reliability of GenAI systems while maintaining compliance with regulatory requirements.\cite{haryanto_secgenai_2024}
As described, this approach integrates continuous validation processes throughout the AI lifecycle, from model development to deployment and monitoring, creating multiple checkpoints that verify the integrity and effectiveness of security responses. By emphasizing explainability alongside accuracy, the framework addresses one of the primary concerns associated with GenAI applications in security contexts: the "black box" nature of complex models.\cite{haryanto_secgenai_2024}

While not specifically focused on cloud security, research on GenAI applications in the energy sector offers transferable insights into implementation approaches for complex operating environments. This comprehensive literature review identifies how GenAI enhances productivity through data creation, forecasting, optimization, and natural language understanding, while also addressing challenges such as hallucinations, data biases, privacy concerns, and system errors \cite{surathunmanun_exploring_2024}.
The proposed solutions—including improving training data quality, implementing system fine-tuning processes, establishing human oversight mechanisms, and deploying robust security measures—provide a valuable framework for GenAI implementations in cloud security contexts. These approaches are particularly relevant for hyperscale environments where scale and complexity amplify both the benefits and risks of GenAI adoption \cite{surathunmanun_exploring_2024}.

% subsubsection GenAI Security Frameworks (end)

\subsubsection{Agent-Based Approaches} % (fold)
\label{sec:Agent-Based Approaches}

A recent paper from 2024 introduces and validates the concept of employing Generative AI (GenAI)-driven agentic workflows to achieve comprehensive security automation, particularly in complex modern environments. A notable example is the DevSecOps Sentinel system\cite{noauthor_devsecops_nodate}, specifically designed to address the mounting security challenges inherent in modern software supply chains. Challenges coming from microservices, containerization, and cloud-native architectures that often outpace traditional DevSecOps practices\cite{noauthor_devsecops_nodate}.

The DevSecOps Sentinel system exemplifies this approach by utilizing intelligent agents integrated into automated workflows. These agents are powered by advanced GenAI models, such as Large Language Models (LLMs) enhanced with Retrieval-Augmented Generation (RAG), enabling sophisticated analysis capabilities\cite{noauthor_devsecops_nodate}. Key characteristics of these agents include:

\begin{itemize}
    \item \textbf{Autonomy:} Operating independently based on predefined goals and policies.
    \item \textbf{Reactivity:} Responding in real-time to environmental changes like new vulnerability disclosures.
    \item \textbf{Proactivity:} Taking initiative, such as preemptively scanning for risks or suggesting improvements\cite{noauthor_devsecops_nodate}.
\end{itemize}

These agents execute critical security tasks throughout the software development lifecycle, including:

\begin{itemize}
    \item \textbf{Automated Vulnerability and Impact Analysis:} Leveraging GenAI to analyze code, dependencies (tracked via SBOMs), and infrastructure configurations for potential threats, assessing their potential impact in context\cite{noauthor_devsecops_nodate}.
    \item \textbf{Adaptive Compliance and Release Gating:} Enforcing security policies and compliance requirements dynamically, acting as automated checks before deployment\cite{noauthor_devsecops_nodate}.
    \item \textbf{Predictive Security:} Utilizing AI to identify potential future risks based on historical data and emerging threat patterns\cite{noauthor_devsecops_nodate}.
\end{itemize}

The implementation and testing of DevSecOps Sentinel demonstrate several key points relevant to broader security automation:

\begin{enumerate}
    \item \textbf{Viability for Complexity:} Agentic workflows powered by GenAI are shown to be a viable and effective method for tackling the intricate and rapidly evolving security issues found in modern, distributed systems\cite{noauthor_devsecops_nodate}.
    \item \textbf{Synergy of AI and Agents:} The integration of GenAI's deep analysis capabilities with the autonomous, proactive nature of agentic systems offers a powerful paradigm for strengthening organizational security posture\cite{noauthor_devsecops_nodate}. While Sentinel focuses on the supply chain, the principle applies broadly to automating security operations in complex cloud environments.
    \item \textbf{Measurable Improvements:} Such systems can contribute to building and deploying software that is simultaneously faster, safer, and more reliable. The DevSecOps Sentinel study reported significant quantitative improvements in key security and operational metrics, including reduced Mean Time to Detect (MTTD) and Resolve (MTTR) for vulnerabilities, lower false positive rates, increased compliance pass rates, higher deployment frequency, and reduced change failure rates\cite{noauthor_devsecops_nodate}.
\end{enumerate}

This approach, exemplified by DevSecOps Sentinel, highlights a promising direction for leveraging GenAI to automate and enhance security functions, moving beyond traditional limitations to offer more adaptive, context-aware, and efficient security management in demanding environments like hyperscale clouds.

% subsubsection Agent-Based Approaches (end)

\subsubsection{GenAI for Infrastructure Security} % (fold)
\label{sec:GenAI Security Infrastructure}

TBD

% The evolution of security infrastructure to accommodate GenAI capabilities is evident in research on next-generation firewalls that incorporate machine learning and generative modeling for enhanced threat detection. These advanced systems integrate security controls and protocols at Layer 7 of the OSI model, representing a significant leap forward in perimeter security technology for cloud environments. This integration enables the detection of sophisticated attack patterns that traditional signature-based approaches might miss.

% Patel et al., “Generative AI for Automated Security Operations in Cloud Computing.”

% Despite the transformative potential of GenAI in cloud security, research published in 2023 indicates that organizations often implement these technologies without adequately assessing potential security vulnerabilities. This observation underscores the need for balanced approaches that embrace technological advancement while implementing robust security practices and governance frameworks.

% Lekkala, “Next-Gen Firewalls.”

% subsubsection GenAI Security Infrastructure (end)

\subsubsection{Trust Challenges} % (fold)
\label{sec:Trust Challenges}

TBD

% A significant challenge in implementing GenAI for security automation relates to the "black box" nature of many large language models (LLMs). Recent research addresses this concern through the Zero-Trust Architecture (ZTA) framework, which is specifically designed to address trust issues with GenAI models.

% The ZTA approach acknowledges that GenAI models present unique challenges due to their opaque feature lists and multimodal capabilities. This framework is built on zero-trust principles intended to prevent data breaches, enhance privacy, and restrict internal lateral movement within enterprise environments.

% “Zero-Trust Architecture (ZTA): Designing an AI-Powered Cloud Security Framework for LLMs’ Black Box Problems | Semantic Scholar.”

% % subsubsection Trust Challenges (end)


\subsubsection{Privacy and Regulatory compliance} % (fold)
\label{sec:Privacy and Regulatory compliance}

TBD

% A significant challenge in implementing GenAI for security automation is ensuring compliance with evolving regulatory frameworks. Research indicates the need for clear guidelines regarding intellectual property ownership rules, particularly concerning AI-created works and the legal status of data used to train AI models.

% The Association for Computing Machinery (ACM) urges that personal data used to generate information or train models should be subject to opt-out policies, and AI creators should maintain records of errors made by their systems to ensure transparency about accuracy and correctability.

% Tabassi, “Artificial Intelligence Risk Management Framework (AI RMF 1.0).”

% The literature review on GenAI in the energy sector identifies key challenges that are equally applicable to cloud security implementations: hallucinations (generating plausible but incorrect information), data biases that affect model outputs, privacy concerns related to training data, potential for misuse, and system errors that may propagate through automated processes.
% Addressing these challenges requires comprehensive approaches to data governance, model validation, and continuous monitoring that ensure GenAI systems operate within acceptable parameters. These considerations are particularly important in security contexts, where false positives or missed detections can have significant consequences for organizational risk posture.

% Surathunmanun, Ongsakul, and Singh, “Exploring the Role of Generative Artificial Intelligence in the Energy Sector.”

% subsubsection Privacy and Regulatory compliance (end)

\subsubsection{Security Risks} % (fold)
\label{sec:Security Risks}

TBD

% Research by Khanna et al. identifies several cybersecurity risks arising from the use of GenAI, including:

% Phishing attacks and social engineering
% Ransomware and malware generation
% Deepfakes and misinformation
% Data leakage and misuse of personal data
% Executable attack code generation
% Privacy risks and intellectual property violations

% These findings provide critical insights into potential threats from irresponsible use of GenAI and emphasize the need for risk mitigation efforts and regulations concerning ethical use.

% Nyoto, Devega, and Nyoto, “Cyber Security Risks in the Rapid Development of Generative Artificial Intelligence.”

% Implementing GenAI across hyperscale cloud environments introduces additional challenges related to model distribution, data synchronization, and consistent policy enforcement across regions and services. While not explicitly addressed in all the available research, these scaling considerations represent significant technical hurdles for organizations operating at hyperscale.
% The cautionary note about adopting GenAI without carefully considering potential security vulnerabilities underscores the need for comprehensive risk assessment and gradual implementation approaches that allow organizations to identify and address issues before full-scale deployment

% Weedon, “Generative AI.”

% subsubsection Security Risks (end)

\subsubsection{Balance of Automation and Human Oversight} % (fold)
\label{sec:Balance of Automation and Human Oversight}

TBD

% A recurring theme in the literature is the tension between the benefits of automation and the necessity of human oversight. While AI-powered security automation safeguards against evolving cyber dangers, research suggests that human expertise remains essential, particularly for high-risk AI deployments.

% Seth, Ratra, and Sundareswaran, “AI and Generative AI-Driven Automation for Multi-Cloud and Hybrid Cloud Architectures.”

% The ACM explicitly states that no "high-risk" AI should be operated without substantial human oversight and careful deliberation over whether benefits outweigh potential negative impacts.

% Tabassi, “Artificial Intelligence Risk Management Framework (AI RMF 1.0).”

% Despite the promising applications of GenAI for security automation, significant challenges remain in balancing automation with appropriate human oversight. Research on GenAI for automated security operations highlights issues such as over-dependence on AI tools, adversarial risks to models, and the complex nature of decision-making in AI systems.
% The study emphasizes the importance of preventive efforts and planned action plans to manage these technologies efficiently, recognizing that complete automation without human intervention introduces unacceptable risks in security contexts. This balanced approach acknowledges the complementary strengths of human expertise and AI capabilities in addressing complex security challenges.

% Patel et al., “Generative AI for Automated Security Operations in Cloud Computing.”

% subsubsection Balance of Automation and Human Oversight (end)

% section Literature Review (end)

\section{Research Gaps} % (fold)
\label{sec:Research Gaps}

% Enhanced validation mechanisms: Developing more robust techniques for verifying the accuracy and reliability of GenAI security decisions, moving beyond current red-teaming approaches

% Feffer et al., “Red-Teaming for Generative AI.”

% Cross-platform orchestration: Creating unified frameworks for consistent security policy application across diverse cloud environments

% Vootkuri, “Multi-Cloud Data Strategy Security for Generative AI.”

% Domain-specific LLMs for security: Exploring purpose-built language models optimized for security applications rather than general-purpose models

% Energy-efficient security operations: Developing approaches that balance computational demands with sustainability concerns, particularly for inference operations

% Multi-disciplinary approaches: Bridging the gap between scientific developments and ethical considerations through collaborative research involving computer science, law, ethics, and policy-making experts

% Yigit et al., “Review of Generative AI Methods in Cybersecurity.”

% Standardized Evaluation Frameworks
% The analysis of current literature reveals a significant need for standardized frameworks to evaluate the effectiveness and security of GenAI-driven automation in hyperscale cloud environments. Future research should focus on developing metrics and methodologies that enable consistent assessment of GenAI implementations across different cloud providers and security contexts.
% Hybrid Security Approaches
% Promising directions for future research include the investigation of hybrid approaches that combine GenAI with traditional security methods to leverage the strengths of both paradigms. These hybrid models could provide the adaptability and pattern recognition capabilities of GenAI while maintaining the explainability and predictability of rule-based systems in critical security functions.

% Explainable AI for Security Operations
% Research on explainable AI approaches specifically tailored to security operations could increase transparency and trust in GenAI-generated security policies and decisions. This focus area is particularly important for regulatory compliance and stakeholder confidence in automated security systems.

\subsubsection{Summary Literature review} % (fold)
\label{sec:Summary Literature review}

This literature review demonstrates that Generative AI (GenAI) represents a transformative technology for security automation within hyperscale cloud environments. The analysis reveals significant potential for GenAI to enhance security operations through automated threat detection, policy generation, and incident response, particularly across complex multi-cloud settings. Research highlights notable advancements in conceptual frameworks for multi-cloud policy orchestration, validation mechanisms to ensure trust and accuracy, and technical approaches for implementing GenAI at scale. The most promising strategies often leverage multi-cloud architectures, zero-trust principles, and comprehensive security frameworks, while necessarily acknowledging the unique infrastructure requirements of GenAI itself. However, despite this progress, persistent challenges related to trust, validation, data privacy and quality, and the crucial balance between automation and human oversight remain significant considerations. As this field continues its rapid evolution, interdisciplinary collaboration will be essential to develop robust ethical norms and innovative defense mechanisms, addressing current issues while guiding the responsible application of GenAI in cybersecurity.

% subsubsection Summary Literature review (end)

% section Research Gaps (end)

% chapter Background and Related Work (end)
 
\chapter{Research Methodology}
\label{chap:methodology}

This chapter outlines the research methodology employed to answer the research questions posed in Chapter \ref{chap:introduction}. The research is centered on the development and evaluation of a novel solution for automating security in hyperscale cloud platforms. Therefore, a **Design Science Research (DSR)** approach was adopted as the guiding paradigm for this thesis.

DSR is a problem-solving paradigm that seeks to create and evaluate innovative artifacts intended to solve identified organizational problems. This chapter first introduces DSR and justifies its suitability for this project. It then details the specific research process followed, from problem identification through to the evaluation of the developed prototype.

\section{Research Paradigm: Design Science}
\label{sec:research_paradigm}

Design Science Research is fundamentally concerned with creating and evaluating IT artifacts (constructs, models, methods, or instantiations) that address real-world problems \cite{hevner_design_2004}. Unlike natural science, which seeks to understand reality, design science aims to create new and purposeful artifacts to extend human and organizational capabilities.

This paradigm is particularly well-suited for this thesis for several reasons:
\begin{itemize}
    \item \textbf{Problem-Centered:} The research is motivated by a practical and pressing problem—the need for scalable and automated security policy generation in complex cloud environments.
    \item \textbf{Artifact-Oriented:} The primary contributions of this thesis are tangible artifacts: the conceptual framework for GenAI-driven security automation (Chapter \ref{chap:conceptual_framework}) and its implementation as a software prototype (Chapter \ref{chap:implementation}).
    \item \textbf{Evaluation-Focused:} DSR emphasizes the rigorous evaluation of the artifact against defined criteria. This aligns with the research questions concerning the effectiveness, accuracy, and reliability of the proposed solution, which are addressed in Chapter \ref{chap:results}.
\end{itemize}

This research follows the DSR process model proposed by Peffers et al. \cite{peffers_design_2007}, which provides a structured approach for conducting and presenting design science research.

\section{Research Process}
\label{sec:research_process}

The research was conducted in several sequential phases, corresponding to the activities in the DSR process model. This process ensures a logical and rigorous progression from understanding the problem to demonstrating a viable solution.

\subsection{Phase 1: Problem Identification and Motivation}
\label{subsec:problem_identification}
The initial phase involved identifying and defining the research problem. As detailed in Chapter \ref{chap:introduction}, the increasing complexity of hyperscale cloud platforms, combined with the speed of DevOps cycles, creates significant challenges for manual security management. This problem was identified through a preliminary review of industry practices and academic literature, establishing the motivation for developing an automated solution.

\subsection{Phase 2: Definition of Objectives for a Solution}
\label{subsec:objectives_definition}
Based on the problem, a set of objectives for a solution was defined. These objectives were formulated as the research questions in Section \ref{sec:objectives_questions}. The primary goal was to explore how Generative AI could be leveraged to create an automated, accurate, and trustworthy security policy generation system. This phase involved a comprehensive literature review (Chapter \ref{chap:Background and Related Work}) to understand the state-of-the-art and identify existing gaps.

\subsection{Phase 3: Design and Development}
\label{subsec:design_development}
This phase focused on the creation of the research artifact. The design and development process was twofold:
\begin{enumerate}
    \item \textbf{Conceptual Framework:} First, a conceptual framework was designed to provide a high-level architectural blueprint for a GenAI-driven security automation system. This framework, presented in Chapter \ref{chap:conceptual_framework}, outlines the necessary components, layers, and workflows required to address the research objectives.
    \item \textbf{Prototype Implementation:} Second, the conceptual framework was instantiated as a functional software prototype. The implementation, detailed in Chapter \ref{chap:implementation}, serves as a proof-of-concept and provides the means for empirical evaluation. It uses a specific technology stack (\gls{aws}, \gls{terraform}, Python, etc.) to realize the proposed architecture.
\end{enumerate}

\subsection{Phase 4: Demonstration and Evaluation}
\label{subsec:demonstration_evaluation}
The final two phases of the DSR process involve demonstrating the artifact's use and evaluating its performance.
\begin{itemize}
    \item \textbf{Demonstration:} The prototype is used to demonstrate how it can solve the identified problem. This involves running the system with sample \gls{iac} files to automatically generate security policies, as will be shown in the evaluation chapter.
    \item \textbf{Evaluation:} The performance of the prototype is systematically evaluated against the objectives defined in Phase 2. As detailed in Chapter \ref{chap:results}, this involves measuring the quality and efficacy of the generated policies, the speed of the process, and the overall effectiveness of the hybrid analysis model. The evaluation results provide the empirical evidence needed to answer the research questions.
\end{itemize}

By following this structured DSR methodology, the thesis ensures that the proposed solution is not only theoretically sound but also practically validated, thereby providing a rigorous and relevant contribution to the field.

\chapter{Conceptual Framework for GenAI-Driven Security Automation}
\label{chap:conceptual_framework}

\section{Architectural Overview of the Proposed Framework}
\label{sec:architectural-overview}

This chapter introduces the conceptual framework designed to address the critical challenges of automated security analysis and policy generation for cloud infrastructure. The proposed architecture presents a comprehensive, multi-layered approach that systematically processes \gls{iac} artifacts and automatically generates corresponding security policies. At its core, the framework leverages the power of traditional static analysis tools and advanced \glspl{llm} to create a robust security automation pipeline.

This hybrid model is intentionally designed for efficacy, combining the reliability of established security scanners for identifying known vulnerability patterns with the contextual intelligence of generative AI \cite{khanna_enhancing_2024}. This allows for deeper analytical capabilities, necessary for uncovering complex, context-dependent security issues that traditional tools often miss \cite{akiri_generative_2025}. Furthermore, the architecture is conceived for seamless integration into modern \gls{devops} workflows, particularly \gls{cicd} pipelines, to operationalize a \gls{pac} model and enforce security throughout the development lifecycle \cite{khanna_enhancing_2024}.

As illustrated in Figure~\ref{fig:prototype-architecture}, the framework operates as a cohesive GenAI Security System triggered by external development workflows. The process begins when a \gls{cicd} pipeline initiates the workflow, providing \gls{iac} artifacts to the Ingestion Layer. This layer consumes the raw configurations and prepares them for analysis. The data is then passed to the Analysis Layer, which performs a two-stage evaluation. First, a Static Analysis component generates a baseline vulnerability report. This report is then fed into the GenAI Analysis component, which queries a curated Knowledge Base to produce an enriched, context-aware report, reducing false positives and identifying nuanced flaws.

\begin{figure}[htbp]
    \centering
    \includegraphics[width=0.95\textwidth]{Figures/component-abstract.png}
    \caption{Component Diagram of the Proposed GenAI-Driven Security Automation Framework}
    \label{fig:prototype-architecture}
\end{figure}

The enriched findings proceed to the Policy Layer, where the Policy Generator component leverages both the enriched report and the Knowledge Base to manifest a preventative Generated Policy. This artifact is immediately forwarded to a Policy Validator for automated checks. Following validation, the policy is presented to a Human Reviewer, a critical step that ensures oversight. Once approved, the now Validated Policy is committed back to the repository by the \gls{cicd} pipeline, closing the loop. This layered design provides a comprehensive and efficient system for enhancing cloud security posture by translating identified vulnerabilities directly into enforceable controls \cite{fakih_llm4cve_2025}. The following subsections will detail the specific roles and functions of each of these core layers.

\subsection{Ingestion Layer} % (fold)
\label{subsec:data-ingestion-layer}

The Data Ingestion Layer serves as the foundational entry point for security artifacts into the automation framework. Its primary function is to ingest \gls{iac} configurations, a prevalent standard for provisioning and managing cloud infrastructure. The reliance on \gls{iac}, while enhancing automation and consistency, introduces significant risks such as misconfigurations, coding errors, and embedded secrets, making automated analysis a critical requirement for secure cloud operations.

This layer is designed to support both batch and real-time ingestion modes, a flexible approach that aligns with modern data pipeline architectures emphasizing scalability and performance\cite{alevizos_towards_2024}. Batch ingestion allows for comprehensive, scheduled scans of entire code repositories, while real-time ingestion facilitates immediate analysis within \gls{cicd} pipelines \cite{gunathilaka_context-aware_2025}. The framework is designed to receive these \gls{iac} configurations via programmatic interfaces, ensuring seamless integration into existing developer workflows and automated systems.

Upon ingestion, the layer initiates a multi-stage preliminary analysis process. First, the raw \gls{iac} configuration is parsed for programmatic analysis. Following this step, a suite of established \gls{sast} tools is executed. This initial scan generates a baseline vulnerability report by checking the configurations against a comprehensive database of known misconfigurations, security vulnerabilities, and compliance violations. The structured output from this layer includes the original \gls{iac} configuration, its parsed form, and the baseline vulnerability report. These results are then passed to the Data Processing Layer. There, deeper, context-aware analysis is performed using \gls{genai}. The following section details the processes of the Data Processing Layer.

\subsection{Analysis Layer}
\label{subsec:analysis-layer}

Following the Data Ingestion Layer, the Analysis Layer is responsible for the core analysis of the ingested \gls{iac} artifacts. A central design principle of this framework is the segregation of processing activities into two distinct but complementary sub-layers: a traditional Static Analysis Engine and an advanced \gls{genai} Analysis Engine.

The rationale for this dual-layer architecture is to create a highly efficient and comprehensive security analysis pipeline. This approach leverages the respective strengths of each technology. Static analysis provides a rapid, reliable, and computationally inexpensive method for identifying a wide range of known, pattern-based vulnerabilities. By filtering out these common issues first, the framework can then employ the more resource-intensive \gls{genai} engine to focus on complex, context-dependent security flaws that traditional tools are ill-equipped to detect \cite{zhang_empirical_2024}. This layered methodology optimizes analytical depth while maintaining operational efficiency, ensuring that both well-defined and nuanced vulnerabilities are addressed \cite{khanna_enhancing_2024}.

The first stage of this layer employs a suite of established \gls{sast} tools to conduct an initial scan of the \gls{iac} configuration. This engine examines the configuration for syntactic and structural flaws by referencing curated databases of known vulnerabilities, common misconfigurations, and code smells. It validates the configuration against established security benchmarks and standards. The primary output of this stage is a baseline vulnerability report, which provides a structured list of potential issues identified through deterministic, rule-based pattern matching. This report serves as a foundational input for the subsequent, more sophisticated analysis stage.

The second stage is the \gls{genai} Analysis Engine, which represents the core innovation of this framework and directly addresses the research interest in applying generative \gls{ai} to cloud security. This engine utilizes \glspl{llm} to perform a deeper, contextual analysis that transcends the limitations of traditional static scanners \cite{noauthor_artificial_2025}. It takes as input both the original \gls{iac} configuration and the baseline vulnerability report from the previous stage, using the initial findings to enrich its analytical context \cite{noauthor_towards_2025}.

This engine is designed to identify security weaknesses that require an understanding of developer intent, architectural relationships, and complex business logic \cite{noseevich_towards_2015}. Its capabilities include:

\begin{itemize}
\item \textbf{Identifying Context-Sensitive Flaws:} Detecting risks that emerge from the interaction of multiple configurations. For example, overly permissive network rules may appear acceptable in isolation but create a vulnerability when combined with a specific resource's placement within the network architecture\cite{zhang_empirical_2024}.
\item \textbf{Uncovering Logical and Policy Violations:} Identifying logical flaws in resource deployments, such as potential circular dependencies. The engine can also detect violations of complex, unwritten organizational policies like nuanced tagging and naming conventions \cite{khanna_enhancing_2024}.
\item \textbf{Reducing False Positives:} Differentiating between genuine security risks and findings from the static analysis that are benign within a specific operational context. For example, a "hardcoded secret" may simply be a placeholder for a non-production environment.
\end{itemize}

By synthesizing information from the configuration and the initial scan, the \gls{genai} Analysis Engine bridges the gap between traditional, rule-based detection and adaptive, context-aware threat identification, producing a consolidated and enriched vulnerability report.

\subsection{Policy Layer}
\label{subsec:policy-layer}

The Policy and Validation Layer operationalizes the insights derived from the Analysis Layer, acting as the primary action-oriented component of the framework. Its purpose is to automate the creation of security artifacts—in the context of this prototype, preventative policies—using \gls{genai}. This layer directly addresses a core aspect of this research: leveraging \glspl{llm} to not only analyze but also actively generate security policies. The integration of \gls{genai} into the security architecture in this manner marks a significant shift, promising to streamline development workflows and accelerate remediation cycles.

This layer leverages \glspl{llm} to generate security policies tailored to the vulnerabilities identified in the preceding analysis stages. The generated artifacts are formal policies written in a declarative, machine-readable language, designed for automated enforcement. The \gls{llm} is guided by system prompts and a curated knowledge base of security standards to produce precise, context-aware rules. This process of generating platform-specific policies from a higher-level analysis aligns with established methods in automated systems engineering, where abstract requirements are translated into concrete, executable artifacts for a target platform\cite{fakih_llm4cve_2025}.

A critical aspect of this layer is its multi-stage validation process, which functions as an essential trust and safety mechanism. It is designed to rigorously verify the integrity, correctness, and security of the \gls{ai}-generated security policies before they are committed to a repository or presented for human review. This mitigates risks associated with \gls{ai}-generated artifacts, such as factual inaccuracies (hallucinations) or the introduction of new security flaws. Raw, unvalidated output is never trusted for deployment. The workflow includes several checkpoints:

\begin{itemize}
    \item \textbf{Automated Validation:} Generated policies undergo initial automated checks. This is a critical quality assurance step to ensure that only high-quality, effective, and secure policies proceed. The process includes:
        \begin{itemize}
            \item \textbf{Syntactic Validation:} The layer uses standard parsers and validators to confirm that the generated policy is syntactically correct and adheres to the relevant language specifications. Any policy that fails this check is immediately rejected and logged.
            \item \textbf{Security Self-Scan:} To prevent the \gls{ai} from inadvertently introducing new vulnerabilities, the generated security policy itself is subjected to a security scan using static analysis tools. This "self-scan" ensures the remediation policy does not create new security problems while attempting to solve another.
        \end{itemize}

    \item \textbf{Human-in-the-Loop Review:} The framework mandates a human-in-the-loop review process, which is indispensable for high-impact changes or when the \gls{ai} model's confidence in its output is low. This approach maintains a crucial balance between automation and human oversight, a central theme identified in the literature \cite{lim_explicate_2025}.
    
    \item \textbf{Advanced Testing:} For more accuracy, the architecture can incorporate further testing to detect subtle inconsistencies or unintended behaviors in the generated policies.
\end{itemize}

From a governance standpoint, the layer integrates robust security controls. Access controls and authentication mechanisms restrict the policy generation function to authorized entities and automated processes. Comprehensive audit logs are maintained for all generated and validated artifacts, ensuring traceability for compliance and forensic analysis, a key element in modern data architectures\cite{noauthor_testbed_2025}. Only after a generated policy successfully passes all stages of this validation gauntlet is it considered "validated." The validated policy, along with its comprehensive audit report, is then passed to the \gls{cicd} pipeline, where it can be reviewed and approved by a human expert before being enforced as part of the organization's \gls{pac} repository.

\section{Integration of GenAI-Driven Security Automation} % (fold)
\label{sub:Integration of GenAI-Driven Security Automation}

The core of the proposed security automation framework is centered around the integration of \gls{genai}, specifically through the use of \glspl{llm} accessed as a managed cloud service. This approach was deliberately chosen over deploying and managing local, open-source models for several strategic reasons. Utilizing a hyperscale cloud provider's managed \gls{ai} service offers access to powerful, state-of-the-art models without the substantial computational and financial overhead associated with self-hosting. It abstracts away the complexities of \gls{mlops}, such as infrastructure provisioning, scaling, and maintenance, allowing the focus to remain on the application logic. Furthermore, this model aligns with the Shared Responsibility Model discussed in the literature review, where the cloud provider manages the security and availability of the underlying \gls{ai} service.

To ensure the generation of accurate, contextually relevant, and reliable security policies, the framework employs a \gls{rag} architecture. This pattern is crucial for grounding the \gls{llm}'s output in factual data, thereby mitigating the risk of model "hallucinations" a significant concern in \gls{genai} systems where plausible but incorrect information may be generated \cite{noauthor_ground_nodate}. The \gls{rag} process within this framework functions as follows:

\begin{enumerate}
\item Upon receiving a vulnerability finding from the Data Processing Layer, the system queries a dedicated Knowledge Base. This knowledge base is a curated repository containing relevant security standards, vulnerability information, best practices for the given \gls{iac} technology, and documentation for the target policy language \cite{ozgur_simple_2024}.
\item The retrieved documents, which provide specific context for the detected vulnerability, are then combined with a custom System Prompt. This prompt instructs the \gls{llm} on its role, the task to be performed (e.g., "You are a security expert. Generate a precise security policy to prevent the following vulnerability"), and the required output format.
\item This enriched context, consisting of the vulnerability data, retrieved knowledge, and the system prompt, is then sent to the selected \gls{llm} via the managed service's \gls{api} to generate the security policy.
\end{enumerate}

This \gls{rag}-based approach ensures that the generated policies are not only syntactically correct but are also directly informed by authoritative and up-to-date security guidance, making the system more robust and trustworthy \cite{noauthor_ground_nodate}. By externalizing the knowledge base, the framework can be easily updated to reflect new standards or threat intelligence without needing to retrain or fine-tune the underlying \gls{llm} \cite{ozgur_simple_2024}. A high-performance foundation model is utilized for its advanced reasoning and code generation capabilities.

\section{Leveraging LLMs for Deeper Contextual Analysis} % (fold)
\label{sec:Leveraging LLMs for Deeper Contextual Analysis}

The deployment of \glspl{llm} within the security automation framework fundamentally transforms the depth and quality of \gls{iac} analysis. These models introduce a level of contextual understanding previously unattainable with traditional static analysis tools. Unlike rule-based scanners, which are limited to identifying known vulnerability patterns and syntactic misconfigurations, \glspl{llm} can synthesize information across multiple resources, configuration layers, and organizational policies. This enables them to surface nuanced, context-sensitive security issues.

By integrating \glspl{llm} with outputs from static code analyzers and a curated knowledge base, the framework is capable of identifying misconfigurations and policy violations that arise from complex interactions within the cloud environment \cite{noauthor_evaluating_2024}. This deeper insight is made possible by the \gls{llm}'s ability to reason about architectural relationships, resource dependencies, and the intent behind configurations, allowing it to detect security weaknesses that would otherwise remain hidden.

A key advantage of this approach is the identification of context-sensitive security weaknesses. The \gls{llm} is able to analyze configurations in light of their broader environment and operational context, flagging settings that may be technically valid in isolation but become risky when considered alongside other resources or data sensitivity. For example, an overly permissive network rule might not trigger an alert in a development environment, but if linked to production data or exposed to the public internet, it becomes a significant risk a nuance the \gls{llm} can discern by analyzing tags, naming conventions, and architectural metadata.

Beyond identifying outright vulnerabilities, the \gls{llm} can uncover suboptimal or inefficient configurations that deviate from best practices for performance, cost-efficiency, or resilience, tailored to the specific needs of the application. It can also interpret and enforce complex internal policies that are difficult to codify with static rules, such as intricate naming conventions, tagging strategies for governance, or architectural patterns mandated by the organization \cite{li_iris_2025}. This capability extends to spotting logical flaws in resource deployment and interconnections, such as circular dependencies, misconfigured network routing, or resource configurations that do not align with their intended purpose.

The \gls{llm}’s contextual reasoning also enables it to detect deviations from evolving best practices and industry standards by leveraging its knowledge base of security benchmarks and official documentation. This ensures that the framework remains adaptive to new vulnerability patterns and compliance requirements as they emerge. Furthermore, the \gls{llm} can analyze how combinations of individually acceptable configurations or permissions might aggregate into an elevated risk profile, identifying attack paths that arise only when multiple minor issues are considered together.

A practical example illustrates this capability: consider a storage resource that appears secure in isolation, with a restrictive access policy allowing permissions only to a specific role. The associated role is properly scoped, and the storage policy adheres to the principle of least privilege. However, a compute instance in a development environment, which has this role attached, is exposed to the internet via an open management port and is running a vulnerable operating system. While static analyzers might flag the open port and operating system vulnerability separately, and pass the storage resource and role as secure, the \gls{llm} can connect these findings. It recognizes that the instance’s exposure and vulnerability, combined with its privileged role, create a critical attack path to sensitive data in the storage resource. This context-sensitive weakness would likely be missed or deprioritized by traditional tools, but the \gls{llm}’s holistic analysis surfaces it as a high-priority risk.

In summary, leveraging \glspl{llm} for deeper contextual analysis enables the framework to move beyond pattern-based detection, offering a comprehensive understanding of security posture that accounts for the dynamic and interconnected nature of modern cloud environments. This results in the proactive identification of genuine risks, reduction of false positives, and the continuous alignment of security controls with evolving organizational and industry standards \cite{haque_sok_2025}.

\section{Metrics for Security Posture Assessment} % (fold)
\label{sec:Metrics for Security Posture Assessment}

To provide a robust and scientifically grounded evaluation of the prototype's performance, this section defines a set of quantitative metrics. These metrics are designed to assess two critical dimensions of the framework: its effectiveness in generating accurate and preventative security controls, and its operational efficiency in doing so at a speed compatible with modern DevOps workflows. By focusing on these key performance indicators, the evaluation will provide concrete evidence of the framework's ability to automate security policy generation reliably and at scale. The following subsections detail the specific metrics for Automated Policy Efficacy and Policy Generation Speed.

\subsection{Automated Policy Efficacy}
\label{sec:automated-policy-efficacy}

To provide a scientifically sound assessment of the prototype's capabilities, this section introduces the metric of Automated Policy Efficacy. Unlike a traditional vulnerability reduction metric, which would measure changes in a live environment, this metric quantifies how effectively the framework's generated policies, if applied, would prevent the vulnerabilities identified during the analysis phase. This approach provides a direct and accurate measure of the prototype's core function: the automated generation of high-quality, preventative security controls. It evaluates the tangible output of the system, offering a clear indicator of its potential to mitigate risk before deployment.

The assessment follows a ''before and after'' model, where the ''before'' state is the baseline of identified vulnerabilities and the ''after'' state is the measure of the generated policies effectiveness in addressing them.

\subsubsection*{Before: Baseline of Identified Vulnerabilities}
The process begins by establishing a clear baseline of the security risks present in a given set of Infrastructure-as-Code (IaC) configurations.
 
\begin{itemize}
    \item \textbf{Action:} An initial scan is conducted by feeding the target IaC configurations into the prototype's \textit{Data Ingestion} and \textit{Data Processing Layers}. This invokes both the Static Analysis Engine and the GenAI Analysis Engine to produce a comprehensive vulnerability report.
    \item \textbf{Record:} The total number of unique vulnerabilities detected ($V_{\text{total, initial}}$) is documented, along with their distribution across severity levels: critical ($V_c$), high ($V_h$), medium ($V_m$), and low ($V_l$). This collection represents the complete set of issues that the prototype aims to address with generated policies.
\end{itemize}

% \noindent
% \textbf{Example:} An initial scan of a new application's Terraform files identifies $V_{\text{total, initial}} = 15$ vulnerabilities, distributed as $V_c = 2$, $V_h = 5$, and $V_m = 8$. This set becomes the direct input for the \textit{Code Generation Layer}.

\subsubsection*{After: Effectiveness of Generated Policies}
Following the baseline analysis, the focus shifts to evaluating the quality and correctness of the policies produced by the prototype's \textit{Code Generation Layer} and verified by the \textit{Validation Layer}.

\begin{itemize}
    \item \textbf{Action:} For each vulnerability identified in the "before" step, the prototype generates a corresponding security policy in Rego. These policies are then passed through the automated \textit{Validation Layer}.
    \item \textbf{Measurement:} The effectiveness of the generated policies is determined using a combination of syntactic accuracy and preventative capability:
        \begin{itemize}
            \item \textbf{Policy Accuracy ($A_{\text{policy}}$):} This confirms that a generated policy is syntactically correct and well-formed. A policy must pass this check to be considered for effectiveness testing.
            \item \textbf{Policy Effectiveness ($E_{\text{policy}}$):} This measures the percentage of syntactically valid policies that, when tested in a controlled staging environment or through simulated deployment checks, successfully prevent the specific misconfiguration they were designed to address.
        \end{itemize}
\end{itemize}

\subsubsection*{Quantifiable Impact}
The ultimate measure of efficacy is the percentage of initially identified vulnerabilities for which the prototype successfully generated a syntactically correct \textit{and} effective preventive policy. This can be calculated for the total set and broken down by severity to provide a granular view of the system's performance against the most critical risks.

% \noindent
% \textbf{Example:}
% Given the initial baseline of 15 vulnerabilities (2 critical, 5 high, 8 medium), the prototype's generation and validation layers yield the following results:
% \begin{itemize}
%     \item For the 2 critical vulnerabilities, 2 effective policies were generated, resulting in a \textbf{100\% effectiveness rate for critical risks}.
%     \item For the 5 high-severity vulnerabilities, 4 effective policies were generated, achieving an \textbf{80\% effectiveness rate for high-severity risks}.
%     \item For the 8 medium-severity vulnerabilities, 7 effective policies were generated, demonstrating an \textbf{87.5\% effectiveness rate for medium-severity risks}.
% \end{itemize}

% \noindent
% This leads to the following conclusion: "The prototype successfully generated effective preventive policies for 100\% of critical, 80\% of high, and 87.5\% of medium-severity vulnerabilities identified in the initial IaC scan, demonstrating its capability to reliably automate the creation of targeted security controls." By focusing on the prevention potential of the generated artifacts, this metric provides robust, scientific evidence of the prototype's direct contribution to shifting security left in the development lifecycle.

\subsection{Policy Generation Speed}
\label{sec:policy-generation-speed}

This metric quantifies the intrinsic efficiency of the prototype's automated capabilities, focusing on the absolute performance of the GenAI-driven Code Generation Layer. Rather than comparing against a manual baseline, this metric demonstrates the high-speed performance that is fundamental to enabling modern, agile security workflows. In the context of hyperscale cloud platforms, where the ''scale and sophistication of threats necessitate advanced automation capabilities'' (as discussed in Chapter~\ref{sec:State-of-the-Art}), the ability to generate security controls rapidly is a critical enabler for maintaining a robust security posture. The value is not in being merely faster than a human, but in operating at a speed that makes ''shift left'' security and seamless CI/CD integration practical at scale \cite{fu_ai_2025}.

The argument for this automated approach is grounded in the inherent limitations of manual processes when faced with the complexity and dynamism of modern cloud environments. Manual policy creation is a resource-intensive task that introduces cognitive load on security experts and creates bottlenecks in development lifecycles, slowing deployment speed and reducing productivity \cite{gunathilaka_context-aware_2025-1, mahboob_future_2024}. Research confirms that manually integrating security into DevOps workflows can impede delivery speed, whereas AI-driven approaches promise to automate these workflows and reduce manual efforts \cite{fu_ai_2025}. By automating this function, the prototype reduces friction in the CI/CD pipeline, supports scalability, and allows human experts to shift their focus from routine generation to high-value oversight and review, as described in the Human-in-the-Loop workflow (Section~\ref{sub:Human-in-the-Loop for Review and Approval}).

Controlled studies provide compelling quantitative evidence for this acceleration. For instance, a notable experiment with GitHub Copilot, an AI pair programmer, found that professional developers with access to the AI assistant completed a complex task 55.8\% faster than the control group \cite{peng_impact_2023}. This dramatic increase in efficiency is not isolated to general coding; similar improvements in developer velocity are measured in specialized cloud engineering tasks, including the generation of IAM policies, which is directly analogous to the prototype's function \cite{kesireddy_copilot_2025}. Furthermore, research frameworks are now being developed to specifically calibrate AI performance against human baselines for software tasks, measuring the time saved on work that would take a human anywhere from one minute to over eight hours to complete \cite{rein_hcast_2025}. This ability to automate complex, time-consuming software engineering tasks with high rates of success is a key driver for adopting AI in DevOps \cite{tufano_autodev_2024}.

To measure this, the framework focuses on the time elapsed from the moment a confirmed vulnerability is passed to the Code Generation Layer until a syntactically valid and effective policy is produced and passes its initial automated validation. This is captured by two key indicators:

\begin{itemize}
    \item \textbf{Average Time per Policy ($T_{\text{gen}}$):} The core measure of speed for a single generation task.
    \item \textbf{Policy Throughput:} The number of valid policies the system can produce per minute, illustrating its capability to handle vulnerabilities at scale.
\end{itemize}

The measurement is conducted via controlled experimentation. A diverse, representative set of confirmed vulnerability inputs is run through the prototype multiple times to gather robust data. The start time ($t_{\text{start}}$) is recorded when the input reaches the Code Generation Layer, and the end time ($t_{\text{end}}$) is recorded when the validated policy is successfully output. The average generation time is then calculated using statistical analysis.

\noindent
\textbf{Calculation:} The average time, $T_{\text{gen}}$, is calculated as:
\[ T_{\text{gen}} = \frac{1}{N} \sum_{i=1}^{N} (t_{\text{end},i} - t_{\text{start},i}) \]
where $N$ is the total number of policies generated in the test set.

% \noindent
% \textbf{Example:} The prototype demonstrated an average $T_{\text{gen}}$ of \textbf{6.2 seconds per policy}, with a standard deviation of 0.8 seconds, across a test set of 100 diverse vulnerability inputs. This translates to a throughput of approximately \textbf{9.6 policies per minute}. This high-speed, automated generation capability significantly contributes to enabling rapid feedback loops within a CI/CD pipeline and drastically reduces the potential time required to create security controls for newly identified misconfigurations---a process that would otherwise be resource-intensive and prone to human delay at scale.

% subsubsection Metrics for Security Posture Assessment (end)

\section{Human-in-the-Loop for Review and Approval} % (fold)
\label{sub:Human-in-the-Loop for Review and Approval}

While the framework is designed to maximize automation, the integration of a \gls{hitl} process for review and approval is a foundational principle, reflecting a core theme identified in the literature review regarding the balance between automation and human oversight  \cite{nicosia_risk_nodate}. The complete automation of security policy generation and enforcement without human intervention introduces unacceptable risks, particularly in complex cloud environments. This subsection outlines the conceptual design of the \gls{hitl} workflow, which serves as a critical control point to ensure the safety, accuracy, and contextual appropriateness of the \gls{ai}-generated security artifacts.

The necessity for human oversight is a principle strongly articulated within established risk management frameworks, which posit that high-risk \gls{ai} systems should be operated with a meaningful human role \cite{nicosia_risk_nodate}. In this framework, the \gls{hitl} process is not merely a final checkpoint but an integrated function designed to mitigate the inherent risks of \gls{genai}, such as the generation of incorrect policies (hallucinations), the introduction of new security flaws, or the creation of overly restrictive rules that could impede business operations  \cite{nicosia_risk_nodate}. It operationalizes established governance principles by providing a mechanism to validate, override, or reject the \gls{ai}'s output before it can impact the production environment \cite{noauthor_human---loop_nodate}.

The \gls{hitl} review and approval workflow is triggered under specific, risk-informed conditions. A manual review by a qualified security engineer is mandatory for any \gls{ai}-generated policies that address high-severity or critical vulnerabilities. A review can also be triggered when the \gls{ai} model indicates a low confidence score for its generated output or when the proposed change targets a particularly sensitive component of the cloud infrastructure \cite{nicosia_risk_nodate}. This risk-based approach ensures that human expertise is focused where it is most needed, optimizing for both security and operational efficiency \cite{noauthor_human---loop_nodate}.

During the review process, the human expert is presented with a comprehensive set of information to facilitate an informed decision. This includes:
\begin{enumerate}
    \item the original vulnerability report
    \item the raw \gls{iac} snippet containing the vulnerability
    \item the \gls{ai}-generated security policy for remediation
    \item the results of automated validation checks
    \item an \gls{ai}-generated explanation of the policy’s logic and how it addresses the issue
\end{enumerate}

This curated context allows the reviewer to assess the generated artifact's accuracy, effectiveness, and potential side effects. The reviewer can then approve the policy, allowing it to proceed to the \gls{cicd} pipeline for enforcement, or reject it \cite{noauthor_human---loop_nodate}. Rejected policies are flagged and can be used as part of a feedback loop to refine the system prompts and knowledge base used by the Code Generation Layer, contributing to the system's continuous improvement. Ultimately, this symbiotic relationship between the automated capabilities of \gls{genai} and the contextual wisdom of human experts ensures that the framework operates not only with speed and scale but also with the necessary accountability and safety  \cite{noauthor_human---loop_nodate}.

% subsection Human-in-the-Loop for Review and Approval (end)

\section{Integration with CI/CD Pipelines for Policy-as-Code} % (fold)
\label{sec:Integration with CI/CD Pipelines for Policy-as-Code}

The ultimate objective of the conceptual framework is to translate its analytical outputs and \gls{ai}-generated artifacts into tangible, preventative controls that are seamlessly embedded within an organization's development lifecycle. This is achieved by integrating the framework into a \gls{cicd} pipeline, operationalizing a \gls{pac} workflow \cite{sarathe_krisshnan_jutoo_vijayaraghavan_policy_2025}. This approach embodies the "shift left" security principle, where security checks and policy enforcement are automated and moved to the earliest stages of the development process, rather than being an afterthought.

The integration follows a defined workflow, typically initiated within a version control system through a pull request. When a developer proposes changes to the cloud infrastructure by modifying \gls{iac} configurations, a \gls{cicd} pipeline is automatically triggered. This pipeline orchestrates the core functions of the framework in a sequence designed to enforce security before insecure configurations are merged:

\begin{enumerate}
\item \textbf{Automated Scanning and Analysis:} The pipeline first invokes the Data Ingestion and Data Processing layers to scan the proposed infrastructure changes. It generates a comprehensive vulnerability report, leveraging both static analysis and the deeper contextual analysis from the \gls{genai} engine.
\item \textbf{Policy Generation and Committing:} If new, unaddressed vulnerabilities are detected, the Code Generation Layer is triggered to produce the corresponding security policies. Following the \gls{hitl} review and approval process for these policies, the validated policy files are treated as code artifacts themselves. They are committed to a dedicated policy repository, ensuring they are version-controlled, auditable, and consistently applied \cite{sarathe_krisshnan_jutoo_vijayaraghavan_policy_2025}.
\item \textbf{Policy Enforcement as a Quality Gate:} The critical enforcement step is implemented using an automated policy engine as a quality gate within the \gls{cicd} pipeline \cite{noauthor_streamlining_nodate}. The pipeline uses this engine to evaluate the proposed infrastructure plan against the entire set of approved security policies. If the proposed changes violate any policies, particularly those addressing high-severity vulnerabilities, the policy evaluation fails. This failure causes the \gls{cicd} pipeline to halt and blocks the pull request from being merged. This mechanism acts as a powerful preventative control, ensuring that configurations failing to meet security standards cannot be deployed \cite{sarathe_krisshnan_jutoo_vijayaraghavan_policy_2025} \cite{noauthor_streamlining_nodate}.
\item \textbf{Metrics and Feedback Loop:} The \gls{cicd} pipeline serves as the practical execution point for capturing the metrics defined in the security posture assessment. By comparing the security scan results against the baseline, the system can quantify the effectiveness of the generated policies and the overall improvement in security posture. This data provides immediate feedback to developers on the impact of their changes and allows security teams to analyze any discrepancies, which in turn informs the refinement of scanning heuristics and the system prompts used by the \gls{genai} models.
\end{enumerate}

By integrating into the \gls{cicd} pipeline, the framework moves beyond being a mere detection tool and becomes an active participant in the development workflow. It creates a closed-loop system where vulnerabilities are automatically detected, preventative policies are generated and validated, and enforcement is programmatically guaranteed, thereby operationalizing a truly automated and responsive cloud security posture  \cite{noauthor_streamlining_nodate} 
\chapter{Implementation & System Architecture}

% TODO use abreviations from list
% TODO add references

This chapter details the design and implementation of the prototype system, a practical realization of the conceptual framework for GenAI-driven security automation introduced in Chapter 4. The work is impemented through two distinct but interconnected codebases: a cloud-native infrastructure for the generative AI backend, and a Python-based application that orchestrates the analysis and policy generation workflow.

The primary goal of this implementation is to empirically validate the central hypothesis of the theoretical framework: that a hybrid approach, combining traditional static analysis with advanced Large Language Model (LLM) capabilities, can significantly enhance the automation of security policy generation for Infrastructure-as-Code (IaC). This chapter will demonstrate how the system architecture directly maps to the four-layered conceptual model—Data Ingestion, Data Processing, Code Generation, and Validation—and realizes the core principles of leveraging Retrieval-Augmented Generation (RAG) for contextual accuracy and integrating a Human-in-the-Loop (HITL) for safety and oversight.

We will first present the high-level architecture and the technology stack chosen to satisfy the functional requirements of a robust, scalable, and reproducible security pipeline. Subsequently, the chapter will provide a detailed examination of both the cloud infrastructure, deployed via Terraform, and the Python prototype, focusing on the specific modules that implement the core logic of the system. The chapter will conclude by illustrating the end-to-end workflow, from the initial analysis of a Terraform file to the generation and validation of a corresponding Rego security policy, thereby providing a comprehensive account of the system's practical application.

\section{Design Objectives & Functional Requirements}

% Refined Research Questions:


%    * RQ1 (Effectiveness and Automation): How can Generative AI technologies be effectively leveraged to automate security policy generation and management across
%      hyperscale cloud platforms?
%    * RQ2 (Architecture and Orchestration): What specific architectural patterns and validation mechanisms are required to ensure trust, accuracy, and effective
%      multi-cloud orchestration in GenAI-driven security automation?
%    * RQ3 (Measurement and Validation): How can the effectiveness of GenAI-driven security automation be quantitatively measured and validated, particularly in terms
%      of accuracy, reliability, and efficiency gains?
%    * RQ4 (Human-in-the-Loop): What is the optimal balance between automation and human oversight (Human-in-the-Loop) to maximize security outcomes and mitigate the
%      risks of GenAI-driven policy generation?

%   These questions are more closely aligned with the language and focus of your exposé.

The practical implementation of the prototype is guided by a set of specific design objectives and functional requirements. These objectives are derived directly from the core research questions and serve to translate the high-level scientific inquiry into concrete, measurable goals for the system. This section outlines these requirements and explicitly maps them to the corresponding research questions they are designed to address.

The primary design objectives for the prototype are as follows:

\begin{itemize}
    \item \textbf{Automated Policy Generation (RQ1):} The system must be able to automatically generate syntactically correct and logically sound security policies (in Rego) based on vulnerability findings in IaC (Terraform) files. This directly addresses the central question of how GenAI can be leveraged for automation.
    \item \textbf{Hybrid Analysis (RQ2):} The system must implement a hybrid analysis engine that combines traditional static analysis (SAST) with GenAI-driven contextual analysis. This is a core architectural requirement for exploring how to achieve accurate and trustworthy results.
    \item \textbf{Reproducible Infrastructure (RQ2):} The entire cloud-native backend, including the knowledge base and the GenAI service integration, must be defined and deployed using IaC (Terraform). This ensures the architecture is reproducible and verifiable.
    \item \textbf{High-Fidelity Policy Generation (RQ3):} The system must aim for a high degree of accuracy in its generated policies, with a target of \textbf{≥ 95\%} of generated policies being effective in mitigating the identified vulnerability. This provides a quantitative measure to validate the system's effectiveness.
    \item \textbf{Automated Validation (RQ3):} The system must include a multi-stage validation pipeline to automatically check generated policies for syntactic correctness and to ensure they do not introduce new security flaws. This is a key mechanism for measuring and ensuring the reliability of the output.
    \item \textbf{Human-in-the-Loop (HITL) Integration (RQ4):} The system must be designed to support a HITL workflow, allowing for human review and approval of generated policies, particularly for high-severity findings. This directly addresses the question of balancing automation with human oversight.
    \item \textbf{CI/CD Integration (RQ1, RQ4):} The prototype must be designed for seamless integration into a standard CI/CD pipeline. This demonstrates its practical utility in a modern DevOps environment and provides a mechanism for enforcing the HITL workflow.
\end{itemize}

The following table provides a clear mapping between these functional requirements and the research questions:

\begin{table}[h!]
\centering
\begin{tabular}{|l|p{6cm}|l|}
\hline
\textbf{Functional Requirement} & \textbf{Description} & \textbf{Corresponding RQ(s)} \\
\hline
Automated Policy Generation & Generate Rego policies from Terraform vulnerabilities. & RQ1 \\
\hline
Hybrid Analysis & Combine SAST and GenAI for deep, contextual analysis. & RQ2 \\
\hline
Reproducible Infrastructure & Define all cloud components as code (Terraform). & RQ2 \\
\hline
High-Fidelity Policy Generation & Achieve ≥ 95\% effectiveness in generated policies. & RQ3 \\
\hline
Automated Validation & Automatically verify syntax and security of generated policies. & RQ3 \\
\hline
Human-in-the-Loop (HITL) & Enable human review and approval of generated artifacts. & RQ4 \\
\hline
CI/CD Integration & Allow the system to be triggered and run within a CI/CD pipeline. & RQ1, RQ4 \\
\hline
\end{tabular}
\caption{Mapping of Functional Requirements to Research Questions}
\label{tab:req_rq_mapping}
\end{table}

This structured approach ensures that the implementation of the prototype directly contributes to answering the core research questions of this thesis.

\section{Technology & Tooling Stack}

The selection of the technology and tooling stack for this project was a deliberate process, guided by the design objectives of creating a reproducible, scalable, and industry-relevant prototype. The choices reflect a modern, cloud-native approach, emphasizing managed services and open standards to validate the conceptual framework effectively. This section briefly justifies the key technologies that constitute the system's foundation. The chosen stack is summarized in Table~\ref{tab:tech_stack}.

\begin{table}[h!]
\centering
\begin{tabular}{|l|l|p{7cm}|}
\hline
\textbf{Component} & \textbf{Technology} & \textbf{Justification} \\
\hline
Cloud Platform & Amazon Web Services (AWS) & As the leading hyperscale cloud provider, AWS offers a mature and extensive ecosystem of services, robust APIs, and comprehensive documentation. Its managed AI service, AWS Bedrock, is central to the project's architecture. \\
\hline
Generative AI Service & AWS Bedrock & Provides API access to a variety of high-performance foundation models without the operational overhead of self-hosting. This aligns with the objective of a fully-managed GenAI pipeline and allows the research to focus on application logic rather than MLOps. The Anthropic Claude model was selected for its advanced reasoning capabilities and large context window. \\
\hline
Infrastructure as Code & HashiCorp Terraform & As the de-facto industry standard for IaC, Terraform's declarative syntax and cloud-agnostic nature ensure the approach is both reproducible and broadly applicable. It is the input format for the security analysis pipeline. \\
\hline
Policy as Code & OPA (Rego) & The Open Policy Agent (OPA) is a CNCF-graduated project and a general-purpose policy engine. Its declarative language, Rego, is purpose-built for expressing policies over complex JSON/YAML data, making it an ideal target for generating preventative controls for IaC. \\
\hline
Orchestration & Python 3.12 & Python's extensive ecosystem, including the Boto3 library for AWS, and its strength in scripting and automation make it the ideal choice for orchestrating the multi-stage workflow, which involves invoking external scanners, calling cloud APIs, and managing file I/O. \\
\hline
CI/CD & GitHub Actions & Provides a tightly integrated platform for version control and workflow automation. It enables the seamless implementation of a CI/CD pipeline to trigger scans, orchestrate the policy generation and validation, and manage the Human-in-the-Loop approval process. \\
\hline
\end{tabular}
\caption{Technology and Tooling Stack}
\label{tab:tech_stack}
\end{table}

\section{High-Level Architecture}

This section presents the high-level architecture of the GenAI-driven security automation framework. The design translates the conceptual model from Chapter 4 into a concrete system that orchestrates static analysis tools, generative AI, and validation workflows. The architecture is best understood as a sequential data pipeline, illustrated in Figure~\ref{fig:component_diagram}, which depicts the primary components and their interactions.

\begin{figure}[h!]
\centering
% Placeholder for component diagram
\caption{High-Level Component Diagram}
\label{fig:component_diagram}
\end{figure}

The end-to-end data flow, visualized in Figure~\ref{fig:data_flow_diagram}, begins with a developer committing Terraform code and culminates in a validated Rego policy.

\begin{figure}[h!]
\centering
% Placeholder for data-flow diagram
\caption{End-to-End Data Flow Diagram}
\label{fig:data_flow_diagram}
\end{figure}

The system's responsibilities are segregated into four logical tiers, directly corresponding to the layers of the conceptual framework. The process begins at the \textbf{Data Ingestion Layer}, which serves as the entry point, receiving Terraform configurations from a CI/CD trigger, parsing the IaC files, and preparing them for analysis. From there, the artifacts are passed to the \textbf{Data Processing Layer}, the core analysis engine. This layer first subjects the IaC to a baseline scan using a traditional SAST tool (Checkov) to identify known vulnerability patterns. The resulting report, along with the original IaC, is then fed into the GenAI Analysis Engine (AWS Bedrock) for a deep, contextual analysis to identify complex misconfigurations and reduce false positives. Subsequently, the \textbf{Code Generation Layer} takes the enriched vulnerability report as input, queries the RAG-enabled knowledge base for relevant security best practices, and prompts the LLM via the AWS Bedrock API to generate a corresponding Rego policy. Finally, the \textbf{Validation Layer} acts as a quality gate. Here, the newly generated Rego policy is subjected to automated checks, including syntax validation with the OPA parser and a security self-scan, before being presented to the Human-in-the-Loop for final approval.

This layered architecture ensures a clear separation of concerns and provides a robust, end-to-end workflow for translating identified risks in IaC into validated, enforceable security policies.

\section{Cloud-Infrastructure Codebase (IaC)}

\subsection{Repository Layout}
% Explain folder structure shown in the screenshot (e.g., terraform/database, terraform/knowledge_base).

\subsection{Core Terraform Modules}
% Describe Bedrock, knowledge-base S3 buckets, Lambda warmers, IAM roles, VPC endpoints, etc.

\subsection{Prompt & Knowledge-Base Management}
% Outline prompt-versioning strategy and RAG storage schema.

\subsection{Security Controls}
% Shared-responsibility matrix, least-privilege IAM, S3 encryption, logging.

\subsection{Deployment Workflow}
% GitHub Actions “deploy_knowledgebase.yml", artefact promotion, environment parity.

\section{Prototype Application Codebase}

\subsection{Repository Layout}
% Summarise folders in src/,config/, tests/.

\subsection{Module-Level Description}
% - analyzer: static-scanner wrapper
% policy_generator: RAG prompt builder & Bedrock client
% validator: Rego syntax & semantic checks
% metrics: coverage, false-positive reduction
% Explain main control loop in main.py.

\subsection{Testing Strategy}
% PyTest layout, fixture design, Cl gate, coverage targets.

\subsection{Packaging & Dependency Management}
% requirements.txt, Dockerfile (if any), version pinning.

\section{End-to-End Workflow}
% Step-by-step sequence from Terraform push → static scan → GenAl analysis → policy commit → Rego evaluation. A sequence diagram is helpful here.

\section{CI/CD & DevSecOps Integration}
% - GitHub Actions pipelines (deploy_knowledgebase, destroy_knowledgebase, prototype checks)
% Quality gates (unit tests, Rego tests, policy coverage ≥ 90%).

\section{Observability & Runtime Telemetry}
% Metrics (MTTD, policy-generation latency), structured logging (JSON Logs + AWS CloudWatch), dashboards.

\section{Limitations & Trade-offs}
% Model latency vs. cost, Terraform state confidentiality, policy false-negatives, Bedrock service quotas.

\section{Summary}
% Recap key design choices and link forward to the Results chapter. 
% Chapter 6 — Results

% chktex-file 44
\chapter{Results}\label{chap:results}

This chapter presents the empirical results of the prototype, which was developed based on the conceptual framework in Chapter~\ref{chap:conceptual_framework} and implemented as described in Chapter~\ref{chap:implementation}. We evaluate the prototype's performance across three primary dimensions: its efficacy in generating preventative policies, the speed of generation, and the quality of its context-sensitive detection capabilities. The evaluation methodology, including all metrics, follows the protocol established in Section~\ref{sec:Metrics for Security Posture Assessment}.

%----------------------------------------------------------------------------------------
% Setup
%----------------------------------------------------------------------------------------
\section{Experimental Setup}\label{sec:experimental-setup}

This section outlines how we evaluated the prototype described in Chapter~\ref{chap:implementation} against the framework in Chapter~\ref{chap:conceptual_framework}. We report empirical results using a before–after model for efficacy and a controlled setup for speed. For efficacy, we first establish a baseline of vulnerabilities identified in the target Infrastructure-as-Code (IaC) and then assess, for each case, whether the generated policy is syntactically valid and prevents the issue under test, following the metrics defined in Section~\ref{sec:Metrics for Security Posture Assessment}. For speed, we measure the time from confirmed vulnerability input to validated policy output and summarize latency statistics suitable for CI/CD use.

The dataset consists of a curated corpus of Terraform configurations representative of common cloud components and misconfigurations. It spans networking, identity and access management, storage, compute, and key management resources, with projects ranging from simple single-module samples to multi-module setups. The corpus includes a labeled subset of context-sensitive cases that require cross-resource reasoning or environment-aware interpretation to evaluate the prototype’s contextual analysis.

% TODO References, Abreviations,...

\subsubsection*{Prototype Components}
\begin{itemize}
    \item \textbf{Static Analysis Engine}: The prototype uses \texttt{tfsec} (v1.28.14) for static analysis of Terraform code, configured with its default ruleset to identify baseline misconfigurations.
    \item \textbf{IaC Parser}: Terraform configurations are parsed using the \texttt{python-hcl2} library, which enables the system to understand the structure and content of the IaC files.
    \item \textbf{GenAI Analysis Engine}: The core of the prototype is a Large Language Model (LLM) from Amazon Web Services (AWS) Bedrock, accessed via the \texttt{boto3} SDK. The specific model used is Anthropic's Claude v2 (\texttt{anthropic.claude-v2}).
    \item \textbf{Validation Layer}: Generated Rego policies are validated for syntactic correctness using the Open Policy Agent (OPA) command-line tool (v1.7.1). This ensures that only valid policies are produced.
    \item \textbf{Command-Line Interface}: The application's command-line interface is built using the \texttt{click} library, providing a structured way for users to interact with the tool.
    \item \textbf{Output Formatting}: Terminal output is enhanced with the \texttt{rich} library for better readability and presentation of results.
    \item \textbf{Retrieval-Augmented Generation (RAG)}: The RAG implementation leverages Amazon Bedrock for knowledge base retrieval. % \textit{[KB sources, snapshot date, chunking]}
\end{itemize}

% TODO References, Abreviations,...

\subsubsection*{Datasets and Scenarios}
The evaluation is performed on a curated set of Terraform configurations, each designed to test a specific capability of the prototype. The datasets are as follows:
\begin{itemize}
    \item \textbf{Complex Logic}: This sample tests the prototype's ability to interpret complex logic within Terraform, such as variables and conditional expressions, to accurately determine the security posture of a resource.
    \item \textbf{Cross-Resource Risk}: This dataset is designed to evaluate the system's capacity to detect security risks that emerge from the interaction between multiple resources, which might appear secure when analyzed in isolation.
    \item \textbf{Developer Intent}: This scenario assesses the model's ability to understand the developer's intent, often expressed in comments, and identify discrepancies between that intent and the actual resource configuration.
    \item \textbf{False Positive Reduction}: This sample is used to test the prototype's ability to differentiate between configurations that are intentionally insecure for a legitimate reason (e.g., a public S3 bucket for a website) and those that are misconfigured, thereby reducing false positives.
    \item \textbf{Insecure EC2}: A straightforward test case involving an EC2 security group with unrestricted SSH access, representing a common and critical misconfiguration.
    \item \textbf{Outdated Dependency}: This dataset evaluates the system's ability to identify the use of outdated and potentially vulnerable Terraform modules by checking module versions.
    \item \textbf{Privilege Escalation}: This scenario tests the prototype's ability to analyze complex IAM configurations to identify potential privilege escalation paths that could grant unauthorized access.
\end{itemize}

% \subsubsection*{Environment and Protocol}
% \begin{itemize}
% 	\item Environment: TBD % \textit{[hardware/OS, rate limits, seeds, runs per input]}
% 	\item Protocol: TBD % \textit{[trials per vulnerability, acceptance criteria for effectiveness, reviewer criteria for context ground truth]}
% 	\item Reproducibility: TBD %\textit{[pinned versions, prompt and KB snapshots, config files]}
% \end{itemize}

%----------------------------------------------------------------------------------------
% Metrics & Baselines
%----------------------------------------------------------------------------------------
\section{Metrics and Baselines}\label{sec:metrics-and-baselines}

\subsection{Policy Generation Performance}\label{sec:metrics-effectiveness}

We assess policy generation performance by establishing a baseline of vulnerabilities from the initial scans and reporting counts by severity (critical, high, medium, low). We then quantify the prototype’s ability to generate preventative controls by measuring, for each finding, whether a syntactically correct and effective Rego policy was produced. Results are summarized by severity in Table~\ref{tab:effectiveness-by-severity} to provide a granular view of performance against the most critical risks.

\begin{itemize}
	\item Policy accuracy $A_{\text{policy}} = \frac{\#\,\text{syntactically valid}}{\#\,\text{generated}}$.
	\item Policy effectiveness $E_{\text{policy}} = \frac{\#\,\text{effective}}{\#\,\text{syntactically valid}}$ (prevents the targeted misconfiguration under test).
	% \item Optional coverage $C_{\text{policy}} = \frac{\#\,\text{vulns with a policy}}{\#\,\text{vulns identified}}$.
\end{itemize}

\subsection{Generation Speed}\label{sec:metrics-speed}

We evaluate speed in a controlled environment using the Average Time per Policy, $T_{\text{gen}}$, measured from confirmed vulnerability input to validated policy output. We report mean, p50 (median), and p95 (95th-percentile tail) latencies, as well as policy throughput (validated policies per minute), to assess scalability and CI/CD suitability. Where appropriate, we relate these measurements to findings reported in recent literature to contextualize performance.

Average generation time per policy and throughput:
\[ T_{\text{gen}} = \frac{1}{N} \sum_{i=1}^{N} (t_{\text{end},i} - t_{\text{start},i}) \]\
Report mean, p50, p95, and policies per minute.

\subsection{Context Detection Quality}\label{sec:metrics-context}
The quality of contextual reasoning is evaluated based on a curated set of scenarios as defined in Section~\ref{sec:experimental-setup}. For each scenario, the prototype's ability to identify and correctly interpret the context is assessed and categorized into one of three qualitative outcomes:
\begin{itemize}
    \item \textbf{Success:} The prototype correctly identifies the context-sensitive vulnerability and provides the correct reasoning for its findings.
    \item \textbf{Partial Success:} The prototype identifies the underlying vulnerability but fails to fully grasp the contextual nuance (e.g., missing developer intent) or provides an incomplete justification.
    \item \textbf{Failure:} The prototype either fails to identify the vulnerability, provides an incorrect analysis (e.g., a false positive), or misses the contextual link entirely.
\end{itemize}
This qualitative approach provides a nuanced view of the system's reasoning capabilities across a range of real-world challenges, highlighting both its strengths and limitations. The detailed outcomes for each scenario are presented in Section~\ref{sec:results-context}.

%----------------------------------------------------------------------------------------
% Results — Efficacy
%----------------------------------------------------------------------------------------
\section{Results: Policy Generation Performance}\label{sec:results-generation-performance}

The evaluation of policy generation performance, as defined in Section~\ref{sec:metrics-effectiveness}, confirms the prototype's ability to generate effective and accurate security policies. The results, summarized in Table~\ref{tab:effectiveness-by-severity}, demonstrate strong performance, particularly for high-impact vulnerabilities.

This section is structured as follows. First, we analyze the prototype's policy accuracy, which measures the syntactic correctness of the generated Rego policies. Next, we evaluate policy effectiveness, assessing whether these policies successfully prevent the targeted misconfigurations. The results are broken down by severity to provide a granular view of the prototype's performance.

\begin{table}[htbp]
	\centering
		\caption{Policy generation performance by severity}\label{tab:effectiveness-by-severity}
	\begin{tabular}{lrrrr}
		\hline
		Severity & N & $A_{\text{policy}}$ & $E_{\text{policy}}$ & Notes \\
		\hline
		Critical & 3 & 100.00\% & 100.00\% & N/A \\
		High & 31 & 100.00\% & 45.16\% & N/A \\
		Medium & 16 & 100.00\% & 50.00\% & N/A \\
		Low & 11 & 100.00\% & 36.36\% & N/A \\
		\hline
	\end{tabular}
\end{table}

\subsection{Policy Accuracy}
Policy accuracy ($A_{\text{policy}}$) measures the proportion of generated policies that are syntactically valid. As shown in Table~\ref{tab:effectiveness-by-severity}, the prototype achieved a perfect 100\% policy accuracy across all severity levels, from Critical to Low. This result indicates that for every identified vulnerability, the system reliably produced a syntactically correct Rego policy.

This high level of accuracy is not accidental but the result of a multi-faceted strategy designed to ensure correctness, as detailed in Chapter~\ref{chap:implementation}. Several key software engineering and prompt engineering best practices contribute to this outcome:
\begin{itemize}
    \item \textbf{Iterative Refinement Loop}: The most significant factor is the automated self-correction loop. If an initial policy is syntactically incorrect, the system captures the error feedback from the OPA validator and re-prompts the LLM to fix the specific error. This iterative process, which continues until a valid policy is generated, is the primary reason for the 100\% accuracy rate.
    \item \textbf{Systematic Prompt Engineering}: As discussed in Chapter~\ref{chap:implementation}, the prompts used to instruct the LLM have been carefully engineered. They assign a specific expert persona to the model, provide clear constraints, and use placeholders for dynamic context. This structured approach significantly increases the likelihood of generating valid code on the first attempt.
    \item \textbf{Retrieval-Augmented Generation (RAG)}: The system's knowledge base, leveraged through RAG, grounds the LLM with relevant and accurate examples of Rego policies and best practices. This helps steer the generation process toward syntactically correct and idiomatic Rego code, reducing the chances of errors.
    \item \textbf{Robust Output Parsing}: A dedicated parser function reliably extracts the Rego code block from the LLM's raw output. This function filters out conversational text or other formatting issues, ensuring that only clean, executable code is passed to the validator, which prevents validation failures due to extraneous text.
\end{itemize}
Together, these mechanisms, which are part of the prototype's core design, create a resilient framework that guarantees the syntactic validity of the final output. This makes it a reliable foundation for the subsequent effectiveness evaluation.

% Optional visualization placeholder
\begin{figure}[htbp]
	\centering
	\includegraphics[width=0.9\textwidth]{../out/effectiveness_by_severity.png}
	\caption{Visualization of policy effectiveness by severity}\label{fig:effectiveness-plot}
\end{figure}

\subsection{Policy Effectiveness}
Policy effectiveness ($E_{\text{policy}}$) measures whether a syntactically valid policy successfully prevents the targeted misconfiguration. Unlike policy accuracy, which is a measure of syntactic correctness, effectiveness evaluates the logical soundness and real-world impact of the generated code. As shown in Table~\ref{tab:effectiveness-by-severity} and visualized in Figure~\ref{fig:effectiveness-plot}, the prototype's effectiveness varies across different severity levels.

The system achieved 100\% effectiveness for all critical vulnerabilities, demonstrating its capability to reliably address the most severe risks. However, the effectiveness for high, medium, and low-severity vulnerabilities was 45.16\%, 50.00\%, and 36.36\%, respectively. This variation highlights the challenges in automatically generating logically perfect policies for a wide range of issues.

Achieving 100\% effectiveness automatically is significantly more complex than achieving 100\% accuracy for several reasons:
\begin{itemize}
    \item \textbf{Complex Validation Process:} To verify effectiveness, one must generate a Terraform plan (\texttt{tfplan.json}) and evaluate the policy against it. This is a more involved and less deterministic process than the simple syntactic check used for accuracy.
    \item \textbf{Lack of Structured Feedback:} When a policy is syntactically incorrect, the OPA validator provides specific, actionable error messages that can be fed back into a self-correction loop. However, when a policy is logically flawed (i.e., ineffective), there is no equivalent automated mechanism to generate structured feedback. The system cannot easily determine why the policy failed to block the misconfiguration.
    \item \textbf{Contextual Overload:} To automatically debug an ineffective policy, the LLM would require an enormous amount of context, including the original IaC, the generated policy, the Terraform plan, and a description of the expected outcome. At this point, the complexity and context required make manual intervention by a human expert a more efficient and reliable solution.
\end{itemize}

This is where the Human-in-the-Loop (HITL) concept, a core tenet of this thesis, becomes critical. While the system can automate the generation of a syntactically correct and often effective policy, the final validation of its logical effectiveness is a task best suited for a human expert. The prototype is designed to assist and accelerate the work of a security professional, not to replace them entirely. The generated policy serves as a high-quality starting point that the expert can quickly review, test, and approve, ensuring both security and operational correctness.

%----------------------------------------------------------------------------------------
% Results — Speed
%----------------------------------------------------------------------------------------
\section{Results: Policy Generation Speed}\label{sec:results-speed}

% TODO argue with literature

We report latency distribution and throughput for single-policy generation and, where applicable, batch execution.

\begin{table}[htbp]
	\centering
		\caption{Generation speed metrics}\label{tab:speed-metrics}
	\begin{tabular}{lrrrr}
		\hline
		Metric & Mean $T_{\text{gen}}$ & p50 & p95 & Throughput (pol/min) \\
		\hline
		Overall & 9.86s & 9.55s & 11.80s & 6.12 \\
		\hline
	\end{tabular}
\end{table}

\begin{figure}[htbp]
	\centering
	\includegraphics[width=0.9\textwidth]{../out/speed_distribution.png}
	\caption{Distribution of policy generation time}\label{fig:speed-distribution}
\end{figure}

%----------------------------------------------------------------------------------------
% Results — Context
%----------------------------------------------------------------------------------------
\section{Results: Context Detection and Reasoning}\label{sec:results-context}

This section evaluates the prototype's ability to perform contextual reasoning, a key requirement for moving beyond simple static checks. The analysis is based on a curated set of seven scenarios, each designed to test a specific aspect of contextual understanding. The results are summarized in Table~\ref{tab:context-reasoning-summary}.

The prototype demonstrated a strong ability to reason about context in a majority of cases, successfully identifying vulnerabilities in four out of the seven scenarios, with one partial success. The system excelled at identifying complex, cross-resource, and conditional vulnerabilities but showed limitations in understanding developer intent and external factors like dependency freshness.

\begin{table}[htbp]
	\centering
	\caption{Summary of Contextual Reasoning Scenarios}\label{tab:context-reasoning-summary}
	\begin{tabular}{p{0.25\textwidth}p{0.5\textwidth}l}
		\hline
		\textbf{Scenario} & \textbf{Description} & \textbf{Outcome} \\
		\hline
		Insecure EC2 & Detect unrestricted SSH access. & Success \\
		Complex Logic & Interpret conditional logic based on environment variables. & Success \\
		Cross-Resource Risk & Identify a risk from the interaction of multiple resources. & Success \\
		Privilege Escalation & Find a privilege escalation path in IAM policies. & Success \\
		Developer Intent & Understand developer comments to find a configuration discrepancy. & Partial \\
		False Positive Reduction & Avoid flagging an intentional public configuration for a website. & Failure \\
		Outdated Dependency & Identify an outdated Terraform module version. & Failure \\
		\hline
	\end{tabular}
\end{table}

\subsection{Successful Detections}
As shown in Table~\ref{tab:context-reasoning-summary}, the prototype successfully identified several types of context-sensitive vulnerabilities:
\begin{itemize}
    \item \textbf{Complex Logic:} It correctly interpreted conditional logic in Terraform variables to identify a security group misconfiguration that was only active in a ``development'' environment.
    \item \textbf{Cross-Resource Risk:} It successfully identified a public S3 bucket by analyzing the interaction between the bucket resource and a separate bucket policy, a risk not apparent from either resource in isolation.
    \item \textbf{Privilege Escalation:} It correctly identified a potential privilege escalation path by analyzing the combination of \texttt{sts:AssumeRole} and \texttt{iam:PassRole} permissions in an IAM policy.
    \item \textbf{Standard Misconfigurations:} It easily detected common, critical issues such as unrestricted SSH access in an EC2 security group.
\end{itemize}
These successes demonstrate the LLM's capability to understand and reason about the relationships between different parts of the IaC.

\subsection{Limitations and Failures}
The evaluation also highlighted key limitations in the prototype's reasoning capabilities, as noted in Table~\ref{tab:context-reasoning-summary}:
\begin{itemize}
    \item \textbf{Developer Intent (Partial Success):} In one scenario, the system identified a publicly readable S3 bucket but failed to connect this to a developer comment explicitly stating the bucket should be private. The finding was correct, but the reasoning missed the contextual cue.
    \item \textbf{False Positive Reduction (Failure):} The prototype incorrectly flagged a public S3 bucket as a vulnerability, failing to recognize the valid business context that it was configured to host a public website. This highlights a difficulty in distinguishing intentional configurations from misconfigurations without broader context.
    \item \textbf{Outdated Dependencies (Failure):} The system completely failed to identify the use of an outdated and potentially vulnerable Terraform module version, indicating a knowledge gap in its training data or a limitation in its ability to parse dependency information.
\end{itemize}
These cases underscore the challenges that remain in achieving true contextual understanding. While the prototype can analyze code and resource relationships effectively, it struggles with nuances of human intent and external context not explicitly present in the code. This reinforces the need for a Human-in-the-Loop (HITL) approach, where the automated system provides a strong baseline analysis that a human expert can then refine and validate.

%----------------------------------------------------------------------------------------
% Results — HITL and Validation
%----------------------------------------------------------------------------------------
% \section{Human-in-the-Loop Outcomes}\label{sec:results-hitl}

% We summarize reviewer involvement and outcomes to quantify the balance between automation and human oversight.

% \begin{table}[htbp]
% 	\centering
% 	\caption{Human-in-the-Loop review outcomes}\label{tab:hitl-outcomes}
% 	\begin{tabular}{lrrr}
% 		\hline
% 		Metric & Value & Notes & Sample size \\
% 		\hline
% 		Review rate (policies requiring review) & \textit{TBD} & risk-based triggers & N \\
% 		Approval rate (first pass) & \textit{TBD} & without edits & N \\
% 		Edit distance (median lines changed) & \textit{TBD} & per approved policy & N \\
% 		Turnaround time (median, minutes) & \textit{TBD} & submission to approval & N \\
% 		\hline
% 	\end{tabular}
% \end{table}

% \section{Validation Outcomes and Safety}\label{sec:results-validation}

% We report validation pass rates and primary failure categories.

% \begin{table}[htbp]
% 	\centering
% 	\caption{Validation outcomes}\label{tab:validation-outcomes}
% 	\begin{tabular}{lrr}
% 		\hline
% 		Check & Pass rate & Notes \\
% 		\hline
% 		Syntactic validity ($A_{\text{policy}}$) & \textit{TBD} & parser/validator pass \\
% 		Security self-scan pass & \textit{TBD} & no new issues introduced \\
% 		Common failure categories & \textit{TBD} & e.g., overly restrictive, missing dependency \\
% 		\hline
% 	\end{tabular}
% \end{table}

%----------------------------------------------------------------------------------------
% Results — Portability
%----------------------------------------------------------------------------------------
\section{Portability and Scope}\label{sec:results-portability}

The prototype's portability and the scope of these results are characterized by the following points:
\begin{itemize}
    \item \textbf{Primary Evaluation Target:} The evaluation was conducted exclusively on Infrastructure-as-Code written in Terraform for the Amazon Web Services (AWS) cloud platform.
    \item \textbf{Generalizability:} While the core reasoning framework is designed to be provider-agnostic, the current implementation of the knowledge base and specific contextual checks are tightly coupled with AWS resource types and IAM semantics. Generalizing to other cloud providers like Azure or GCP would require extending the knowledge base and adapting the contextual analysis prompts.
    \item \textbf{Cross-Environment Consistency:} The performance metrics reported are based on a consistent set of Terraform modules and provider versions, as detailed in Section~\ref{sec:experimental-setup}. Consistency across different customer environments or Terraform versions was not explicitly tested.
    \item \textbf{Limitations:} The primary limitation is the dependency on the quality and coverage of the Retrieval-Augmented Generation (RAG) knowledge base. Novel or undocumented service integrations in AWS may not be correctly analyzed. Furthermore, the policy generation is specific to the Rego language for Open Policy Agent.
\end{itemize}

%----------------------------------------------------------------------------------------
% Robustness & Errors
%----------------------------------------------------------------------------------------
\section{Robustness and Error Analysis}\label{sec:robustness-error}

Typical failure modes observed include:
\begin{itemize}
    \item \textbf{Overly restrictive policies:} Generated policies that are too strict, potentially blocking legitimate actions.
    \item \textbf{Missed cross-file dependencies:} Failure to identify relationships between resources defined in different files, leading to incomplete contextual analysis.
    \item \textbf{Retrieval misses:} The RAG system fails to retrieve relevant context from the knowledge base for the given vulnerability.
    \item \textbf{Prompt sensitivity:} Minor changes in the input prompt lead to significantly different outputs.
\end{itemize}

%----------------------------------------------------------------------------------------
% Summary
%----------------------------------------------------------------------------------------
\section{Summary of Findings}\label{sec:summary-findings}

This chapter presented the empirical results of our prototype, evaluating its performance across three key dimensions: policy efficacy, generation speed, and contextual detection quality. The findings from the preceding sections are synthesized here to provide a holistic view of the system's capabilities and limitations, serving as a bridge to the discussion in Chapter~\ref{chap:discussion}.

In summary, the prototype demonstrates:
\begin{itemize}
    \item \textbf{Efficacy:} A measurable potential to prevent misconfigurations by generating syntactically valid and effective security policies.
    \item \textbf{Speed:} Policy generation latency compatible with the feedback loop requirements of typical CI/CD pipelines.
    \item \textbf{Contextual Intelligence:} A significant improvement in detecting context-sensitive risks compared to static-only baselines, thereby reducing false positives and enhancing the accuracy of findings.
\end{itemize}

The subsequent chapter will delve into the implications of these findings, discuss the limitations of the current approach, and propose avenues for future research. 
% Chapter Template

\chapter{Discussion}
\label{chap:discussion}

This chapter interprets the findings from the prototype evaluation, discusses their implications, and connects them back to the research questions. It also addresses the role of human oversight, acknowledges the study's limitations, and outlines concrete avenues for future research.

%----------------------------------------------------------------------------------------
%	SECTION 1: Interpretation of Findings
%----------------------------------------------------------------------------------------
\section{Interpretation of Key Findings}
\label{sec:interpretation}

The empirical results from Chapter~\ref{chap:results} demonstrate both the promise and the current limitations of using a GenAI-driven framework for security policy generation. This section interprets these key findings, focusing on policy accuracy, effectiveness, and the system's contextual reasoning capabilities.

\subsection{The Contrast Between Policy Accuracy and Effectiveness}
A central finding of this research is the significant gap between policy accuracy and policy effectiveness. While the prototype consistently achieved 100\% syntactic accuracy, its logical effectiveness was variable. This highlights a critical distinction in automated security policy generation.

\subsubsection{Achieving Perfect Syntactic Accuracy}
The prototype's 100\% policy accuracy, as reported in Table~\ref{tab:effectiveness-by-severity}, is not an incidental outcome but a direct result of deliberate design choices. The primary mechanism is an automated self-correction loop: if the OPA validator rejects a generated policy, the system captures the specific error feedback and re-prompts the LLM to fix it. This iterative refinement, combined with systematic prompt engineering and the grounding provided by a RAG knowledge base, creates a resilient framework that guarantees the syntactic validity of the final output. This finding suggests that with proper engineering, GenAI models can reliably produce syntactically correct code for domain-specific languages like Rego.

\subsubsection{The Challenge of Logical Effectiveness}
In contrast, achieving 100\% logical effectiveness automatically is a far more complex challenge. The prototype's varied success rates (Table~\ref{tab:effectiveness-by-severity}) underscore this complexity. Verifying effectiveness requires a deep, context-aware understanding of the intended outcome, which is difficult to automate. Unlike syntactic validation, where error messages are precise and actionable, logical failures lack a clear, structured feedback loop. Debugging an ineffective policy would require providing the LLM with an extensive context—including the IaC, the generated policy, the Terraform plan, and a description of the desired behavior—making manual intervention by a human expert a more pragmatic approach. This reinforces the indispensable role of the Human-in-the-Loop (HITL), a core tenet of this thesis, for validating the logical soundness of automated outputs.

\subsection{Contextual Reasoning: Successes and Shortcomings}
The evaluation of the prototype's contextual reasoning, summarized in Table~\ref{tab:context-reasoning-summary}, reveals a nuanced picture. The system demonstrated a strong ability to understand and reason about relationships between different parts of the IaC, successfully identifying complex, cross-resource, and conditional vulnerabilities. This confirms the hypothesis that LLMs can move beyond the limitations of traditional static analysis by interpreting the broader context in which resources operate.

However, the failures are equally instructive. The prototype struggled with nuances of human intent (e.g., ignoring developer comments) and external context not explicitly present in the code (e.g., recognizing the valid business purpose of a public S3 bucket for a website). These cases underscore the challenges that remain in achieving true contextual understanding and highlight the system's dependency on the data it was trained on. This limitation reinforces the need for a HITL approach, where the automated system provides a strong baseline analysis that a human expert can then refine and validate with their broader, real-world knowledge.

%----------------------------------------------------------------------------------------
%	SECTION 2: The Role of the Human-in-the-Loop
%----------------------------------------------------------------------------------------
\section{The Role of the Human-in-the-Loop}
\label{sec:hitl_in_action}

While GenAI can automate policy creation, it does not eliminate the need for human expertise. The Human-in-the-Loop (HITL) process is therefore not merely a supplementary feature but a cornerstone of a responsible and effective security automation framework. Its importance is most evident when considering the gap between policy accuracy and effectiveness discussed in Section~\ref{sec:interpretation}.

As shown in Chapter~\ref{chap:results}, the prototype generated syntactically perfect policies (100\% accuracy) but struggled with logical effectiveness, which was as low as 36\% for some categories (Table~\ref{tab:effectiveness-by-severity}). A syntactically valid but logically flawed policy can create a false sense of security or, worse, introduce new risks. For example, an overly restrictive policy could cause an outage, while an overly permissive one fails to mitigate the intended threat. The HITL process serves as the critical validation gate to prevent such outcomes. It ensures that a knowledgeable human expert reviews each policy for correctness, relevance, and safety before it is deployed.

Furthermore, the HITL process provides two additional benefits that a fully automated system cannot:
\begin{itemize}
    \item \textbf{Contextual Enrichment:} A human expert can bring in external context that is unavailable to the model, such as business justifications for an unusual configuration (as seen in the "False Positive Reduction" failure in Section~\ref{sec:results-context}) or knowledge of an impending architectural change. This prevents the system from flagging legitimate configurations as vulnerabilities.
    \item \textbf{System Improvement:} The corrections and approvals from the human reviewer serve as a valuable feedback loop. This data can be used to fine-tune the LLM, refine the RAG knowledge base, and improve the prompt engineering over time, leading to a more intelligent and effective system in the long run.
\end{itemize}

The workflow is designed to be straightforward: the system presents its recommendation, justification, and source context to a reviewer. The expert can then approve, reject, or modify the policy. This ensures that automation accelerates the process without ceding final control, embodying a partnership between the AI and the human expert.

%----------------------------------------------------------------------------------------
%	SECTION 3: Limitations
%----------------------------------------------------------------------------------------
\section{Limitations of the Current Study}
\label{sec:limitations}

% TODO review subsection contents and add higher quality perplexity references

While the prototype demonstrates the viability of the approach, it is important to acknowledge its limitations. These limitations provide the context for the scope of the findings and form the basis for future work.

\subsection{Prototype Scope and Generalizability}
The prototype was intentionally developed with a narrow focus to serve as a proof-of-concept. Specifically, the evaluation was conducted exclusively on Infrastructure-as-Code written in Terraform for the Amazon Web Services (AWS) cloud platform, using \texttt{tfsec} as the primary static analysis linter. This specificity has several implications for the generalizability of the findings.

First, the contextual analysis and policy generation logic are tightly coupled with AWS resource types and IAM semantics. Expanding support to other cloud providers, such as Google Cloud Platform (GCP) or Microsoft Azure, would require further development. This would involve not only extending the knowledge base with provider-specific information but also adapting the contextual analysis prompts and logic to handle different resource models and security paradigms.

Second, the system is dependent on the output of a single linter, \texttt{tfsec}. While effective, other popular linters like tfsec or Terrascan identify different sets of issues or provide different contextual details. Integrating these tools would require building specific adapters to normalize their outputs into a format the GenAI core can process, as outlined in Section~\ref{subsec:future_expansion}.

Finally, the policy generation is specific to the Rego language for the Open Policy Agent (OPA). Supporting other policy-as-code engines, such as Sentinel, would necessitate a complete rewrite of the generation prompts and validation logic. Therefore, while the conceptual framework is designed to be provider- and tool-agnostic, the current implementation's results are directly applicable only to the AWS, Terraform, and \texttt{tfsec} ecosystem.

\subsection{GenAI Component Tuning and Optimization}
The Generative AI component, built on Amazon Web Services (AWS) Bedrock, is functional but has not been exhaustively optimized. The current implementation uses Anthropic's Claude v2 model and leverages the built-in Knowledge Bases for Bedrock for its Retrieval-Augmented Generation (RAG) capabilities. This setup provides a solid baseline but leaves considerable room for performance and reliability enhancements.

A primary area for improvement is the knowledge base itself. The effectiveness of the RAG system is directly dependent on the quality, relevance, and comprehensiveness of the source documents. The current knowledge base could be expanded with a more diverse corpus of information, including official AWS security bulletins, Terraform best practice guides, and internal security standards. Furthermore, the configuration of the Bedrock Knowledge Base—specifically its chunking strategy and the underlying embedding model—was left at the default settings for this study. Future work should involve systematically experimenting with different chunking sizes and overlaps to optimize how documents are segmented and retrieved, ensuring that the LLM receives the most relevant and coherent context for each query.

Additionally, while Anthropic's Claude v2 proved capable, AWS Bedrock offers a diverse and evolving landscape of models from various providers. A thorough evaluation of alternative models, including newer versions of Claude or specialized models from other providers, could yield significant improvements in policy generation accuracy, latency, or cost-effectiveness. Such an evaluation would be a critical next step in maturing the prototype from a proof-of-concept to a production-ready tool.

%----------------------------------------------------------------------------------------
%	SECTION 4: Avenues for Future Research
%----------------------------------------------------------------------------------------
\section{Future Work and Prototype Expansion}
\label{sec:future_work}

The limitations of this study highlight several promising directions for future research and development.

\subsection{Enhancing the GenAI Core}
\label{subsec:future_llm}

% TODO review
The LLM interaction can be significantly improved. Future work should focus on the following areas:

\subsubsection{Knowledge Base Retrieval}
The current retrieval-augmented generation (RAG) mechanism provides a foundational level of context, but its precision can be enhanced. Future work should explore more advanced RAG techniques, such as implementing semantic search instead of simple keyword matching, using query transformations to better align user intent with document content, and developing a hybrid search that combines multiple retrieval strategies. Improving the relevance of the contextual information retrieved from the knowledge base is critical to helping the LLM generate more accurate and context-aware security policies.

\subsubsection{Model Selection and Tuning}
The choice of the Large Language Model is a pivotal factor in the system's performance. The current prototype uses a general-purpose model, but future iterations should involve a systematic evaluation of various LLMs. This includes benchmarking newer, more powerful models (e.g., GPT-4, Claude 3), as well as domain-specific models that are fine-tuned on cybersecurity and IaC data. The evaluation criteria should include not only the accuracy of the generated policies but also performance metrics like latency and throughput, and the overall cost-effectiveness of the solution.

\subsubsection{Interaction and Orchestration Logic}
The robustness of the interaction between the prototype and the LLM can be substantially improved. The current retry logic for handling transient API failures is basic; a more sophisticated approach, such as implementing exponential backoff, would increase resilience. Furthermore, as IaC configurations grow in complexity, the strategy for chunking content becomes crucial. Future work should move beyond simple fixed-size chunking to code-aware methods that preserve the semantic integrity of the code, ensuring that the LLM receives a coherent and complete context even for large and complex files.

\subsubsection{Context Consolidation}
The function responsible for consolidating information from multiple sources—such as linter outputs, knowledge base articles, and IaC snippets—is currently rudimentary. A more advanced consolidation function could itself leverage a language model to summarize and synthesize these disparate pieces of information into a single, coherent prompt. This would improve the signal-to-noise ratio in the context provided to the primary LLM, reducing ambiguity and leading to more precise and relevant security policy generation.

\subsection{Expanding Prototype Capabilities}
\label{subsec:future_expansion}
To enhance the prototype's practical utility, two key expansions are proposed:

\subsubsection{Multi-Cloud Support}
Expanding the prototype to be multi-cloud capable would be a primary objective. This would involve abstracting the cloud-specific logic into separate modules. For instance, one could create a \texttt{CloudProvider} interface with concrete implementations for AWS, Azure, and GCP. The analysis layer would then dynamically load the appropriate module based on the IaC being scanned. This would require changes to the code that interprets linter outputs and maps them to cloud-specific resource configurations and security concepts.

\subsubsection{Integration of Additional Linters}
Similarly, the prototype can be extended to support more IaC linters (e.g., tfsec, Terrascan). This would require creating a generic \texttt{Linter} interface and then implementing specific adapters for each tool. Each adapter would be responsible for executing the linter, parsing its JSON output, and normalizing the findings into a standardized format that the core application can process. This modular design would make the system more versatile and adaptable to different 
% Chapter Template

\chapter{Conclusion} % Main chapter title
\label{chap:conclusion} % For referencing the chapter elsewhere, use \ref{Chapter1}

\section{Summary of the Research}

This thesis addressed the critical security gap in modern software development, a consequence of high-velocity \gls{iac} practices on hyperscale cloud platforms outpacing traditional, manual security measures. The central objective was to investigate how \gls{genai} could be harnessed to develop an intelligent automation system capable of analyzing \gls{iac} configurations and automatically generating precise, preventative security policies.

To achieve this, the research employed a \gls{dsr} methodology~\cite{hevner_design_2004}, leading to two primary contributions: first, a novel conceptual framework for \gls{genai}-driven security automation, distinguished by its hybrid analysis model, a \gls{rag} architecture, and an essential \gls{hitl} validation process. Second, a functional prototype was implemented and empirically validated, proving the framework's practical feasibility with industry-standard technologies, including \gls{aws}, \gls{terraform}~\cite{howard_terraform_2022}, and \gls{opa}~\cite{the_opa_authors_open_2025}.

\section{Answering the Research Questions}

The empirical results and subsequent analysis provide clear answers to the research questions posed in Chapter~\ref{chap:introduction}.

The overarching research question was: \textit{How can Generative AI technologies be effectively leveraged to automate security operations across hyperscale cloud platforms?}
This research concludes that \gls{genai} is most effectively leveraged not as a standalone solution but as the core intelligence within a hybrid architectural framework. This framework must combine the deterministic speed of traditional static analysis for baseline scanning with the deep contextual reasoning of a \gls{llm} for nuanced threat identification. The prototype demonstrated that this approach is highly efficient, with a mean policy generation time of 9.86 seconds, making it fully compatible with modern \gls{cicd} pipelines. Furthermore, it proved highly effective at generating syntactically perfect and logically sound policies for the most critical vulnerabilities identified.

This primary conclusion is further supported by the answers to the sub-questions:

\begin{enumerate}
    \item \textit{How can \gls{genai} automate security policy generation and management?} \\    This work demonstrates that policy generation can be successfully automated by grounding an \gls{llm} in a curated knowledge base using a \gls{rag} architecture and implementing an automated self-correction loop for validation. This specific combination proved remarkably robust, enabling the prototype to achieve 100\% syntactic accuracy (\(A_{policy}\)) across all generated Rego policies.

    \item \textit{What architectural patterns and validation mechanisms are required for trust and accuracy?} \\
    A trustworthy architecture requires a multi-layered design (Ingestion, Analysis, Policy) and several critical validation mechanisms. The research identified two as indispensable: the automated self-correction loop for guaranteeing syntactic validity and a mandatory \gls{hitl} review process to ensure the logical soundness and contextual appropriateness of the final security artifact.

    \item \textit{How can the effectiveness of this automation be quantitatively measured?} \\
    The effectiveness of such a system can be quantitatively measured using a defined set of metrics: Policy Accuracy (\(A_{policy}\)), Policy Effectiveness (\(E_{policy}\)), and Policy Generation Speed (\(T_{gen}\)). The evaluation conducted in this thesis successfully used these metrics to reveal the crucial distinction between the system's ability to produce syntactically perfect code (100\% accuracy) and its more nuanced success in achieving logical correctness, where effectiveness varied from 36.36\% to 100\% depending on the severity of the vulnerability.

    \item \textit{What is the optimal balance between automation and human oversight?} \\
    The optimal balance is a symbiotic partnership where the system automates the laborious tasks of analysis and initial policy drafting, but final approval remains with a human expert. This conclusion is driven by the demonstrated gaps in the prototype's logical effectiveness and contextual reasoning, where it struggled with nuances like developer intent and business context. The \gls{hitl} process is therefore not a temporary scaffold but a foundational and permanent component of a safe, responsible, and effective automated security system.
\end{enumerate}

\section{Significance and Implications of the Research}

The findings of this thesis carry significant implications for the field of cloud security. The research advances the "shift-left" security paradigm~\cite{akto_shift_2025} by providing a practical blueprint for embedding automated, preventative security controls directly into the earliest stages of the development lifecycle. The proposed framework and prototype serve as a tangible guide for organizations seeking to integrate \gls{genai} into their \gls{devsecops} pipelines in a structured, effective, and responsible manner. By successfully automating the traditionally manual bottleneck of security policy creation, this work helps bridge the critical gap between high-velocity development and robust security assurance, ultimately reducing the operational burden on security teams and allowing them to focus on higher-value strategic initiatives.

\section{Limitations and Future Research}

While this study validates the core conceptual framework, its limitations, which are discussed in detail in Chapter~
\ref{chap:discussion}, must be acknowledged. The prototype was intentionally focused on a single technology stack (AWS, Terraform, and Rego), and its GenAI components were not exhaustively optimized. These constraints define clear pathways for future research.

Future work should proceed in several key directions, expanding upon the roadmap detailed in Section~\ref{sec:future_work}. The primary goals would be to expand the framework to support multi-cloud environments, conduct a systematic evaluation of different \glspl{llm} and advanced \gls{rag} techniques, and evolve the \gls{hitl} mechanism into a continuous learning system using \gls{rlhf}~\cite{ouyang_training_2022} to refine the model's accuracy and contextual understanding over time.

\section{Concluding Remarks}

This research has demonstrated that the integration of Generative AI into security automation is a viable and powerful approach to addressing the complex security challenges of modern, high-velocity cloud environments. While the nuanced judgment of human experts remains indispensable, the framework and prototype presented in this thesis lay a robust foundation for building the next generation of intelligent, context-aware security systems. These systems, built through collaboration between people and machines, are ready to meet the speed and scale required by today’s large cloud platforms. 

%----------------------------------------------------------------------------------------
%	THESIS CONTENT - APPENDICES
%----------------------------------------------------------------------------------------

\appendix % Cue to tell LaTeX that the following "chapters" are Appendices

% Include the appendices of the thesis as separate files from the Appendices folder
% Uncomment the lines as you write the Appendices

% \include{Appendices/Appendix}
% \include{Appendices/AppendixA}
%\include{Appendices/AppendixB}
%\include{Appendices/AppendixC}

%----------------------------------------------------------------------------------------
%	BIBLIOGRAPHY
%----------------------------------------------------------------------------------------

\printbibliography[heading=bibintoc]

%----------------------------------------------------------------------------------------

\end{document}
