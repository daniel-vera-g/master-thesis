%----------------------------------------------------------------------------------------
%	GLOSSARY TERMS AND ACRONYMS
%----------------------------------------------------------------------------------------

% This file contains definitions of terms and acronyms used in the thesis.
% To use a glossary term in your text, use \gls{key} for first use (full form)
% or \glspl{key} for plural. Subsequent uses will show the short form.
% For acronyms, you can also use \acrshort{key}, \acrlong{key}, or \acrfull{key}

%----------------------------------------------------------------------------------------
%	ACRONYMS
%----------------------------------------------------------------------------------------

\newacronym{genai}{GenAI}{Generative Artificial Intelligence}
\newacronym{ai}{AI}{Artificial Intelligence}
\newacronym{aws}{AWS}{Amazon Web Services}
\newacronym{soc}{SOC}{Security Operations Center}
\newacronym{airmf}{AI RMF}{Artificial Intelligence Risk Management Framework}
\newacronym{nist}{NIST}{U.S. National Institute of Standards and Technology}
\newacronym{saif}{SAIF}{Secure AI Framework}
\newacronym{llm}{LLM}{Large Language Model}
\newacronym{coe}{CoE}{Center of Excellence}
\newacronym{owasp}{OWASP}{Open Web Application Security Project}
\newacronym{api}{API}{Application Programming Interface}
\newacronym{rag}{RAG}{Retrieval-Augmented Generation}
\newacronym{app}{APP}{Australian Privacy Principles}
\newacronym{iso}{ISO}{International Organization for Standardization}
\newacronym{mlops}{MLOps}{Machine Learning Operations}
\newacronym{srm}{SRM}{Shared Responsibility Model}
\newacronym{ip}{IP}{Internet Protocol}
\newacronym{yaml}{YAML}{YAML Ain't Markup Language}
\newacronym{it}{IT}{Information Technology}
\newacronym{idps}{IDPS}{Intrusion Detection and Prevention System}
\newacronym{devsecops}{DevSecOps}{Development, Security, and Operations}
\newacronym{mttd}{MTTD}{Mean Time to Detect}
\newacronym{mttr}{MTTR}{Mean Time to Resolve}
\newacronym{prisma}{PRISMA}{Preferred Reporting Items for Systematic Reviews and Meta-Analyses}
\newacronym{sql}{SQL}{Structured Query Language}
\newacronym{gdpr}{GDPR}{General Data Protection Regulation}
\newacronym{rmf}{RMF}{Risk Management Framework}
\newacronym{zta}{ZTA}{Zero Trust Architecture}
\newacronym{iac}{IaC}{Infrastructure-as-Code}
\newacronym{devops}{DevOps}{Development and Operations}
\newacronym{cicd}{CI/CD}{Continuous Integration/Continuous Deployment}
\newacronym{sast}{SAST}{static analysis security testing}
\newacronym{cis}{CIS}{Center for Internet Security}
\newacronym{opa}{OPA}{Open Policy Agent}
\newacronym{json}{JSON}{JavaScript Object Notation}
\newacronym{s3}{S3}{Simple Storage Service}
\newacronym{iam}{IAM}{Identity and Access Management}
\newacronym{ec2}{EC2}{Elastic Compute Cloud}
\newacronym{ssh}{SSH}{Secure Shell}
\newacronym{os}{OS}{Operating System}
\newacronym{hitl}{HITL}{Human-in-the-Loop}
\newacronym{pac}{PaC}{Policy-as-Code}


\newacronym{mlm}{MLM}{Machine Learning Model}
\newacronym{nlp}{NLP}{Natural Language Processing}
\newacronym{rest}{REST}{Representational State Transfer}
\newacronym{xml}{XML}{eXtensible Markup Language}
\newacronym{http}{HTTP}{HyperText Transfer Protocol}
\newacronym{https}{HTTPS}{HyperText Transfer Protocol Secure}
\newacronym{ssl}{SSL}{Secure Sockets Layer}
\newacronym{tls}{TLS}{Transport Layer Security}
\newacronym{rbac}{RBAC}{Role-Based Access Control}
\newacronym{abac}{ABAC}{Attribute-Based Access Control}
\newacronym{oauth}{OAuth}{Open Authorization}
\newacronym{saml}{SAML}{Security Assertion Markup Language}
\newacronym{oidc}{OIDC}{OpenID Connect}
\newacronym{mfa}{MFA}{Multi-Factor Authentication}
\newacronym{sso}{SSO}{Single Sign-On}
\newacronym{vpc}{VPC}{Virtual Private Cloud}
\newacronym{vpn}{VPN}{Virtual Private Network}
\newacronym{cdn}{CDN}{Content Delivery Network}
\newacronym{dns}{DNS}{Domain Name System}
\newacronym{ddos}{DDoS}{Distributed Denial of Service}
\newacronym{waf}{WAF}{Web Application Firewall}
\newacronym{ids}{IDS}{Intrusion Detection System}
\newacronym{ips}{IPS}{Intrusion Prevention System}
\newacronym{siem}{SIEM}{Security Information and Event Management}
\newacronym{soar}{SOAR}{Security Orchestration, Automation and Response}
\newacronym{ciso}{CISO}{Chief Information Security Officer}
\newacronym{hipaa}{HIPAA}{Health Insurance Portability and Accountability Act}
\newacronym{sox}{SOX}{Sarbanes-Oxley Act}
\newacronym{pci}{PCI}{Payment Card Industry}
\newacronym{dsci}{DSS}{Data Security Standard}
\newacronym{cvss}{CVSS}{Common Vulnerability Scoring System}
\newacronym{cve}{CVE}{Common Vulnerabilities and Exposures}
\newacronym{cwe}{CWE}{Common Weakness Enumeration}
\newacronym{mitre}{MITRE}{MITRE Corporation}
\newacronym{azure}{Azure}{Microsoft Azure}
\newacronym{gcp}{GCP}{Google Cloud Platform}
\newacronym{rds}{RDS}{Relational Database Service}
\newacronym{lambda}{Lambda}{AWS Lambda}
\newacronym{ecs}{ECS}{Elastic Container Service}
\newacronym{eks}{EKS}{Elastic Kubernetes Service}
\newacronym{k8s}{K8s}{Kubernetes}
\newacronym{docker}{Docker}{Docker containerization platform}
\newacronym{git}{Git}{Git version control system}
\newacronym{github}{GitHub}{GitHub platform}
\newacronym{gitlab}{GitLab}{GitLab platform}
\newacronym{jenkins}{Jenkins}{Jenkins automation server}
\newacronym{terraform}{Terraform}{Terraform infrastructure as code}
\newacronym{ansible}{Ansible}{Ansible automation platform}
\newacronym{chef}{Chef}{Chef configuration management}
\newacronym{puppet}{Puppet}{Puppet configuration management}
\newacronym{kms}{KMS}{Key Management Service}
\newacronym{hsm}{HSM}{Hardware Security Module}
\newacronym{pki}{PKI}{Public Key Infrastructure}
\newacronym{ca}{CA}{Certificate Authority}
\newacronym{csr}{CSR}{Certificate Signing Request}
\newacronym{crl}{CRL}{Certificate Revocation List}
\newacronym{ocsp}{OCSP}{Online Certificate Status Protocol}
\newacronym{ml}{ML}{Machine Learning}
\newacronym{dl}{DL}{Deep Learning}
\newacronym{ann}{ANN}{Artificial Neural Network}
\newacronym{cnn}{CNN}{Convolutional Neural Network}
\newacronym{rnn}{RNN}{Recurrent Neural Network}
\newacronym{lstm}{LSTM}{Long Short-Term Memory}
\newacronym{gru}{GRU}{Gated Recurrent Unit}
\newacronym{gan}{GAN}{Generative Adversarial Network}
\newacronym{vae}{VAE}{Variational Autoencoder}
\newacronym{bert}{BERT}{Bidirectional Encoder Representations from Transformers}
\newacronym{gpt}{GPT}{Generative Pre-trained Transformer}
\newacronym{cpu}{CPU}{Central Processing Unit}
\newacronym{gpu}{GPU}{Graphics Processing Unit}
\newacronym{tpu}{TPU}{Tensor Processing Unit}
\newacronym{ram}{RAM}{Random Access Memory}
\newacronym{ssd}{SSD}{Solid State Drive}
\newacronym{hdd}{HDD}{Hard Disk Drive}
\newacronym{iops}{IOPS}{Input/Output Operations Per Second}
\newacronym{load-balancer}{LB}{Load Balancer}
\newacronym{auto-scaling}{AS}{Auto Scaling}
\newacronym{high-availability}{HA}{High Availability}
\newacronym{rpo}{RPO}{Recovery Point Objective}
\newacronym{rto}{RTO}{Recovery Time Objective}
\newacronym{sla}{SLA}{Service Level Agreement}
\newacronym{slo}{SLO}{Service Level Objective}
\newacronym{sli}{SLI}{Service Level Indicator}
\newacronym{mtbf}{MTBF}{Mean Time Between Failures}
\newacronym{nosql}{NoSQL}{Not Only SQL}
\newacronym{oltp}{OLTP}{Online Transaction Processing}
\newacronym{olap}{OLAP}{Online Analytical Processing}
\newacronym{etl}{ETL}{Extract, Transform, Load}
\newacronym{data-warehouse}{DW}{Data Warehouse}
\newacronym{data-lake}{DL}{Data Lake}
\newacronym{big-data}{BD}{Big Data}
\newacronym{hadoop}{Hadoop}{Apache Hadoop}
\newacronym{spark}{Spark}{Apache Spark}
\newacronym{kafka}{Kafka}{Apache Kafka}
\newacronym{elasticsearch}{ES}{Elasticsearch}
\newacronym{kibana}{Kibana}{Kibana}
\newacronym{logstash}{Logstash}{Logstash}
\newacronym{elk}{ELK}{Elasticsearch, Logstash, and Kibana}
\newacronym{grafana}{Grafana}{Grafana}
\newacronym{prometheus}{Prometheus}{Prometheus}
\newacronym{nagios}{Nagios}{Nagios}
\newacronym{zabbix}{Zabbix}{Zabbix}
\newacronym{splunk}{Splunk}{Splunk}
\newacronym{zero-trust}{ZT}{Zero Trust}
\newacronym{ztna}{ZTNA}{Zero Trust Network Access}
\newacronym{sase}{SASE}{Secure Access Service Edge}
\newacronym{sd-wan}{SD-WAN}{Software-Defined Wide Area Network}
\newacronym{casb}{CASB}{Cloud Access Security Broker}
\newacronym{dlp}{DLP}{Data Loss Prevention}
\newacronym{ueba}{UEBA}{User and Entity Behavior Analytics}
\newacronym{xdr}{XDR}{Extended Detection and Response}
\newacronym{edr}{EDR}{Endpoint Detection and Response}
\newacronym{ndr}{NDR}{Network Detection and Response}
\newacronym{mdr}{MDR}{Managed Detection and Response}
\newacronym{pg}{PG}{Policy Generation}

%----------------------------------------------------------------------------------------
%	GLOSSARY TERMS
%----------------------------------------------------------------------------------------

\newglossaryentry{hyperscale}{
    name=hyperscale,
    description={A type of computing infrastructure that can scale appropriately as increased demand is added to the system. Hyperscale computing is often associated with big data and cloud computing environments}
}

\newglossaryentry{cloud-native}{
    name=cloud-native,
    description={An approach to building and running applications that exploits the advantages of the cloud computing delivery model. Cloud-native technologies empower organizations to build and run scalable applications in modern, dynamic environments such as public, private, and hybrid clouds}
}

\newglossaryentry{microservices}{
    name=microservices,
    description={An architectural approach to building software applications as a collection of small, loosely coupled services that communicate over well-defined APIs. Each service is independently deployable and scalable}
}

\newglossaryentry{containerization}{
    name=containerization,
    description={A method of operating system virtualization that allows you to run an application and its dependencies in resource-isolated processes. Containers package code and dependencies so applications run quickly and reliably across computing environments}
}

\newglossaryentry{orchestration}{
    name=orchestration,
    description={The automated arrangement, coordination, and management of complex computer systems, middleware, and services. In cloud computing, this often refers to managing containers, services, and infrastructure}
}

\newglossaryentry{infrastructure-as-code}{
    name=infrastructure as code,
    description={The practice of managing and provisioning computing infrastructure through machine-readable definition files, rather than physical hardware configuration or interactive configuration tools}
}

\newglossaryentry{serverless}{
    name=serverless,
    description={A cloud computing execution model where the cloud provider dynamically manages the allocation and provisioning of servers. The application code runs in stateless compute containers that are event-triggered and fully managed by the cloud provider}
}

\newglossaryentry{edge-computing}{
    name=edge computing,
    description={A distributed computing paradigm that brings computation and data storage closer to the location where it is needed to improve response times and save bandwidth}
}

\newglossaryentry{multi-cloud}{
    name=multi-cloud,
    description={A strategy that uses cloud services from multiple cloud providers. This approach helps avoid vendor lock-in, increases redundancy, and allows organizations to choose the best services from different providers}
}

\newglossaryentry{hybrid-cloud}{
    name=hybrid cloud,
    description={A computing environment that combines on-premises infrastructure with public and private cloud services, with orchestration between the platforms}
}

\newglossaryentry{load-balancing}{
    name=load balancing,
    description={The process of distributing network traffic across multiple servers to ensure optimal resource utilization, minimize response time, and avoid overloading any single server}
}

\newglossaryentry{threat-modeling}{
    name=threat modeling,
    description={A structured approach to identify, quantify, and address security threats and vulnerabilities in a system. It helps understand the attack surface and prioritize security controls}
}

\newglossaryentry{security-posture}{
    name=security posture,
    description={The overall security status of an organization's networks, information, and systems based on information security resources and capabilities in place to manage the defense of the enterprise}
}

\newglossaryentry{attack-surface}{
    name=attack surface,
    description={The sum of all possible attack vectors that an unauthorized user could use to enter a system and potentially extract data or cause damage}
}

\newglossaryentry{vulnerability-assessment}{
    name=vulnerability assessment,
    description={The process of identifying, quantifying, and prioritizing vulnerabilities in a system. It provides a baseline for understanding security weaknesses and their potential impact}
}

\newglossaryentry{penetration-testing}{
    name=penetration testing,
    description={A simulated cyber attack against a computer system to check for exploitable vulnerabilities. It involves actively probing systems for security weaknesses that could be exploited by attackers}
}

\newglossaryentry{security-automation}{
    name=security automation,
    description={The use of technology to perform security tasks with minimal human intervention. It includes automated threat detection, incident response, compliance monitoring, and vulnerability management}
}

\newglossaryentry{incident-response}{
    name=incident response,
    description={A structured approach to addressing and managing the aftermath of a security breach or cyber attack. The goal is to handle the situation in a way that limits damage and reduces recovery time and costs}
}

\newglossaryentry{threat-intelligence}{
    name=threat intelligence,
    description={Evidence-based knowledge about existing or emerging threats to assets that can be used to inform decisions regarding the subject's response to those threats}
}

\newglossaryentry{behavioral-analytics}{
    name=behavioral analytics,
    description={The use of data science techniques to analyze user and entity behavior patterns to detect anomalies that may indicate security threats or insider risks}
}

\newglossaryentry{machine-learning-model}{
    name=machine learning model,
    description={A mathematical representation of a real-world process, created by training an algorithm on data. In cybersecurity, models are used for threat detection, anomaly detection, and risk assessment}
}

\newglossaryentry{training-data}{
    name=training data,
    description={The dataset used to teach a machine learning algorithm. The quality and quantity of training data significantly impact the performance and accuracy of the resulting model}
}

\newglossaryentry{model-inference}{
    name=model inference,
    description={The process of using a trained machine learning model to make predictions or decisions on new, unseen data. In security contexts, this often involves real-time threat detection}
}

\newglossaryentry{false-positive}{
    name=false positive,
    description={An alert or detection that incorrectly identifies normal, benign activity as malicious or threatening. High false positive rates can overwhelm security teams and reduce the effectiveness of security systems}
}

\newglossaryentry{false-negative}{
    name=false negative,
    description={A failure to detect or alert on actual malicious activity. False negatives are particularly dangerous in security contexts as they represent missed threats}
}

\newglossaryentry{data-privacy}{
    name=data privacy,
    description={The aspect of information privacy that deals with the proper handling, processing, storage, and usage of personal information. It encompasses both technological and legal frameworks}
}

\newglossaryentry{data-sovereignty}{
    name=data sovereignty,
    description={The concept that digital data is subject to the laws and governance structures of the nation where it is collected or processed. This is particularly important in cloud computing scenarios}
}

\newglossaryentry{compliance-framework}{
    name=compliance framework,
    description={A structured set of guidelines that details an organization's processes for maintaining accordance with established regulations, specifications, or legislation}
}

\newglossaryentry{risk-assessment}{
    name=risk assessment,
    description={The systematic process of identifying, analyzing, and evaluating risks to determine their potential impact on business operations and the likelihood of occurrence}
}

\newglossaryentry{business-continuity}{
    name=business continuity,
    description={An organization's ability to maintain essential functions during and after a disaster or disruption. It encompasses planning, procedures, and systems needed to continue operations}
}

\newglossaryentry{disaster-recovery}{
    name=disaster recovery,
    description={The process of resuming normal business operations after a disruptive event. It focuses on restoring IT infrastructure, data, and systems to minimize downtime and data loss}
}
