% Chapter 1

\chapter{Introduction}
\label{chap:introduction}

%----------------------------------------------------------------------------------------

% Define some commands to keep the formatting separated from the content 
\newcommand{\keyword}[1]{\textbf{#1}}
\newcommand{\tabhead}[1]{\textbf{#1}}
\newcommand{\code}[1]{\texttt{#1}}
\newcommand{\file}[1]{\texttt{\bfseries#1}}
\newcommand{\option}[1]{\texttt{\itshape#1}}

%----------------------------------------------------------------------------------------

\section{Motivation and Problem Statement}
\label{sec:motivation_problem}

The modern software development landscape is dominated by the widespread adoption of hyperscale cloud platforms and the practice of managing infrastructure as code (IaC). This paradigm enables organizations to build and deploy applications with unprecedented speed and agility. However, this velocity comes at a cost: the scale, complexity, and dynamic nature of these environments have outpaced the capabilities of traditional, manual security practices\cite{khanna_enhancing_2024}.

Security teams are faced with a deluge of complex IaC configurations that change daily, making manual analysis slow, error-prone, and insufficient for modern DevSecOps cycles\cite{gunathilaka_context-aware_2025}. Misconfigurations have become a leading cause of cloud security breaches, and the manual creation of preventative security policies acts as a significant bottleneck, hindering development velocity\cite{tunc_cloud_2017, fu_ai_2025}. This creates a critical security gap between the speed of development and the pace of security assurance.

This thesis confronts this challenge by investigating the convergence of Generative AI (GenAI) and cloud security. The core problem this research addresses is the need for advanced, intelligent automation that can proactively analyze IaC configurations and automatically generate precise, preventative security controls. By leveraging GenAI, this work aims to create a system that keeps pace with rapid development cycles, reduces the manual burden on security teams, and embeds security directly into the development workflow, truly shifting security left.

% Section 1.2: Clearly state what this thesis aims to achieve and the questions it seeks to answer.
% This should be derived from your work in Chapters 4 and 5.
\section{Research Objectives and Questions}
\label{sec:objectives_questions}

This thesis explores how Generative AI (GenAI) can help solve the security challenges in modern cloud environments. The research focuses on using GenAI to automate security tasks and asks the following central question:

\textit{How can Generative AI technologies be effectively leveraged to automate security operations across hyperscale cloud platforms?}

To comprehensively answer this overarching question, the research is decomposed into four distinct but interrelated sub-questions.
The first sub-question explores the effectiveness and automation of GenAI for security policy generation: \textit{How can Generative AI technologies be effectively leveraged to automate security policy generation and management across hyperscale cloud platforms?}
The second investigates the architecture and orchestration needed for trust and accuracy: \textit{What specific architectural patterns and validation mechanisms are required to ensure trust, accuracy, and effective multi-cloud orchestration in GenAI-driven security automation?}
The third seeks to determine how to measure and validate the effectiveness of this automation: \textit{How can the effectiveness of GenAI-driven security automation be quantitatively measured and validated, particularly in terms of accuracy, reliability, and efficiency gains?}
Finally, the fourth sub-question explores the role of human oversight: \textit{What is the optimal balance between automation and human oversight to maximize security outcomes and mitigate the risks of GenAI-driven policy generation?}

These questions guide the structure of the thesis. The current state of the art is examined in Chapter~\ref{chap:Background and Related Work}. The conceptual framework to address the question of architecture and orchestration is detailed in Chapter~\ref{chap:conceptual_framework}. The practical implementation and validation of the framework, which addresses the remaining questions, are described in Chapter~\ref{chap:implementation}, and the results are evaluated in Chapter~\ref{chap:results}.

By methodically addressing these questions, this thesis aims to provide a comprehensive, empirically grounded contribution to the field of automated cloud security.

\section{Contribution and Significance}
\label{sec:contribution}

This thesis delivers two primary contributions to the field of automated cloud security: a novel conceptual framework and a functional prototype that provides an empirical validation of that framework.

First, it introduces a novel conceptual framework for GenAI-driven security automation. The framework's innovation lies in its hybrid architectural model, which systematically integrates the speed of traditional static analysis with the deep contextual reasoning of a Large Language Model (LLM). It leverages a Retrieval-Augmented Generation (RAG) architecture to ground the LLM's output in curated data—mitigating inaccuracies and incorporates a multi-stage validation process with a mandatory Human-in-the-Loop (HITL) review to ensure a critical balance between automation and expert oversight.

Second, this work presents the design and implementation of a functional prototype that realizes the conceptual model. More than just a theoretical construct, the prototype is a working system built with industry-standard technologies, including Amazon Web Services (AWS), Terraform for Infrastructure-as-Code (IaC), and Open Policy Agent (OPA) for policy enforcement. This demonstrates the practical feasibility and applicability of the proposed approach in a real-world technological stack.

The significance of this work is threefold. It advances the shift-left security paradigm by providing an automated mechanism for creating preventative controls early in the development lifecycle \cite{akto_shift_2025}. It also serves as a practical blueprint for organizations seeking to engineer and integrate GenAI into their DevSecOps pipelines. Finally, by automating a traditionally manual and time-consuming task, this research helps reduce the cognitive load and operational burden on security teams, allowing them to focus on higher-value strategic initiatives and ultimately bridging the critical gap between high-velocity development and robust security assurance.

\section{Outline of the Thesis}
\label{sec:outline}

This thesis is structured into eight chapters to systematically address the research questions and validate the proposed solution. Chapter~\ref{chap:introduction} introduces the research by defining the motivation and problem statement, outlining the core research objectives and questions, and summarizing the primary contributions and significance of the work. Chapter~\ref{chap:Background and Related Work} provides the necessary theoretical foundation, covering key concepts in cloud security and Generative AI, and includes a comprehensive literature review that analyzes the current state of the art and identifies the critical research gaps that this thesis aims to address. Chapter~\ref{chap:methodology} details the research approach, justifying the use of a Design Science Research paradigm and outlining the four-phase process followed in this study: a foundational literature review, the development of the conceptual framework, the implementation of a prototype, and its empirical evaluation. Chapter~\ref{chap:conceptual_framework} presents the architectural blueprint for the proposed solution, detailing its multi-layered design, the integration of a RAG-based LLM for analysis and policy generation, and the essential role of the Human-in-the-Loop validation process. Chapter~\ref{chap:implementation} describes the practical realization of the conceptual framework, covering the technology stack, the cloud infrastructure defined as code, the Python application that orchestrates the workflow, and its integration into a CI/CD pipeline. Chapter~\ref{chap:results} presents the empirical findings from the evaluation of the prototype, assessing its performance against a defined set of metrics including policy generation accuracy and effectiveness, generation speed, and the quality of its contextual reasoning. Chapter~\ref{chap:discussion} interprets these results, analyzing the key findings and their implications, addressing the limitations of the current study, and proposing concrete directions for future research. Finally, Chapter~\ref{chap:conclusion} summarizes the entire body of work, revisits the research questions to provide final answers, and reflects on the overall contribution of the thesis to the field of automated cloud security.
