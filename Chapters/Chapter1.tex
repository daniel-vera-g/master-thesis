% Chapter 1

\chapter{Introduction}
\label{chap:introduction}

%----------------------------------------------------------------------------------------

% Define some commands to keep the formatting separated from the content 
\newcommand{\keyword}[1]{\textbf{#1}}
\newcommand{\tabhead}[1]{\textbf{#1}}
\newcommand{\code}[1]{\texttt{#1}}
\newcommand{\file}[1]{\texttt{\bfseries#1}}
\newcommand{\option}[1]{\texttt{\itshape#1}}

%----------------------------------------------------------------------------------------

% Section 1.1: Start with the broad context and then narrow down to the specific problem.
% Context: The increasing adoption of hyperscale cloud platforms and Infrastructure-as-Code (IaC).
% Problem: The scale and complexity of these environments make manual security analysis and policy creation slow, error-prone, and insufficient for modern DevSecOps cycles.
\section{Motivation and Problem Statement}
\label{sec:motivation_problem}
% In this section, introduce the convergence of GenAI and cloud security.
% State the core problem: the need for advanced automation to analyze IaC configurations and proactively generate preventative security controls to keep pace with rapid development cycles.

TBD

% Section 1.2: Clearly state what this thesis aims to achieve and the questions it seeks to answer.
% This should be derived from your work in Chapters 4 and 5.
\section{Research Objectives and Questions}
\label{sec:objectives_questions}

This thesis explores how Generative AI (GenAI) can help solve the security challenges in modern cloud environments. The research focuses on using GenAI to automate security tasks and asks the following central question:

\textit{How can Generative AI technologies be effectively leveraged to automate security operations across hyperscale cloud platforms?}

To comprehensively answer this overarching question, the research is decomposed into four distinct but interrelated sub-questions.
The first sub-question explores the effectiveness and automation of GenAI for security policy generation: \textit{How can Generative AI technologies be effectively leveraged to automate security policy generation and management across hyperscale cloud platforms?}
The second investigates the architecture and orchestration needed for trust and accuracy: \textit{What specific architectural patterns and validation mechanisms are required to ensure trust, accuracy, and effective multi-cloud orchestration in GenAI-driven security automation?}
The third seeks to determine how to measure and validate the effectiveness of this automation: \textit{How can the effectiveness of GenAI-driven security automation be quantitatively measured and validated, particularly in terms of accuracy, reliability, and efficiency gains?}
Finally, the fourth sub-question explores the role of human oversight: \textit{What is the optimal balance between automation and human oversight to maximize security outcomes and mitigate the risks of GenAI-driven policy generation?}

These questions guide the structure of the thesis. The current state of the art is examined in Chapter~\ref{chap:Background and Related Work}. The conceptual framework to address the question of architecture and orchestration is detailed in Chapter~\ref{chap:conceptual_framework}. The practical implementation and validation of the framework, which addresses the remaining questions, are described in Chapter~\ref{chap:implementation}, and the results are evaluated in Chapter~\ref{chap:results}.

By methodically addressing these questions, this thesis aims to provide a comprehensive, empirically grounded contribution to the field of automated cloud security.

% Section 1.3: Explain the unique contribution of your work.
% Your main contributions are the conceptual framework (Ch 4) and the prototype implementation (Ch 5).
\section{Contribution and Significance}
\label{sec:contribution}
% Detail the key contributions:
% 1. A novel conceptual framework for GenAI-driven security automation that integrates static analysis, a RAG-based LLM approach, and a multi-stage validation process.
% 2. The design and implementation of a prototype system that validates this framework using industry-standard technologies (AWS, Terraform, OPA).
% Explain the significance: This work advances "shift-left" security, provides a practical blueprint for integrating GenAI into DevSecOps pipelines, and reduces the manual burden on security teams.

TBD

% Section 1.5: Give the reader a roadmap of the entire thesis.
% This is a summary of each chapter's purpose.
\section{Outline of the Thesis}
\label{sec:outline}
% Briefly describe the content and purpose of each subsequent chapter.
% For example:
% \textbf{Chapter 2} provides a comprehensive background on foundational concepts in cloud security and GenAI, reviews the state of the art in related work, and identifies the research gaps that this thesis addresses.
% \textbf{Chapter 3} outlines the research methodology employed in this work.
% \textbf{Chapter 4} introduces the conceptual framework for GenAI-driven security automation, detailing its multi-layered architecture, the integration of LLMs, and the Human-in-the-Loop workflow.
% \textbf{Chapter 5} describes the practical implementation of the framework as a prototype, covering the system architecture, technology stack, and end-to-end workflow.
% \textbf{Chapter 6} presents the results from the evaluation of the prototype based on the metrics defined in the framework.
% \textbf{Chapter 7} discusses the implications of the results, connecting them back to the research questions and the existing literature.
% \textbf{Chapter 8} concludes the thesis by summarizing the key findings, acknowledging limitations, and suggesting directions for future research.

TBD