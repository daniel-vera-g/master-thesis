% Chapter 1

\chapter{Introduction}
\label{chap:introduction}

%----------------------------------------------------------------------------------------

% Define some commands to keep the formatting separated from the content 
\newcommand{\keyword}[1]{\textbf{#1}}
\newcommand{\tabhead}[1]{\textbf{#1}}
\newcommand{\code}[1]{\texttt{#1}}
\newcommand{\file}[1]{\texttt{\bfseries#1}}
\newcommand{\option}[1]{\texttt{\itshape#1}}

%----------------------------------------------------------------------------------------

% Section 1.1: Start with the broad context and then narrow down to the specific problem.
% Context: The increasing adoption of hyperscale cloud platforms and Infrastructure-as-Code (IaC).
% Problem: The scale and complexity of these environments make manual security analysis and policy creation slow, error-prone, and insufficient for modern DevSecOps cycles.
\section{Motivation and Problem Statement}
\label{sec:motivation_problem}
% In this section, introduce the convergence of GenAI and cloud security.
% State the core problem: the need for advanced automation to analyze IaC configurations and proactively generate preventative security controls to keep pace with rapid development cycles.

TBD

% Section 1.2: Clearly state what this thesis aims to achieve and the questions it seeks to answer.
% This should be derived from your work in Chapters 4 and 5.
\section{Research Objectives and Questions}
\label{sec:objectives_questions}

This thesis explores how Generative AI (GenAI) can help solve the security challenges in modern cloud environments. The research focuses on using GenAI to automate security tasks and asks the following central question:

\textit{How can Generative AI technologies be effectively leveraged to automate security operations across hyperscale cloud platforms?}

To comprehensively answer this overarching question, the research is decomposed into three distinct but interrelated areas, each addressed by a set of specific sub-questions.

The first area of investigation focuses on establishing a foundational understanding of the current landscape. This involves a systematic analysis of the state of the art to examine the existing research and implementation of GenAI for security automation at hyperscale cloud providers. This analysis compares existing GenAI toolsets and models across major cloud platforms and identifies the primary limitations and challenges in current GenAI implementations for cloud security. This foundational work is primarily addressed in the literature review presented in Chapter \ref{chap:Background and Related Work}.

The second area of investigation concerns the design of an innovative solution. The goal is to develop a conceptual framework that operationalizes GenAI for security automation. This requires exploring how GenAI can be specifically applied to automate policy generation and management, what architecture is required for effective multi-cloud policy orchestration using GenAI, and what validation mechanisms are necessary to ensure trust and accuracy in GenAI-driven security automation. These questions guide the development of the conceptual framework detailed in Chapter \ref{chap:conceptual_framework}.

Finally, the third area addresses the practical implementation and empirical validation of the proposed framework. This involves determining which technical approach is most effective for implementing GenAI-driven security automation across hyperscale cloud platforms and how the effectiveness of such a system can be measured and validated. Furthermore, this research seeks to understand what balance between automation and human oversight optimizes security outcomes in a real-world context. These questions are addressed through the prototype implementation described in Chapter \ref{chap:implementation} and the evaluation of its results, which will be presented in Chapter \ref{chap:results}.

By methodically addressing these questions, this thesis aims to provide a comprehensive, empirically grounded contribution to the field of automated cloud security.

% Section 1.3: Explain the unique contribution of your work.
% Your main contributions are the conceptual framework (Ch 4) and the prototype implementation (Ch 5).
\section{Contribution and Significance}
\label{sec:contribution}
% Detail the key contributions:
% 1. A novel conceptual framework for GenAI-driven security automation that integrates static analysis, a RAG-based LLM approach, and a multi-stage validation process.
% 2. The design and implementation of a prototype system that validates this framework using industry-standard technologies (AWS, Terraform, OPA).
% Explain the significance: This work advances "shift-left" security, provides a practical blueprint for integrating GenAI into DevSecOps pipelines, and reduces the manual burden on security teams.

TBD

% Section 1.4: Briefly explain HOW you will answer your research questions.
% This should be a high-level summary of the approach detailed in Chapter 3 and executed in Chapters 4-6.
\section{Methodology Overview}
\label{sec:methodology_overview}
% Briefly describe your research approach. For example, mention that the thesis follows a design science methodology, involving:
% 1. A comprehensive literature review to identify research gaps (Chapter 2).
% 2. The development of a conceptual framework (Chapter 4).
% 3. The implementation of a functional prototype (Chapter 5).
% 4. An evaluation of the prototype against defined metrics such as policy efficacy and generation speed (Chapter 6).

TBD or separate section?

% Section 1.5: Give the reader a roadmap of the entire thesis.
% This is a summary of each chapter's purpose.
\section{Outline of the Thesis}
\label{sec:outline}
% Briefly describe the content and purpose of each subsequent chapter.
% For example:
% \textbf{Chapter 2} provides a comprehensive background on foundational concepts in cloud security and GenAI, reviews the state of the art in related work, and identifies the research gaps that this thesis addresses.
% \textbf{Chapter 3} outlines the research methodology employed in this work.
% \textbf{Chapter 4} introduces the conceptual framework for GenAI-driven security automation, detailing its multi-layered architecture, the integration of LLMs, and the Human-in-the-Loop workflow.
% \textbf{Chapter 5} describes the practical implementation of the framework as a prototype, covering the system architecture, technology stack, and end-to-end workflow.
% \textbf{Chapter 6} presents the results from the evaluation of the prototype based on the metrics defined in the framework.
% \textbf{Chapter 7} discusses the implications of the results, connecting them back to the research questions and the existing literature.
% \textbf{Chapter 8} concludes the thesis by summarizing the key findings, acknowledging limitations, and suggesting directions for future research.

TBD