% Chapter Template

\chapter{Conceptual Framework for GenAI-Driven Security Automation} % Main chapter title
% \chapter{Framework Development} % Main chapter title

\section{Architectural Overview of the Proposed Framework}

% TODO
% Sources of Security Data
% Data Ingestion and Preprocessing

\subsection{Core components} % (fold)
\label{sub:Core components}

The core components are meant to...

\subsubsection{Data Ingestion Layer} % (fold)
\label{sec:Data Ingestion Layer}

The Data Ingestion Layer serves as the foundational entry point for security artifacts into the automation framework. Its primary function is to ingest Infrastructure-as-Code (IaC) configurations, with a specific focus on Terraform code, which is a prevalent standard for provisioning and managing cloud infrastructure. The reliance on IaC, while enhancing automation and consistency, introduces significant risks such as misconfigurations, coding errors, and embedded secrets, making automated analysis a critical requirement for secure cloud operations \cite{hayagreevan_security_2024}.

This layer is designed to support both batch and real-time ingestion modes, a flexible approach that aligns with modern data pipeline architectures emphasizing scalability and performance \cite{ismail_big_2025}. Batch ingestion allows for comprehensive, scheduled scans of entire code repositories, while real-time ingestion facilitates immediate analysis within continuous integration and continuous delivery (CI/CD) pipelines. The framework accepts Terraform code through a command-line interface, ensuring seamless integration into existing developer workflows and automated systems.

Upon ingestion, the layer initiates a multi-stage preliminary analysis process as defined by the prototype architecture diagram. First, the raw Terraform code is parsed for programmatic analysis. Following this step, a suite of established static analysis security testing (SAST) tools—including tfsec, Trivy, Checkov, and Terrascan—is executed. This initial scan generates a baseline vulnerability report by checking the code against a comprehensive database of known misconfigurations, security vulnerabilities, and compliance violations. The structured output from this layer, comprising the original code, its AST representation, and the baseline vulnerability report, is then passed to the Data Processing Layer for the deeper, context-aware analysis powered by generative AI that is the focus of this research. In the following, the processes of the Data processing layer are explained more in Detail

% subsubsection Data Ingestion Layer (end)

\subsubsection{Data Processing Layer} % (fold)
\label{sec:Data Processing Layer}

Following the Data Ingestion Layer, the Data Processing Layer is responsible for the core analysis of the ingested Infrastructure-as-Code (IaC) artifacts. A central design principle of this framework is the segregation of processing activities into two distinct but complementary sub-layers: a traditional Static Code Analysis engine and an advanced Generative AI (GenAI) Analysis Engine.

The rationale for this dual-layer architecture is to create a highly efficient and comprehensive security analysis pipeline. This approach leverages the respective strengths of each technology. Static analysis provides a rapid, reliable, and computationally inexpensive method for identifying a wide range of known, pattern-based vulnerabilities. By filtering out these common issues first, the framework can then employ the more resource-intensive GenAI engine to focus on complex, context-dependent security flaws that traditional tools are ill-equipped to detect. This layered methodology optimizes analytical depth while maintaining operational efficiency, ensuring that both well-defined and nuanced vulnerabilities are addressed.

The first stage of this layer employs a suite of established static analysis security testing (SAST) tools to conduct an initial scan of the Terraform code. This engine examines the code for syntactic and structural flaws by referencing curated databases of known vulnerabilities, common misconfigurations, and code smells. It validates the code against established security benchmarks and standards, such as those published by the Center for Internet Security (CIS). The primary output of this stage is a baseline vulnerability report, which provides a structured list of potential issues identified through deterministic, rule-based pattern matching. This report serves as a foundational input for the subsequent, more sophisticated analysis stage.

The second stage is the GenAI Analysis Engine, which represents the core innovation of this framework and directly addresses the research interest in applying generative AI to cloud security. This engine utilizes Large Language Models (LLMs) to perform a deeper, contextual analysis that transcends the limitations of traditional static scanners\cite{hayagreevan_security_2024, ling_enhancing_2024}. It takes as input both the original Terraform code and the baseline vulnerability report from the previous stage, using the initial findings to enrich its analytical context.

% TODO hier ein paar Beispiele

This engine is designed to identify security weaknesses that require an understanding of developer intent, architectural relationships, and complex business logic\cite{noseevich_towards_2015}. Its capabilities include:

\begin{itemize}
    \item \textbf{Identifying Context-Sensitive Flaws:} Detecting risks that emerge from the interaction of multiple configurations, such as overly permissive network rules that appear acceptable in isolation but create a vulnerability when combined with a specific resource's placement within the network architecture\cite{noseevich_towards_2015}.
    \item \textbf{Uncovering Logical and Policy Violations:} Identifying logical flaws in resource deployments, such as potential circular dependencies, or violations of complex, unwritten organizational policies like nuanced tagging and naming conventions.
    \item \textbf{Reducing False Positives:} Differentiating between genuine security risks and findings from the static analysis that are benign within a specific operational context, such as a "hardcoded secret" that is merely a placeholder for a non-production environment.
\end{itemize}

By synthesizing information from the code and the initial scan, the GenAI Analysis Engine bridges the gap between traditional, rule-based detection and adaptive, context-aware threat identification, producing a consolidated and enriched vulnerability report.

% subsubsection Data Processing Layer (end)

\subsubsection{Validation Layer} % (fold)
\label{sec:Validation Layer}

 **Purpose:**  
  Ensures the integrity, correctness, and security of both the analysis results and any generated artifacts before they progress further in the pipeline.
- **Key Functions:**  
  - Validates outputs from both static and GenAI analysis for consistency and accuracy.
  - Applies automated and, where necessary, human-in-the-loop review mechanisms to mitigate risks of false positives, hallucinations, or overlooked vulnerabilities[1][5].
- **Security and Compliance:**  
  - Enforces policy compliance and auditability, supporting traceability and regulatory requirements[1][5].
  - Functions as a critical checkpoint to prevent propagation of errors or insecure code downstream.

% subsubsection Validation Layer (end)

\subsubsection{Code Generation Layr} % (fold)
\label{sec:Code Generation Laye}

- **Purpose:**  
  Automates the creation of security policies, remediation scripts, or configuration changes using GenAI, followed by rigorous validation to ensure quality and safety.
- **Key Functions:**  
  - Leverages LLMs to generate code or policies (e.g., Rego, JSON) tailored to the detected issues and organizational standards[5].
  - Subject to multi-stage validation, including:
    - Automated code review and vulnerability scanning.
    - Manual review for high-impact changes or where AI uncertainty is high[5][7].
    - Differential and metamorphic testing to detect inconsistencies or unintended behaviors[7].
  - Maintains detailed audit logs for traceability and compliance, supporting both operational monitoring and forensic analysis[5].
- **Security and Governance:**  
  - Integrates access controls and authentication to restrict code generation capabilities to authorized entities[5].
  - Ensures that only validated, secure, and compliant code is promoted to deployment or enforcement stages.

% subsubsection Code Generation Laye (end)

% subsection Core components (end)

\subsection{Integration with Hyperscale Cloud Platforms} % (fold)
\label{sec:Integration with Hyperscale Cloud Platforms}

% subsubsection Integration with Hyperscale Cloud Platforms (end)

\subsection{Integration of GenAI-Driven Security Automation} % (fold)
\label{sub:Integration of GenAI-Driven Security Automation}

% TODO
% Use of Retrieval-Augmented Generation (RAG) for Enhanced Accuracy 
% Use of custom System Prompts
% Large Language Models used

% subsection Integration of GenAI-Driven Security Automation (end)

\subsection{Leveraging LLMs for Deeper Contextual Analysis} % (fold)
\label{sec:Leveraging LLMs for Deeper Contextual Analysis}

% TODO
% Identifying Context-Sensitive Security Weaknesses
% Detecting Wrongly Flagged Issues
% Uncovering Suboptimal or Inefficient Configurations
% Spotting Violations of Complex Organizational Policies
% Identifying Logical Flaws and Missing Security Basics

% subsubsection Leveraging LLMs for Deeper Contextual Analysis (end)

\subsection{Using LLMs to Generate Security Policies} % (fold)
\label{sec:Using LLMs to Generate Security Policies}

% TODO
% Incorporating Security Standards and Best Practices into Policy Generation
% Validation of Generated Policies (JSON/Rego Validators) 
% Validation and Trust Mechanisms

% subsubsection Using LLMs to Generate Security Policies (end)

\subsection{Metrics for Security Posture Assessment} % (fold)
\label{sec:Metrics for Security Posture Assessment}

% subsubsection Metrics for Security Posture Assessment (end)

\subsection{Human-in-the-Loop for Review and Approval} % (fold)
\label{sub:Human-in-the-Loop for Review and Approval}

% subsection Human-in-the-Loop for Review and Approval (end)

\subsection{Integration with CI/CD Pipelines for Policy-as-Code} % (fold)
\label{sec:Integration with CI/CD Pipelines for Policy-as-Code}

% subsubsection Integration with CI/CD Pipelines for Policy-as-Code  (end)

%----------------------------------------------------------------------------------------
%	SECTION 1
%----------------------------------------------------------------------------------------

\section{Main Section 1}

Lorem ipsum dolor sit amet, consectetur adipiscing elit. Aliquam ultricies lacinia euismod. Nam tempus risus in dolor rhoncus in interdum enim tincidunt. Donec vel nunc neque. In condimentum ullamcorper quam non consequat. Fusce sagittis tempor feugiat. Fusce magna erat, molestie eu convallis ut, tempus sed arcu. Quisque molestie, ante a tincidunt ullamcorper, sapien enim dignissim lacus, in semper nibh erat lobortis purus. Integer dapibus ligula ac risus convallis pellentesque.

%-----------------------------------
%	SUBSECTION 1
%-----------------------------------
\subsection{Subsection 1}

Nunc posuere quam at lectus tristique eu ultrices augue venenatis. Vestibulum ante ipsum primis in faucibus orci luctus et ultrices posuere cubilia Curae; Aliquam erat volutpat. Vivamus sodales tortor eget quam adipiscing in vulputate ante ullamcorper. Sed eros ante, lacinia et sollicitudin et, aliquam sit amet augue. In hac habitasse platea dictumst.

%-----------------------------------
%	SUBSECTION 2
%-----------------------------------

\subsection{Subsection 2}
Morbi rutrum odio eget arcu adipiscing sodales. Aenean et purus a est pulvinar pellentesque. Cras in elit neque, quis varius elit. Phasellus fringilla, nibh eu tempus venenatis, dolor elit posuere quam, quis adipiscing urna leo nec orci. Sed nec nulla auctor odio aliquet consequat. Ut nec nulla in ante ullamcorper aliquam at sed dolor. Phasellus fermentum magna in augue gravida cursus. Cras sed pretium lorem. Pellentesque eget ornare odio. Proin accumsan, massa viverra cursus pharetra, ipsum nisi lobortis velit, a malesuada dolor lorem eu neque.

%----------------------------------------------------------------------------------------
%	SECTION 2
%----------------------------------------------------------------------------------------

\section{Main Section 2}

Sed ullamcorper quam eu nisl interdum at interdum enim egestas. Aliquam placerat justo sed lectus lobortis ut porta nisl porttitor. Vestibulum mi dolor, lacinia molestie gravida at, tempus vitae ligula. Donec eget quam sapien, in viverra eros. Donec pellentesque justo a massa fringilla non vestibulum metus vestibulum. Vestibulum in orci quis felis tempor lacinia. Vivamus ornare ultrices facilisis. Ut hendrerit volutpat vulputate. Morbi condimentum venenatis augue, id porta ipsum vulputate in. Curabitur luctus tempus justo. Vestibulum risus lectus, adipiscing nec condimentum quis, condimentum nec nisl. Aliquam dictum sagittis velit sed iaculis. Morbi tristique augue sit amet nulla pulvinar id facilisis ligula mollis. Nam elit libero, tincidunt ut aliquam at, molestie in quam. Aenean rhoncus vehicula hendrerit.
