\chapter{Research Methodology}
\label{chap:methodology}

This chapter outlines the research methodology employed to answer the research questions posed in Chapter \ref{chap:introduction}. The research is centered on the development and evaluation of a novel solution for automating security in hyperscale cloud platforms. Therefore, a **Design Science Research (DSR)** approach was adopted as the guiding paradigm for this thesis.

DSR is a problem-solving paradigm that seeks to create and evaluate innovative artifacts intended to solve identified organizational problems. This chapter first introduces DSR and justifies its suitability for this project. It then details the specific research process followed, from problem identification through to the evaluation of the developed prototype.

\section{Research Paradigm: Design Science}
\label{sec:research_paradigm}

Design Science Research is fundamentally concerned with creating and evaluating IT artifacts (constructs, models, methods, or instantiations) that address real-world problems \cite{hevner_design_2004}. Unlike natural science, which seeks to understand reality, design science aims to create new and purposeful artifacts to extend human and organizational capabilities.

This paradigm is particularly well-suited for this thesis for several reasons:
\begin{itemize}
    \item \textbf{Problem-Centered:} The research is motivated by a practical and pressing problem—the need for scalable and automated security policy generation in complex cloud environments.
    \item \textbf{Artifact-Oriented:} The primary contributions of this thesis are tangible artifacts: the conceptual framework for GenAI-driven security automation (Chapter \ref{chap:conceptual_framework}) and its implementation as a software prototype (Chapter \ref{chap:implementation}).
    \item \textbf{Evaluation-Focused:} DSR emphasizes the rigorous evaluation of the artifact against defined criteria. This aligns with the research questions concerning the effectiveness, accuracy, and reliability of the proposed solution, which are addressed in Chapter \ref{chap:results}.
\end{itemize}

This research follows the DSR process model proposed by Peffers et al. \cite{peffers_design_2007}, which provides a structured approach for conducting and presenting design science research.

\section{Research Process}
\label{sec:research_process}

The research was conducted in several sequential phases, corresponding to the activities in the DSR process model. This process ensures a logical and rigorous progression from understanding the problem to demonstrating a viable solution.

\subsection{Phase 1: Problem Identification and Motivation}
\label{subsec:problem_identification}
The initial phase involved identifying and defining the research problem. As detailed in Chapter \ref{chap:introduction}, the increasing complexity of hyperscale cloud platforms, combined with the speed of DevOps cycles, creates significant challenges for manual security management. This problem was identified through a preliminary review of industry practices and academic literature, establishing the motivation for developing an automated solution.

\subsection{Phase 2: Definition of Objectives for a Solution}
\label{subsec:objectives_definition}
Based on the problem, a set of objectives for a solution was defined. These objectives were formulated as the research questions in Section \ref{sec:objectives_questions}. The primary goal was to explore how Generative AI could be leveraged to create an automated, accurate, and trustworthy security policy generation system. This phase involved a comprehensive literature review (Chapter \ref{chap:Background and Related Work}) to understand the state-of-the-art and identify existing gaps.

\subsection{Phase 3: Design and Development}
\label{subsec:design_development}
This phase focused on the creation of the research artifact. The design and development process was twofold:
\begin{enumerate}
    \item \textbf{Conceptual Framework:} First, a conceptual framework was designed to provide a high-level architectural blueprint for a GenAI-driven security automation system. This framework, presented in Chapter \ref{chap:conceptual_framework}, outlines the necessary components, layers, and workflows required to address the research objectives.
    \item \textbf{Prototype Implementation:} Second, the conceptual framework was instantiated as a functional software prototype. The implementation, detailed in Chapter \ref{chap:implementation}, serves as a proof-of-concept and provides the means for empirical evaluation. It uses a specific technology stack (\gls{aws}, \gls{terraform}, Python, etc.) to realize the proposed architecture.
\end{enumerate}

\subsection{Phase 4: Demonstration and Evaluation}
\label{subsec:demonstration_evaluation}
The final two phases of the DSR process involve demonstrating the artifact's use and evaluating its performance.
\begin{itemize}
    \item \textbf{Demonstration:} The prototype is used to demonstrate how it can solve the identified problem. This involves running the system with sample \gls{iac} files to automatically generate security policies, as will be shown in the evaluation chapter.
    \item \textbf{Evaluation:} The performance of the prototype is systematically evaluated against the objectives defined in Phase 2. As detailed in Chapter \ref{chap:results}, this involves measuring the quality and efficacy of the generated policies, the speed of the process, and the overall effectiveness of the hybrid analysis model. The evaluation results provide the empirical evidence needed to answer the research questions.
\end{itemize}

By following this structured DSR methodology, the thesis ensures that the proposed solution is not only theoretically sound but also practically validated, thereby providing a rigorous and relevant contribution to the field.
