% Chapter 2

\chapter{Background and Related Work} % (fold)
\label{chap:Background and Related Work}

\section{Foundational Concepts in Cloud Computing} % (fold)
\label{sec:Foundational Concepts in Cloud Computing}

TBD

% section Foundational Concepts in Cloud Computing (end)

\section{State of Cloud Provider Ecosystems} % (fold)
\label{sec:State of Cloud Provider Ecosystems}

TBD

% section State of Cloud Provider Ecosystems (end)

\section{Literature Review} % (fold)
\label{sec:Literature Review}

% TODO rewrite introduction with content done

In this literature review seminal and recent publications addressing the automation of security operations in hyperscale cloud environments through Generative Artificial Intelligence (GenAI) are analyzed. As cloud infrastructures grow increasingly complex, the integration of advanced AI capabilities offers promising solutions for security enhancement and operational efficiency across multi-cloud deployments.

As mentioned in TODO The application of Generative AI to cloud security represents one of the most significant technological shifts in recent years. GenAI's capabilities extend far beyond traditional rule-based security approaches, offering adaptive, intelligent responses to emerging threats. Nevertheless, it introduces significant challenges regarding security enhancements, workload distribution, and cost optimization.

% Seth, Ratra, and Sundareswaran, “AI and Generative AI-Driven Automation for Multi-Cloud and Hybrid Cloud Architectures.”

\subsubsection{Methodology} % (fold)
\label{sec:Methodology}

This literature review followed a structured approach to identify relevant publications, focusing on peer-reviewed articles addressing GenAI applications in hyperscale cloud security published primarily within the last five years. The search utilized academic databases with key search terms related to generative AI, cloud security automation, hyperscale platforms, and multi-cloud orchestration.
Papers were selected based on their relevance to:

\begin{itemize}
\item GenAI applications specifically in cloud security contexts
\item Hyperscale or multi-cloud environments
\item Technical solutions for security automation
\item Empirical evidence or theoretical frameworks with substantial methodological rigor
\end{itemize}

The selection process involved initial screening of titles and abstracts followed by full-text review of promising papers. The analysis employed a thematic approach, identifying recurring concepts, methodological approaches, and gaps in existing research. Particular attention was paid to identifying the theoretical foundations underpinning GenAI applications in security contexts, empirical evidence of effectiveness, and limitations of current approaches.

% subsubsection Methodology (end)

\subsubsection{AI-Driven Security Approaches} % (fold)
\label{sec:AI-Driven Security Approaches}

The landscape of cloud security is undergoing a significant transformation, shifting from traditional, often reactive methods towards more proactive and adaptive strategies powered by Artificial Intelligence (AI), particularly Generative AI (GenAI). This evolution marks a move beyond basic anomaly detection towards sophisticated security postures capable of learning from and responding dynamically to novel threat vectors in complex cloud environments.

Foundational work by Khanna \cite{khanna_enhancing_2024} explores the integration of GenAI into cloud security, outlining its core applications. According to Khanna, modern GenAI implementations focus on key capabilities such as processing vast amounts of data (e.g., log entries, network packets) for advanced anomaly detection and threat intelligence, enabling automated response mechanisms that dynamically adjust security protocols, and facilitating predictive security measures to forecast potential vulnerabilities. While highlighting these advancements, Khanna also acknowledges inherent challenges, including the need for large datasets and mitigating potential adversarial manipulation \cite{khanna_enhancing_2024}.

Building upon these foundational capabilities, the integration of GenAI represents a significant leap beyond conventional rule-based security systems. Research indicates that GenAI enhances security automation, particularly within multi-cloud and hybrid architectures. It allows systems to adapt infrastructure dynamically in response to varying traffic patterns and implements AI-powered defenses against continuously evolving cyber threats. This adaptive capability directly addresses persistent challenges in cloud security related to optimizing workload distribution, ensuring performance, and managing costs effectively \cite{seth_ai_2025}.

Furthermore, recent studies underscore the practical impact of integrating GenAI with established cloud-native security tools. Patel et al. \cite{patel_generative_2025} demonstrate how layering GenAI onto platforms like AWS GuardDuty and Google Cloud Security Command Center significantly boosts automated threat detection, enables real-time incident response, and improves comprehensive vulnerability management across distributed cloud infrastructures. Their work provides empirical evidence, citing examples like Netflix and JPMorgan Chase, which reported measurable improvements in detection accuracy and notable reductions in security incidents following the adoption of GenAI-driven security automation strategies \cite{patel_generative_2025}. This convergence of GenAI with existing security frameworks highlights its potential to transform security operations centers (SOCs) by enhancing both efficiency and effectiveness.
% subsubsection AI-Driven Security Approaches (end)

\subsubsection{GenAI Security Scoping Matrix} % (fold)
\label{sec: GenAI Security Scoping Matrix}

A significant contribution to the field comes from AWS's introduction of the "Generative AI Security Scoping Matrix," which provides a structured approach to assessing security requirements based on the type of GenAI deployment. This framework categorizes deployments into five scopes:
Consumer apps: Public third-party GenAI services
Enterprise apps: Third-party applications with embedded GenAI features
Pre-trained models: Custom applications using existing third-party foundation models
Fine-tuned models: Refined third-party models with business-specific data
Self-trained models: Custom-built and trained models using proprietary data
This matrix helps security teams prioritize focus areas by identifying five key security disciplines that vary across deployment scopes: governance and compliance, legal and privacy, risk management, controls, and resilience.

“Securing Generative AI.”

% subsubsection  GenAI Security Scoping Matrix (end)

\subsubsection{Trust Challenges} % (fold)
\label{sec:Trust Challenges}

A significant challenge in implementing GenAI for security automation relates to the "black box" nature of many large language models (LLMs). Recent research addresses this concern through the Zero-Trust Architecture (ZTA) framework, which is specifically designed to address trust issues with GenAI models.

The ZTA approach acknowledges that GenAI models present unique challenges due to their opaque feature lists and multimodal capabilities. This framework is built on zero-trust principles intended to prevent data breaches, enhance privacy, and restrict internal lateral movement within enterprise environments.

“Zero-Trust Architecture (ZTA): Designing an AI-Powered Cloud Security Framework for LLMs’ Black Box Problems | Semantic Scholar.”

% subsubsection Trust Challenges (end)

\subsubsection{GenAI Security Frameworks} % (fold)
\label{sec:GenAI Security Frameworks}

A notable contribution to the field is the SecGenAI framework, which provides a comprehensive approach to securing cloud-based GenAI applications, with particular attention to Retrieval-Augmented Generation (RAG) systems.

This framework addresses:

Functional, infrastructure, and governance requirements
End-to-end security analysis
Data privacy considerations
Secure deployment methodologies
Shared responsibility models

SecGenAI aligns with Australian Privacy Principles and AI Ethics Principles while mitigating threats such as data leakage, adversarial attacks, and model inversion. Its novel approach combines advanced machine learning techniques with robust security measures.

Haryanto et al., “SecGenAI.”

The development of specialized security frameworks for cloud-based GenAI applications represents another significant advancement in the field. The SecGenAI framework specifically addresses the security requirements of cloud-based GenAI applications, with particular attention to Retrieval-Augmented Generation (RAG) systems. This comprehensive approach incorporates end-to-end security analysis to generate specifications that emphasize data privacy, secure deployment protocols, and shared responsibility models aligned with regulatory requirements.

What distinguishes SecGenAI is its multi-layered approach to security that addresses threats specific to GenAI systems, including data leakage, adversarial attacks, and model inversion techniques. By aligning with Australian Privacy Principles and AI Ethics Principles, the framework demonstrates how regulatory compliance can be integrated into GenAI implementations without compromising operational efficiency.

Haryanto et al., “SecGenAI.”

As organizations increasingly adopt multi-cloud strategies, effective policy orchestration across diverse environments becomes critical for maintaining consistent security postures. A comprehensive analysis of unified AI and cloud platforms published in 2024 examines architectural frameworks and integration patterns that enable the convergence of AI tools, machine learning operations, and workflow orchestration within cloud-native environments.
This research identifies three key innovations with significant implications for security automation: federated AI implementations that preserve data sovereignty across cloud boundaries, real-time data processing architectures that enable immediate threat response, and multi-cloud integration patterns that ensure consistent policy enforcement. These architectural approaches provide the foundation for GenAI-driven security automation across heterogeneous cloud environments.

Sushil Prabhu Prabhakaran, “Integration Patterns in Unified AI and Cloud Platforms.”

For organizations operating containerized workloads across multiple clusters, research from 2023 proposes an automated approach for generating network security policies in multi-domain Kubernetes deployments. This approach addresses two critical challenges in multi-cluster security: reducing configuration errors commonly made by human administrators and creating transparent cross-cluster communications.
The implementation, known as Multi-Cluster Orchestrator, demonstrates how automated policy generation can improve security consistency across distributed environments while reducing the cognitive load on security administrators. This research is particularly relevant for hyperscale cloud platforms that utilize container orchestration technologies like Kubernetes to manage thousands of workloads across multiple regions and availability zones.

Bringhenti, Sisto, and Valenza, “Security Automation for Multi-Cluster Orchestration in Kubernetes.”

A novel approach to security automation involves the use of digital twins for validating security policies before deployment in production environments. This emulation system creates high-fidelity digital replicas of IT infrastructures that enable security teams to safely test security scenarios and develop effective response strategies.
The digital twin approach facilitates a closed-loop learning process where security scenarios generate data that informs Markov decision process simulations, ultimately leading to reinforcement learning of optimal security policies. This methodology provides continuous feedback and improvement cycles that enhance policy effectiveness over time, representing a significant advancement in validation mechanisms for GenAI-driven security automation.

Hammar and Stadler, “Digital Twins for Security Automation.”

Ensuring the trustworthiness and accuracy of GenAI-generated security policies and responses remains a significant challenge. The SecGenAI framework demonstrates how advanced machine learning techniques can be combined with robust security measures to enhance the reliability of GenAI systems while maintaining compliance with regulatory requirements.
This approach integrates continuous validation processes throughout the AI lifecycle, from model development to deployment and monitoring, creating multiple checkpoints that verify the integrity and effectiveness of security responses. By emphasizing explainability alongside accuracy, the framework addresses one of the primary concerns associated with GenAI applications in security contexts: the "black box" nature of complex models.

Haryanto et al., “SecGenAI.”

While not specifically focused on cloud security, research on GenAI applications in the energy sector offers transferable insights into implementation approaches for complex operating environments. This comprehensive literature review identifies how GenAI enhances productivity through data creation, forecasting, optimization, and natural language understanding, while also addressing challenges such as hallucinations, data biases, privacy concerns, and system errors.
The proposed solutions—including improving training data quality, implementing system fine-tuning processes, establishing human oversight mechanisms, and deploying robust security measures—provide a valuable framework for GenAI implementations in cloud security contexts. These approaches are particularly relevant for hyperscale environments where scale and complexity amplify both the benefits and risks of GenAI adoption.

Surathunmanun, Ongsakul, and Singh, “Exploring the Role of Generative Artificial Intelligence in the Energy Sector.”

% subsubsection GenAI Security Frameworks (end)

\subsubsection{Agent-Based Approaches} % (fold)
\label{sec:Agent-Based Approaches}

Recent research introduces the concept of GenAI-driven agentic workflows for comprehensive security. The DevSecOps Sentinel system employs intelligent agents to improve software supply chain security holistically.

This approach demonstrates that:

Agentic workflows powered by GenAI are viable for tackling intricate security issues
Integration of AI analysis capability with agentic systems strengths offers a way forward for organizations
Such systems can help build software that is simultaneously faster, safer, and more reliable

“DevSecOps Sentinel: GenAI-Driven Agentic Workflows for Comprehensive Supply Chain Security | Semantic Scholar.”

% subsubsection Agent-Based Approaches (end)

\subsubsection{GenAI Security Infrastructure} % (fold)
\label{sec:GenAI Security Infrastructure}

The evolution of security infrastructure to accommodate GenAI capabilities is evident in research on next-generation firewalls that incorporate machine learning and generative modeling for enhanced threat detection. These advanced systems integrate security controls and protocols at Layer 7 of the OSI model, representing a significant leap forward in perimeter security technology for cloud environments. This integration enables the detection of sophisticated attack patterns that traditional signature-based approaches might miss.

Patel et al., “Generative AI for Automated Security Operations in Cloud Computing.”

Despite the transformative potential of GenAI in cloud security, research published in 2023 indicates that organizations often implement these technologies without adequately assessing potential security vulnerabilities. This observation underscores the need for balanced approaches that embrace technological advancement while implementing robust security practices and governance frameworks.

Lekkala, “Next-Gen Firewalls.”

% subsubsection GenAI Security Infrastructure (end)

\subsubsection{Privacy and Regulatory compliance} % (fold)
\label{sec:Privacy and Regulatory compliance}

A significant challenge in implementing GenAI for security automation is ensuring compliance with evolving regulatory frameworks. Research indicates the need for clear guidelines regarding intellectual property ownership rules, particularly concerning AI-created works and the legal status of data used to train AI models.

The Association for Computing Machinery (ACM) urges that personal data used to generate information or train models should be subject to opt-out policies, and AI creators should maintain records of errors made by their systems to ensure transparency about accuracy and correctability.

Tabassi, “Artificial Intelligence Risk Management Framework (AI RMF 1.0).”

The literature review on GenAI in the energy sector identifies key challenges that are equally applicable to cloud security implementations: hallucinations (generating plausible but incorrect information), data biases that affect model outputs, privacy concerns related to training data, potential for misuse, and system errors that may propagate through automated processes.
Addressing these challenges requires comprehensive approaches to data governance, model validation, and continuous monitoring that ensure GenAI systems operate within acceptable parameters. These considerations are particularly important in security contexts, where false positives or missed detections can have significant consequences for organizational risk posture.

Surathunmanun, Ongsakul, and Singh, “Exploring the Role of Generative Artificial Intelligence in the Energy Sector.”

% subsubsection Privacy and Regulatory compliance (end)

\subsubsection{Security Risks} % (fold)
\label{sec:Security Risks}

Research by Khanna et al. identifies several cybersecurity risks arising from the use of GenAI, including:

Phishing attacks and social engineering
Ransomware and malware generation
Deepfakes and misinformation
Data leakage and misuse of personal data
Executable attack code generation
Privacy risks and intellectual property violations

These findings provide critical insights into potential threats from irresponsible use of GenAI and emphasize the need for risk mitigation efforts and regulations concerning ethical use.

Nyoto, Devega, and Nyoto, “Cyber Security Risks in the Rapid Development of Generative Artificial Intelligence.”

Implementing GenAI across hyperscale cloud environments introduces additional challenges related to model distribution, data synchronization, and consistent policy enforcement across regions and services. While not explicitly addressed in all the available research, these scaling considerations represent significant technical hurdles for organizations operating at hyperscale.
The cautionary note about adopting GenAI without carefully considering potential security vulnerabilities underscores the need for comprehensive risk assessment and gradual implementation approaches that allow organizations to identify and address issues before full-scale deployment

Weedon, “Generative AI.”

% subsubsection Security Risks (end)

\subsubsection{Balance of Automation and Human Oversight} % (fold)
\label{sec:Balance of Automation and Human Oversight}

A recurring theme in the literature is the tension between the benefits of automation and the necessity of human oversight. While AI-powered security automation safeguards against evolving cyber dangers, research suggests that human expertise remains essential, particularly for high-risk AI deployments.

Seth, Ratra, and Sundareswaran, “AI and Generative AI-Driven Automation for Multi-Cloud and Hybrid Cloud Architectures.”

The ACM explicitly states that no "high-risk" AI should be operated without substantial human oversight and careful deliberation over whether benefits outweigh potential negative impacts.

Tabassi, “Artificial Intelligence Risk Management Framework (AI RMF 1.0).”

Despite the promising applications of GenAI for security automation, significant challenges remain in balancing automation with appropriate human oversight. Research on GenAI for automated security operations highlights issues such as over-dependence on AI tools, adversarial risks to models, and the complex nature of decision-making in AI systems.
The study emphasizes the importance of preventive efforts and planned action plans to manage these technologies efficiently, recognizing that complete automation without human intervention introduces unacceptable risks in security contexts. This balanced approach acknowledges the complementary strengths of human expertise and AI capabilities in addressing complex security challenges.

Patel et al., “Generative AI for Automated Security Operations in Cloud Computing.”

% subsubsection Balance of Automation and Human Oversight (end)

% section Literature Review (end)

\section{Research Gaps} % (fold)
\label{sec:Research Gaps}

Enhanced validation mechanisms: Developing more robust techniques for verifying the accuracy and reliability of GenAI security decisions, moving beyond current red-teaming approaches

Feffer et al., “Red-Teaming for Generative AI.”

Cross-platform orchestration: Creating unified frameworks for consistent security policy application across diverse cloud environments

Vootkuri, “Multi-Cloud Data Strategy Security for Generative AI.”

Domain-specific LLMs for security: Exploring purpose-built language models optimized for security applications rather than general-purpose models

Energy-efficient security operations: Developing approaches that balance computational demands with sustainability concerns, particularly for inference operations

Multi-disciplinary approaches: Bridging the gap between scientific developments and ethical considerations through collaborative research involving computer science, law, ethics, and policy-making experts

Yigit et al., “Review of Generative AI Methods in Cybersecurity.”

Standardized Evaluation Frameworks
The analysis of current literature reveals a significant need for standardized frameworks to evaluate the effectiveness and security of GenAI-driven automation in hyperscale cloud environments. Future research should focus on developing metrics and methodologies that enable consistent assessment of GenAI implementations across different cloud providers and security contexts.
Hybrid Security Approaches
Promising directions for future research include the investigation of hybrid approaches that combine GenAI with traditional security methods to leverage the strengths of both paradigms. These hybrid models could provide the adaptability and pattern recognition capabilities of GenAI while maintaining the explainability and predictability of rule-based systems in critical security functions.

Explainable AI for Security Operations
Research on explainable AI approaches specifically tailored to security operations could increase transparency and trust in GenAI-generated security policies and decisions. This focus area is particularly important for regulatory compliance and stakeholder confidence in automated security systems.

\subsubsection{Summary Literature review} % (fold)
\label{sec:Summary Literature review}

This systematic review demonstrates that GenAI represents a transformative technology for security automation in hyperscale cloud environments. The literature reveals significant advancements in conceptual frameworks, validation mechanisms, and technical implementations, alongside persistent challenges related to trust, data privacy, and the balance between automation and human oversight.

The most promising approaches leverage multi-cloud strategies, zero-trust architectures, and comprehensive security frameworks while acknowledging the unique infrastructure requirements of GenAI at scale. As this field continues to evolve, interdisciplinary collaboration will be essential to develop ethical norms and innovative defense mechanisms that address current issues while guiding the responsible application of GenAI in cybersecurity.

This systematic literature review has examined the current state of research on applying GenAI to security automation in hyperscale cloud platforms. The analysis reveals significant potential for GenAI to enhance security operations through automated threat detection, policy generation, and incident response across complex multi-cloud environments.
Emerging conceptual frameworks for multi-cloud policy orchestration, validation mechanisms for ensuring trust and accuracy, and technical approaches for implementing GenAI in hyperscale environments demonstrate the rapid evolution of this field. However, challenges related to trust, validation, data quality, and human oversight remain significant considerations that must be addressed for successful deployment of GenAI-driven security automation.

% subsubsection Summary Literature review (end)

% section Research Gaps (end)

% chapter Background and Related Work (end)
